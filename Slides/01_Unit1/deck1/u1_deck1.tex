% Compile with XeLaTeX

%%%%%%%%%%%%%%%%%%%%%%%
% Option 1: Slides: (comment for handouts)   %
%%%%%%%%%%%%%%%%%%%%%%%

\documentclass[slidestop,compress,mathserif,12pt,t,professionalfonts,xcolor=table]{beamer}

% solution stuff
\newcommand{\solnMult}[1]{
\only<1>{#1}
\only<2->{\red{\textbf{#1}}}
}
\newcommand{\soln}[1]{\textit{#1}}

%%%%%%%%%%%%%%%%%%%%%%%%%%%%%%%
% Option 2: Handouts, without solutions (post before class)    %
%%%%%%%%%%%%%%%%%%%%%%%%%%%%%%%

%\documentclass[11pt,containsverbatim,handout,xcolor=xelatex,dvipsnames,table]{beamer}
%
%% handout layout
%\usepackage{pgfpages}
%\pgfpagesuselayout{4 on 1}[letterpaper,landscape,border shrink=5mm]
%
%% solution stuff
%\newcommand{\solnMult}[1]{#1}
%\newcommand{\soln}[1]{}
%
%% tell pgfpages how to set page sizes in XeLaTeX
%\renewcommand\pgfsetupphysicalpagesizes{%
%    \pdfpagewidth\pgfphysicalwidth\pdfpageheight\pgfphysicalheight%
%}

%%%%%%%%%%%%%%%%%%%%%%%%%%%%%%%%%%%%
% Option 3: Handouts, with solutions (may post after class if need be)    %
%%%%%%%%%%%%%%%%%%%%%%%%%%%%%%%%%%%%

%\documentclass[11pt,containsverbatim,handout,xcolor=xelatex,dvipsnames,table]{beamer}

%% handout layout
%\usepackage{pgfpages}
%\pgfpagesuselayout{4 on 1}[letterpaper,landscape,border shrink=5mm]

%% solution stuff
%\newcommand{\solnMult}[1]{\red{\textbf{#1}}}
%\newcommand{\soln}[1]{\textit{#1}}

%% tell pgfpages how to set page sizes in XeLaTeX
%\renewcommand\pgfsetupphysicalpagesizes{%
%    \pdfpagewidth\pgfphysicalwidth\pdfpageheight\pgfphysicalheight%
%}

%%%%%%%%%%
% Load style file   %
%%%%%%%%%%

\input{../../lec_style.tex}


%%%%%%%%%%%
% Cover slide info    %
%%%%%%%%%%%

\title{Unit 1: Introduction to data}
\subtitle{1. Data Collection +\\Observational studies \& experiments}
\author{Sta 101 - Spring 2015}
\date{January 12, 2015}
\institute{Duke University, Department of Statistical Science}


%%%%%%%%%%%%%%%%%%%%%%%%%
% Begin document and set Helvetica Neue font   %
%%%%%%%%%%%%%%%%%%%%%%%%%

\begin{document}
\fontspec[Ligatures=TeX]{Helvetica Neue Light}

%%%%%%%%%%%%%%%%%%%%%%%%%%%%%%%%%%%

% Title Page

\begin{frame}[plain]

\titlepage
\vfill
{\scriptsize \webLink{http://stat.duke.edu/~mc301}{Dr. \c{C}etinkaya-Rundel} \hfill Slides posted at  \webLink{http://bitly.com/sta101sp15}{bitly.com/sta101sp15}}
\addtocounter{framenumber}{-1} 

\end{frame}

%%%%%%%%%%%%%%%%%%%%%%%%%%%%%%%%%%%

\section{Readiness assessment}

%%%%%%%%%%%%%%%%%%%%%%%%%%%%%%%%%%%

\begin{frame}
\frametitle{Readiness assessment}

\begin{itemize}

\item \hl{Individual:} 15 minutes, using clickers

\end{itemize}

\begin{center}
\includegraphics[width=0.3\textwidth]{figures/clicker_self_paced/self_paced_1}
\hspace{1mm}
\includegraphics[width=0.3\textwidth]{figures/clicker_self_paced/self_paced_2}
\hspace{1mm}
\includegraphics[width=0.3\textwidth]{figures/clicker_self_paced/self_paced_3} \\
\includegraphics[width=0.3\textwidth]{figures/clicker_self_paced/self_paced_4}
\hspace{1mm}
\includegraphics[width=0.3\textwidth]{figures/clicker_self_paced/self_paced_5}
\end{center}

\begin{itemize}

\item \hl{Team:} 10 minutes, using scratch off sheets (1 per team)

\end{itemize}

\end{frame}

%%%%%%%%%%%%%%%%%%%%%%%%%%%%%%%%%%%

\section{Housekeeping}

%%%%%%%%%%%%%%%%%%%%%%%%%%%%%%%%%%%

\begin{frame}
\frametitle{Announcements}

\begin{itemize}

\item PS 1 due Wednesday on Sakai, by the beginning of class

\item Lab tomorrow, sit with your teams

\end{itemize}

\end{frame}

%%%%%%%%%%%%%%%%%%%%%%%%%%%%%%%%%%%

\section{Main ideas}

%%%%%%%%%%%%%%%%%%%%%%%%%%%%%%%%%%%

\subsection{Use a sample to make inferences about the population}
\label{mi1}

%%%%%%%%%%%%%%%%%%%%%%%%%%%%%%%%%%%

\begin{frame}
\frametitle{1. Use a sample to make inferences about the population}

\begin{itemize}[<+->]
\item Our ultimate goal is to make inferences about populations

\item However populations are difficult or impossible to access

\item Therefore we use a sample from that population, and use \hl{statistics} from that sample to make inferences about the unknown population \hl{parameters}
\begin{itemize}
\item We want to know how many offspring female lemurs have, on average
\item It's not feasible to obtain offspring data from on all female lemurs, so we use data from the Duke Lemur Center
\item We use the sample mean from these data as an estimate for the unknown population mean
\end{itemize}

\item The better (more \hl{representative}) sample we have, the more reliable our estimates and more accurate our inferences will be

\end{itemize}

\pause

\disc{Can you see any limitations to using data from the Duke Lemur Center to make inferences about all lemurs?}

\end{frame}

%%%%%%%%%%%%%%%%%%%%%%%%%%%%%%%%%%%

\begin{frame}
\frametitle{Sampling is natural}

\begin{center}
\includegraphics[width=0.3\textwidth]{figures/soup}
\end{center}

\begin{itemize}

\item When you taste a spoonful of soup and decide the spoonful you tasted isn't salty enough, that's \hl{exploratory analysis}

\item If you generalize and conclude that your entire soup needs salt, that's an \hl{inference}

\item For your inference to be valid, the spoonful you tasted (the sample) needs to be \hl{representative} of the entire pot (the population)

\end{itemize}

\end{frame}

%%%%%%%%%%%%%%%%%%%%%%%%%%%%%%%%%%%

\subsection{Ideally use a simple random sample, stratify to control for a variable, and cluster to make sampling easier} 
\label{mi2}

%%%%%%%%%%%%%%%%%%%%%%%%%%%%%%%%%%%

\begin{frame}
\frametitle{2. Ideally use a simple random sample, stratify to control for a variable, and cluster to make sampling easier}

\begin{center}
\textbf{Demo:} \webURL{http://bl.ocks.org/avimoondra}
\end{center}

\begin{itemize}
\item \hl{Simple random sampling:} Randomly select cases from the population, each case is equally likely to be selected

\item \hl{Stratified sampling:} First divide the population into homogenous \hl{strata}, then randomly sample from \underline{each} stratum
\begin{itemize}
\item e.g. Stratify to control for socio-economic status
\end{itemize}

\item \hl{Cluster sampling:} First randomly sample \underline{a few} clusters, then randomly sample from within them
\begin{itemize}
\item \hl{Clusters} are not necessarily homogenous, but ideally they're not too different from each other
\item e.g. First sample a few schools from a school district, and then only sample students from within those schools
\item Usually preferred for economical reasons
\end{itemize}

\end{itemize}

\end{frame}

%%%%%%%%%%%%%%%%%%%%%%%%%%%%%%%%%%%

\begin{frame}

\clicker{A city council has requested a household survey be conducted in a suburban area of their city. The area is broken into many distinct and unique neighborhoods, some including large homes, some with only apartments, and others a diverse mixture of housing structures. Which approach would likely be the \emph{least} effective?}

\begin{enumerate}[(a)]
\item Simple random sampling
\item Stratified sampling, where each cluster is a neighborhood
\item \solnMult{Cluster sampling, where each cluster is a neighborhood}
\end{enumerate}

\end{frame}

%%%%%%%%%%%%%%%%%%%%%%%%%%%%%%%%%%%

\subsection{Sampling schemes can suffer from a variety of biases}
\label{mi3}

%%%%%%%%%%%%%%%%%%%%%%%%%%%%%%%%%%%

\begin{frame}
\frametitle{3. Sampling schemes can suffer from a variety of biases}

\begin{itemize}[<+->]

\item \hl{Non-response:} If only a small fraction of the randomly sampled people choose to respond to a survey, the sample may no longer be representative of the population

\item \hl{Voluntary response:} Occurs when the sample consists of people who volunteer to respond because they have strong opinions on the issue since such a sample will also not be representative of the population

\item \hl{Convenience sample:} Individuals who are easily accessible are more likely to be included in the sample

\end{itemize}

\end{frame}

%%%%%%%%%%%%%%%%%%%%%%%%%%%%%%%%%%%

\begin{frame}[shrink]

{\small
\clicker{A school district is considering whether it will no longer allow high school students to park at school after two recent accidents where students were severely injured. As a first step, they survey parents by mail, asking them whether or not the parents would object to this policy change. Of 6,000 surveys that go out, 1,200 are returned. Of these 1,200 surveys that were completed, 960 agreed with the policy change and 240 disagreed. Which of the following statements are true?}

\begin{enumerate}[I.]
\item Some of the mailings may have never reached the parents.
\item Overall, the school district has strong support from parents to move forward with the policy approval.
\item It is possible that majority of the parents of high school students disagree with the policy change.
\item The survey results are unlikely to be biased because all parents were mailed a survey. 
\end{enumerate}

\begin{multicols}{5}
\begin{enumerate}[(a)]
\item Only I
\item I and II
\item \solnMult{I and III}
\item III and IV
\item Only IV
\end{enumerate}
\end{multicols}
}

\end{frame}

%%%%%%%%%%%%%%%%%%%%%%%%%%%%%%%%%%%%

\subsection{Experiments use random assignment to treatment groups, observational studies do not}
\label{mi4}

%%%%%%%%%%%%%%%%%%%%%%%%%%%%%%%%%%%%

\begin{frame}
\frametitle{}

\disc{What type of study is this? What is the scope of inference (causality / generalizability)?}

\begin{center}
\includegraphics[width=0.9\textwidth]{figures/facebook_study}
\end{center}

\ct{\webURL{http://www.nytimes.com/2014/06/30/technology/facebook-tinkers-with-users-emotions-in-news-feed-experiment-stirring-outcry.html}}

\end{frame}

%%%%%%%%%%%%%%%%%%%%%%%%%%%%%%%%%%%%%

\begin{frame}
\frametitle{4. Experiments use random assignment to treatment groups, observational studies do not}

{\small
\disc{A study that surveyed a random sample of otherwise healthy adults found that people are more likely to get muscle cramps when they're stressed. The study also noted that people drink more coffee and sleep less when they're stressed. What type of study is this?}

\soln{\onslide<2->{Observational}}

\disc{What is the conclusion of the study?}

\soln{\onslide<3->{There is an \hl{association} between increased stress \& muscle cramps.}}

\disc{Can this study be used to conclude a causal relationship between increased stress and muscle cramps?}

\soln{\onslide<4->{Muscle cramps might also be due to increased caffeine consumption or sleeping less -- these are potential \hl{confounding} variables.}}
}

\end{frame}

%%%%%%%%%%%%%%%%%%%%%%%%%%%%%%%%%%%%

\subsection{Four principles of experimental design: randomize, control, block, replicate}
\label{mi5}

%%%%%%%%%%%%%%%%%%%%%%%%%%%%%%%%%%%%

\begin{frame}
\frametitle{5. Four principles of experimental design:\\ randomize, control, block, replicate}

\begin{itemize}
\item We would like to design an experiment to investigate if increased stress causes muscle cramps:

\pause

\begin{itemize}
\item Treatment: increased stress
\item Control: no or baseline stress
\end{itemize}

\pause

\item It is suspected that the effect of stress might be different on younger and older people: \hl{block} for age.

\end{itemize}

\pause

\disc{Why is this important? Can you think of other variables to block for?}

\pause

\begin{center}
\textbf{Demo:} \webURL{http://bl.ocks.org/avimoondra}
\end{center}

\end{frame}

%%%%%%%%%%%%%%%%%%%%%%%%%%%%%%%%%%%

\subsection{Random sampling helps generalizability, random assignment helps causality}
\label{mi6}

%%%%%%%%%%%%%%%%%%%%%%%%%%%%%%%%%%%%

\begin{frame}
\frametitle{6. Random sampling helps generalizability,\\ random assignment helps causality}

\begin{center}
\includegraphics[width=\textwidth]{figures/random_sample_assignment}
\end{center}

\end{frame}

%%%%%%%%%%%%%%%%%%%%%%%%%%%%%%%%%%%

\begin{frame}

\app{1.1 Scientific studies in the press}
{{\small Read media coverage of a study titled ``Haters Are Gonna Hate, Study Confirms" and answer the following questions. If the relevant information isn't in the article, refer to the original study.}}

\twocol{0.75}{0.25}{
{\footnotesize
\begin{enumerate}
\item What are the cases?
\item What is (are) the response variable(s) in this study?
\item What is (are) the explanatory variable(s) in this study?
\item Does the study employ random sampling? How about random assignment?
\item Is this an observational study or an experiment? Explain your reasoning.
\item Can we establish a causal link between the explanatory and response variables?
\item Can the results of the study be generalized to the population at large?
\end{enumerate}
}
}
{
\begin{center}
\includegraphics[width=\textwidth]{figures/haters}
\end{center}
}

\end{frame}

%%%%%%%%%%%%%%%%%%%%%%%%%%%%%%%%%%%

\section{Summary}

%%%%%%%%%%%%%%%%%%%%%%%%%%%%%%%%%%%

\begin{frame}
\frametitle{Summary of main ideas}

\vfill

\begin{enumerate}

\item \nameref{mi1}

\item \nameref{mi2}

\item \nameref{mi3}

\item \nameref{mi4}

\item \nameref{mi5}

\item \nameref{mi6}

\end{enumerate}

\vfill

\end{frame}

%%%%%%%%%%%%%%%%%%%%%%%%%%%%%%%%%%%

\end{document}