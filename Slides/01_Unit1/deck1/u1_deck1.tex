% -*- TeX-engine: xetex; eval: (auto-fill-mode 0); eval: (visual-line-mode 1); -*-
% Compile with XeLaTeX

%%%%%%%%%%%%%%%%%%%%%%%
% To do before class
%%%%%%%%%%%%%%%%%%%%%%%

% Print off Readiness Assessment 1

% Send email about registering clicker.
% Test run readiness assessment on iClicker.
% I need to get scratch off sheet from Mine.

% Send the Logistics/Week0Annoucnement (the night before).
% Send an email reminding students to bring a charged computer (the night before).

% Questions for Mine
% Can I get scratch off sheets.
% What do you do during group portion?
% Question: voluntary vs. non-response

%%%%%%%%%%%%%%%%%%%%%%%
% Option 1: Slides: (comment for handouts)   %
%%%%%%%%%%%%%%%%%%%%%%%

\documentclass[slidestop,compress,mathserif,12pt,t,professionalfonts,xcolor=table]{beamer}

% solution stuff
\newcommand{\solnMult}[1]{
\only<1>{#1}
\only<2->{\red{\textbf{#1}}}
}
\newcommand{\soln}[1]{\textit{#1}}

%%%%%%%%%%%%%%%%%%%%%%%%%%%%%%%
% Option 2: Handouts, without solutions (post before class)    %
%%%%%%%%%%%%%%%%%%%%%%%%%%%%%%%

% \documentclass[11pt,containsverbatim,handout,xcolor=xelatex,dvipsnames,table]{beamer}

% % handout layout
% \usepackage{pgfpages}
% \pgfpagesuselayout{4 on 1}[letterpaper,landscape,border shrink=5mm]

% % solution stuff
% \newcommand{\solnMult}[1]{#1}
% \newcommand{\soln}[1]{}

% % % This breaks things for me for some reason.
% % tell pgfpages how to set page sizes in XeLaTeX
% %\renewcommand\pgfsetupphysicalpagesizes{%
% %   \pdfpagewidth\pgfphysicalwidth\pdfpageheight\pgfphysicalheight%
% %}

%%%%%%%%%%%%%%%%%%%%%%%%%%%%%%%%%%%%
% Option 3: Handouts, with solutions (may post after class if need be)    %
%%%%%%%%%%%%%%%%%%%%%%%%%%%%%%%%%%%%

% \documentclass[11pt,containsverbatim,handout,xcolor=xelatex,dvipsnames,table]{beamer}

% % handout layout
% \usepackage{pgfpages}
% \pgfpagesuselayout{4 on 1}[letterpaper,landscape,border shrink=5mm]

% % solution stuff
% \newcommand{\solnMult}[1]{\red{\textbf{#1}}}
% \newcommand{\soln}[1]{\textit{#1}}

% % % This breaks things for me for some reason.
% % % tell pgfpages how to set page sizes in XeLaTeX
% % \renewcommand\pgfsetupphysicalpagesizes{%
% %    \pdfpagewidth\pgfphysicalwidth\pdfpageheight\pgfphysicalheight%
% % }

%%%%%%%%%%%%%%%%%%%%%%%%%%%%%%%
% Option 4: Notes Only
%%%%%%%%%%%%%%%%%%%%%%%%%%%%%%%

% % See http://tex.stackexchange.com/questions/114219/add-notes-to-latex-beamer
% \documentclass[10pt,containsverbatim,xcolor=xelatex,dvipsnames,table,notes=only]{beamer}

% % handout layout
% \usepackage{pgfpages}
% \pgfpagesuselayout{2 on 1}[letterpaper, landscape, border shrink=5mm]

% % solution stuff
% \newcommand{\solnMult}[1]{#1}
% \newcommand{\soln}[1]{}

% % % Having a problem with this.
% % tell pgfpages how to set page sizes in XeLaTeX
% % \renewcommand\pgfsetupphysicalpagesizes{%
% %   \pdfpagewidth\pgfphysicalwidth\pdfpageheight\pgfphysicalheight%
% %}

%%%%%%%%%%
% Load style file, defaults  %
%%%%%%%%%%

\input{../../lec_style.tex}
% You cannot use numbers when defining variables.  Hence the use of letters, A, B, C, etc.

% Personal Info
\newcommand{\FirstName}{Mine}
\newcommand{\LastName}{\c{C}etinkaya-Rundel}
\newcommand{\OfficeHours}{MTWR 3-4pm.}
\newcommand{\OfficeHoursLocation}{Old Chem 213}

% Electronic Info
\newcommand{\PersonalSite}{http://stat.duke.edu/~mc301}
\newcommand{\CourseSite}{http://bitly.com/sta101sp15}
\newcommand{\Email}{mine@stat.duke.edu}

% TAs
\newcommand{\TAA}{Anthony Weishampel}
\newcommand{\TAB}{Fiamma Li}
\newcommand{\TAC}{Jialiang Mao}
\newcommand{\TAD}{Phillip Lee}

% Exam Dates
\newcommand{\ExamADate}{Wed, Feb 18}
\newcommand{\ExamBDate}{Wed, Mar 25}
\newcommand{\FinalDate}{Sat, May 2 (2-5pm)}

% Due Dates
\newcommand{\ClickerRegistrationDD}{Mon, Jan 26}
\newcommand{\GettingToKnowYouDD}{Friday, Jan 9, 11:59pm}
\newcommand{\ProblemSetADD}{Wed., 1/15}


% ALT ALT
% % You cannot use numbers when defining variables.  Hence the use of letters, A, B, C, etc.

% Personal Info
\renewcommand{\FirstName}{Jesse}
\renewcommand{\LastName}{Windle}
\renewcommand{\OfficeHours}{Tue, Thu 3:00pm-4:30pm}
\renewcommand{\OfficeHoursLocation}{Old Chem 211D}

% Electronic Info
\renewcommand{\PersonalSite}{http://stat.duke.edu/~jbw44/}
\renewcommand{\CourseSite}{http://bitly.com/windle2}
\renewcommand{\Email}{jbw44@stat.duke.edu}

% TAs
\renewcommand{\TAA}{David Clancy}
\renewcommand{\TAB}{Xinyi (Chris) Li}
\renewcommand{\TAC}{Tori Hall}
\renewcommand{\TAD}{Radhika Anand}

% Exam Dates
\renewcommand{\ExamADate}{Thu, Feb 19}
\renewcommand{\ExamBDate}{Thu, Mar 26}
\renewcommand{\FinalDate}{Mon, Apr 27 (9-Noon)}

% Due Dates
\renewcommand{\ClickerRegistrationDD}{Thu, Jan 15}
\renewcommand{\GettingToKnowYouDD}{Friday, Jan 9, 11:59pm}
\renewcommand{\ProblemSetADD}{Thu., 1/16}

%%%%%%%%%%%
% Cover slide info    %
%%%%%%%%%%%

\title{Unit 1: Introduction to data}
\subtitle{1. Data Collection +\\Observational studies \& experiments}
\author{Sta 101 - Spring 2015}
\date{January 12, 2015}
% ALT ALT
% \date{January 13, 2015}
\institute{Duke University, Department of Statistical Science}


%%%%%%%%%%%%%%%%%%%%%%%%%
% Begin document and set Helvetica Neue font   %
%%%%%%%%%%%%%%%%%%%%%%%%%

\begin{document}
\fontspec[Ligatures=TeX]{Helvetica Neue Light}

%%%%%%%%%%%%%%%%%%%%%%%%%%%%%%%%%%%

% Title Page

\begin{frame}[plain]

\titlepage
\vfill
{\scriptsize \webLink{\PersonalSite}{Dr. \LastName{}} \hfill Slides posted at  \webLink{\CourseSite}{\CourseSite}}
\addtocounter{framenumber}{-1} 

\end{frame}

%%%%%%%%%%%%%%%%%%%%%%%%%%%%%%%%%%%

\section{Readiness assessment}

%%%%%%%%%%%%%%%%%%%%%%%%%%%%%%%%%%%

\begin{frame}
\frametitle{Readiness assessment}

\begin{itemize}

\item \hl{Individual:} 15 minutes, using clickers

\end{itemize}

\begin{center}
\includegraphics[width=0.3\textwidth]{figures/clicker_self_paced/self_paced_1}
\hspace{1mm}
\includegraphics[width=0.3\textwidth]{figures/clicker_self_paced/self_paced_2}
\hspace{1mm}
\includegraphics[width=0.3\textwidth]{figures/clicker_self_paced/self_paced_3} \\
\includegraphics[width=0.3\textwidth]{figures/clicker_self_paced/self_paced_4}
\hspace{1mm}
\includegraphics[width=0.3\textwidth]{figures/clicker_self_paced/self_paced_5}
\end{center}

\begin{itemize}

\item \hl{Team:} 10 minutes, using scratch off sheets (1 per team)

\end{itemize}

%---Note---%
\note{

Explain individual RA/clickers:
\begin{itemize}
\item Set frequency (hold down power, click AA).

\item Normally, I'd ask you to put everything away.

\item Normally, 15 minutes, today a little onger.

\item Click on blue button.  You should see square brackets.  Click on A, check
mark.  Your answer has been sent.  Go up down to change question.

\item If you don't see brackets click on blue button.

\item Whether you are registered are not should not matter.

\item You might have questions.  Raise hand and someone will help you.

\item When you are done, turn your clicker over and set it on your desk.

\item Write your name on the piece of paper and circle your answer on the piecen of paper.  This is the only record I will have if we encounter technical difficulties.

\end{itemize}

- Informally put together teams during clicker.
- About 8 minutes in, say if you are done, turn over your clicker.
- Stop once people are done.
- Pause individual portion.

Explain teams/team portion of RA.
\begin{itemize}
\item Normally, you would be with team, today you are with impromptu teams.

\item It would be too hard for everyone to find each other today.  Tomorrow in lab you will be put into teams.

\item I want you to sit with your team in class.

\end{itemize}

Explain scratch sheet:
\begin{itemize}
\item Use doc cam.

\item For the team portion you will use scratch out sheets.

\item Scratch once, if correct good.  Scratch again for second try.

\item 10 minutes for group scratching. (Set up timer.)

\item When you are done, I will pick up your scratch off sheet AND your paper readiness assessment.

\end{itemize}

Give 1 minute warning:
\begin{itemize}
\item You have one minute.

\item When time is up if you haven't given me your scratch off or readiness assessment, walk it up to the front and set it on the table.

\end{itemize}

While the team portion of the RA is going:
\begin{itemize}

\item Write percentage next to each question and circle correct answer.  Usually, pick off two or three questions to go over.

\item pick a couple to go over.

\item read the question back to them.

\item tell the students which learning objectives each question covered.

\end{itemize}

Question 1: types of variables
\begin{itemize}
\item first learning objective: describing variables.

\item question 1: populations can only take on whole numbers.  Can you yave 1000.5 people?

\end{itemize}

Question 5: skewed distributions
\begin{itemize}
\item we want to figure out shape of distribution: symmetric, skewed

\item It really helps to draaw things out.  Use doc cam.  Minimum amount 0:
  boundary.  put 50k marks, try to draw to scale.  What is the total area under
  the curve?  100\%.  Break into quarters.  Are there any natural boundaries?
  Costs, salaries, etc.

\end{itemize}


Question 10: randomization test.
\begin{itemize}

\item Just based on data what is it pointing to?  Are people with pets more less likely to be stressed?

\item But this is just true of sample.  Want to say something about the
entire population.  One method is simulation based method.  Randomization.  Last
video you watched.

\item Total of 40 cards.  18 cards.  Wrote word overstressed on 18.  Shuffle.
Randomly split into two groups of 20.  Do I have any control over how many went
to one group or another.  

\item What would I expect the number to be in each group.

\item What would the expected difference be?  0. 

\item If you expect 0 but you see 2 is that something that is due to chance? What we will learn is whether difference of 2 is statistically significant.  This is what we are learning in this class?  Is this difference suprisingly large?

\end{itemize} 

}

\end{frame}

%%%%%%%%%%%%%%%%%%%%%%%%%%%%%%%%%%%

\section{Housekeeping}

%%%%%%%%%%%%%%%%%%%%%%%%%%%%%%%%%%%

\begin{frame}
\frametitle{Announcements}

\begin{itemize}

\item PS 1 due \ProblemSetADD{} on Sakai, by the beginning of class

\item Lab tomorrow, sit with your teams

\item My office hours: \OfficeHours{} at \OfficeHoursLocation{}
% ALT ALT

% \item This week my Thursday Office Hours will be from after class to 3:30PM.

\item Class sessions recorded via Duke Capture (link emailed)
% ALT ALT

\end{itemize}

%---Note---%
\note{

  Read bullet points.  

  Emphasize that tomorrow they are sitting in teams!  And in future classes they will be sitting in teams.

}

\end{frame}

%%%%%%%%%%%%%%%%%%%%%%%%%%%%%%%%%%%

\section{Main ideas}

%%%%%%%%%%%%%%%%%%%%%%%%%%%%%%%%%%%

\subsection{Use a sample to make inferences about the population}
\label{mi1}

%%%%%%%%%%%%%%%%%%%%%%%%%%%%%%%%%%%

\begin{frame}
\frametitle{1. Use a sample to make inferences about the population}

\begin{itemize}[<+->]
\item Our ultimate goal is to make inferences about populations

\item However populations are difficult or impossible to access

\item Therefore we use a sample from that population, and use \hl{statistics} from that sample to make inferences about the unknown population \hl{parameters}
\begin{itemize}
\item We want to know how many offspring female lemurs have, on average
\item It's not feasible to obtain offspring data from on all female lemurs, so we use data from the Duke Lemur Center
\item We use the sample mean from these data as an estimate for the unknown population mean
\end{itemize}

\item The better (more \hl{representative}) sample we have, the more reliable our estimates and more accurate our inferences will be

\end{itemize}

\pause

\disc{Can you see any limitations to using data from the Duke Lemur Center to
  make inferences about all lemurs?}

%---Note---%
\note{

First main idea: use a sample to make an inference about a population.

Get to idea bullet point on population parameters.

\begin{itemize}

\item We will take a survey from you.  You all are interesting, but all duke students is more
interesting.

\item We will assume you are random sample to say something about whole population.

\item Why do we take samples?  Populatiosn are hard to access.

\item How would we get access to all duke students?  Registrar?  Registrar won't give out contact info for all students---privacy considerations.

\item U.S.\ pop., to much money to survey all.

\item Bottom line: usually don't have access to population so we must sample.

\item Then use that sample to make inferences about population we are interested in.

\end{itemize}

Say we want to know how many offspring female lemurs have on average.
\begin{itemize}

\item trip to madagascar?

\item use data from duke lemur center

\item use sample mean from these data as estimate for unknown population parameter

\item sample statistics vs. pop param---things we don't have access to.

\item the better (more representative) sample, the better inferences.

\end{itemize}

To take another example, say we want to estimate the average number of hours enrolled by Duke student's this semester.
\begin{itemize}

\item want to estimate average number of hours

\item used you as a sample.

\item Q: would you make up a good sample?

\item  this is a trinity course.  most pub pol and poli sci., not pratt.
pratt may have more labs or something.

\item but if asking about favorite music may not be a problem.  So whether a sample is representative depends in part on the question.

\end{itemize}

Can you see any limitations to lemur center?
\begin{itemize}
\item they are a captive population of lemurs, there is always going to be some
potential biases that come from that.
\end{itemize}

We want to watch out for and try to avoid such biases, and if we can't we want to report that they might exist.

}

\end{frame}

%%%%%%%%%%%%%%%%%%%%%%%%%%%%%%%%%%%

\begin{frame}
\frametitle{Sampling is natural}

\begin{center}
\includegraphics[width=0.3\textwidth]{figures/soup}
\end{center}

\begin{itemize}

\item When you taste a spoonful of soup and decide the spoonful you tasted isn't salty enough, that's \hl{exploratory analysis}

\item If you generalize and conclude that your entire soup needs salt, that's an \hl{inference}

\item For your inference to be valid, the spoonful you tasted (the sample) needs to be \hl{representative} of the entire pot (the population)

\end{itemize}

%---Note---%
\note{

tasting food analogy
\begin{itemize}
\item do you taste entire pot to see if it is salty enough.

\item one taste: exploratory analysis.

\item going from taste, to entire pot is inference.  this soup is too salty.

\item if you haven't mixed things up are you actually going to get a representative
smaple.

\item you need to mix things up to get representative sample.
\end{itemize}

}

\end{frame}

%%%%%%%%%%%%%%%%%%%%%%%%%%%%%%%%%%%

\subsection{Ideally use a simple random sample, stratify to control for a variable, and cluster to make sampling easier} 
\label{mi2}

%%%%%%%%%%%%%%%%%%%%%%%%%%%%%%%%%%%

\begin{frame}
\frametitle{2. Ideally use a simple random sample, stratify to control for a variable, and cluster to make sampling easier}

\begin{center}
\textbf{Demo:} \webURL{http://bl.ocks.org/avimoondra}
\end{center}

\begin{itemize}
\item \hl{Simple random sampling:} Randomly select cases from the population, each case is equally likely to be selected

\item \hl{Stratified sampling:} First divide the population into homogenous \hl{strata}, then randomly sample from \underline{each} stratum
\begin{itemize}
\item e.g. Stratify to control for socio-economic status
\end{itemize}

\item \hl{Cluster sampling:} First randomly sample \underline{a few} clusters, then randomly sample from within them
\begin{itemize}
\item \hl{Clusters} are not necessarily homogenous, but ideally they're not too different from each other
\item e.g. First sample a few schools from a school district, and then only sample students from within those schools
\item Usually preferred for economical reasons
\end{itemize}

\end{itemize}

%---Note---%
\note{

Simple random sampling:
\begin{itemize}

\item simple random sample is equally likely to be selected.

\end{itemize}

Stratified sampling:
\begin{itemize}
\item denotted by different colors.

\item when stratifying, first group by what is common about them, say socio-economic status.  Then go into each group, each strata, and randomly sample.
\end{itemize}

Clustering:
\begin{itemize}

\item maybe based on geographical information or something like that.

\item figure out what clusters are, since you don't have resources, pick just some clusters, and then sample.

\item ideally, clusters are similar

\item if clusters are similar then this isn't going to be a problem.

\end{itemize}

R: when might you use clustering?
\begin{itemize}
\item schools very similar in school district.
\item first sample a few schools.  then send surveys.
\end{itemize}

Lots of us surveys do clustering by county

}

\end{frame}

%%%%%%%%%%%%%%%%%%%%%%%%%%%%%%%%%%%

\begin{frame}

\clicker{A city council has requested a household survey be conducted in a suburban area of their city. The area is broken into many distinct and unique neighborhoods, some including large homes, some with only apartments, and others a diverse mixture of housing structures. Which approach would likely be the \emph{least} effective?}

\begin{enumerate}[(a)]
\item Simple random sampling
\item Stratified sampling, where each cluster is a neighborhood
\item \solnMult{Cluster sampling, where each cluster is a neighborhood}
\end{enumerate}

%---Note---%
\note{

Clicker question: read question and answers, 45-50 seconds.

C is most popular by far.  Someone that chose C, why?  clusters sound dissimilar.

Give exmaple of duke forrest, lots of Professors live there. That isn't representative of Durhams population.

}

\end{frame}

%%%%%%%%%%%%%%%%%%%%%%%%%%%%%%%%%%%

\subsection{Sampling schemes can suffer from a variety of biases}
\label{mi3}

%%%%%%%%%%%%%%%%%%%%%%%%%%%%%%%%%%%

\begin{frame}
\frametitle{3. Sampling schemes can suffer from a variety of biases}

\begin{itemize}[<+->]

\item \hl{Non-response:} If only a small fraction of the randomly sampled people choose to respond to a survey, the sample may no longer be representative of the population

\item \hl{Voluntary response:} Occurs when the sample consists of people who volunteer to respond because they have strong opinions on the issue since such a sample will also not be representative of the population

\item \hl{Convenience sample:} Individuals who are easily accessible are more likely to be included in the sample

\end{itemize}

%---Note---%
\note{

Non-response: 
\begin{itemize}
\item R: when do we generally have non-response errors?
\item when we have reached out to sample and people still refuse to respond.
\end{itemize}

Voluntary response bias:
\begin{itemize}

\item voluntary response issue: people volunteer to respond.  

\item cnn always has a poll.  disclaimer: this is not a scientific poll.

\item Q: who takes time to repsond to somethign that shows up on screen?

\item people that have really strong opinions.

\end{itemize}

The big distinction between voluntary and non-response is that with non-response the person designing the survey has done everything possible to avoid biasing the results.  Voluntary response may be due to a lack of effort on the part of the scientist.

convenience samples:
\begin{itemize}

\item invididuals who are easily accessible are more likely to be included

\item ask friends, friends more likeliy to be like minded

\item how much you work out, and you like to work out, and you run survey at gym...

\end{itemize}

}

\end{frame}

%%%%%%%%%%%%%%%%%%%%%%%%%%%%%%%%%%%

\begin{frame}[shrink]

{\small
\clicker{A school district is considering whether it will no longer allow high school students to park at school after two recent accidents where students were severely injured. As a first step, they survey parents by mail, asking them whether or not the parents would object to this policy change. Of 6,000 surveys that go out, 1,200 are returned. Of these 1,200 surveys that were completed, 960 agreed with the policy change and 240 disagreed. Which of the following statements are true?}

\begin{enumerate}[I.]
\item Some of the mailings may have never reached the parents.
\item Overall, the school district has strong support from parents to move forward with the policy approval.
\item It is possible that majority of the parents of high school students disagree with the policy change.
\item The survey results are unlikely to be biased because all parents were mailed a survey. 
\end{enumerate}

\begin{multicols}{5}
\begin{enumerate}[(a)]
\item Only I
\item I and II
\item \solnMult{I and III}
\item III and IV
\item Only IV
\end{enumerate}
\end{multicols}
}

%---Note---%
\note{

Clicker question (school district), read Q\&A, give 1min.

blue clicker

quite a few for C

TODO TODO

}

\end{frame}

%%%%%%%%%%%%%%%%%%%%%%%%%%%%%%%%%%%%

\subsection{Experiments use random assignment to treatment groups, observational studies do not}
\label{mi4}

%%%%%%%%%%%%%%%%%%%%%%%%%%%%%%%%%%%%

\begin{frame}
\frametitle{}

\disc{What type of study is this? What is the scope of inference (causality / generalizability)?}

\begin{center}
\includegraphics[width=0.9\textwidth]{figures/facebook_study}
\end{center}

\ct{\webURL{http://www.nytimes.com/2014/06/30/technology/facebook-tinkers-with-users-emotions-in-news-feed-experiment-stirring-outcry.html}}

%---Note---%
\note{

Facebook example:
\begin{itemize}
\item go over difference between experiements and observational studies.  Random assignment.  Random sampling.

\item News with facebook, they published a paper in conjuction with researchers... read from article.

\item Q: what type of study is this? experment.

\item Q: why?  they were actually randomly assigning people.

\item R: how would we make this an observational study?

\item What is the scope of inference.  Q: What can we say about causality, what can we talk about generalizability?

\item R: what does causality require: random assignment

\item R: what does generalizabiltiy require: random sample.

\item Facebook got in big trouble: didn't tell anybody.  big no-no in research.

\end{itemize}

}

\end{frame}

%%%%%%%%%%%%%%%%%%%%%%%%%%%%%%%%%%%%%

\begin{frame}
\frametitle{4. Experiments use random assignment to treatment groups, observational studies do not}

{\small
\disc{A study that surveyed a random sample of otherwise healthy adults found that people are more likely to get muscle cramps when they're stressed. The study also noted that people drink more coffee and sleep less when they're stressed. What type of study is this?}

\soln{\onslide<2->{Observational}}

\disc{What is the conclusion of the study?}

\soln{\onslide<3->{There is an \hl{association} between increased stress \& muscle cramps.}}

\disc{Can this study be used to conclude a causal relationship between increased stress and muscle cramps?}

\soln{\onslide<4->{Muscle cramps might also be due to increased caffeine consumption or sleeping less -- these are potential \hl{confounding} variables.}}
}

%---Note---%
\note{

Muscle cramp study:
\begin{itemize}
\item Read about study.

\item Q: What type of study? observational.

\item What is the conclusion, what type of wording would be appropriate?  There is an association between these things.

\item Can we make a causal statement?  no.

\item Read.  we don't know if it is whatever or whatever.  what we call these are potential confounding variables.
\end{itemize}

}

\end{frame}

%%%%%%%%%%%%%%%%%%%%%%%%%%%%%%%%%%%%

\subsection{Four principles of experimental design: randomize, control, block, replicate}
\label{mi5}

%%%%%%%%%%%%%%%%%%%%%%%%%%%%%%%%%%%%

\begin{frame}
\frametitle{5. Four principles of experimental design:\\ randomize, control, block, replicate}

\begin{itemize}
\item We would like to design an experiment to investigate if increased stress causes muscle cramps:

\pause

\begin{itemize}
\item Treatment: increased stress
\item Control: no or baseline stress
\end{itemize}

\pause

\item It is suspected that the effect of stress might be different on younger and older people: \hl{block} for age.

\end{itemize}

\pause

\disc{Why is this important? Can you think of other variables to block for?}

\pause

\begin{center}
\textbf{Demo:} \webURL{http://bl.ocks.org/avimoondra}
\end{center}

%---Note---%
\note{

Four principles of experimental design.

\begin{itemize}
\item We want to know if increased stress CAUSES muscle cramps.  Need random assignment.
\item Effect of stress might be different on younger than older people.  Block for age if you think age will effect how people will respond.  Then do random assignment within each group.
\item Q: What other variables might we block for?  Sex/gender.  You want the same number of males and females in each group.
\item Blocking gives us an equal distribution of each group to the control and experiment.
\end{itemize}


Demo: 54 patients.  color is important.
\begin{itemize}
\item assign some to treatment and some to control.
\item blocking gives us equal distribution of each type to control and treatment.
\end{itemize}

}

\end{frame}

%%%%%%%%%%%%%%%%%%%%%%%%%%%%%%%%%%%

\subsection{Random sampling helps generalizability, random assignment helps causality}
\label{mi6}

%%%%%%%%%%%%%%%%%%%%%%%%%%%%%%%%%%%%

\begin{frame}
\frametitle{6. Random sampling helps generalizability,\\ random assignment helps causality}

\begin{center}
\includegraphics[width=\textwidth]{figures/random_sample_assignment}
\end{center}

%---Note---%
\note{

random assignment: causality

random sample is hard to get: rarely we will get these.

volunteers: voluntary response bias.

usually most experiments lower left.

most surveys: upper right. (social sciences)

lower right: weakest type of study.

}

\end{frame}

%%%%%%%%%%%%%%%%%%%%%%%%%%%%%%%%%%%

\begin{frame}

\app{1.1 Scientific studies in the press}
{{\small Read media coverage of a study titled ``Haters Are Gonna Hate, Study Confirms" and answer the following questions. If the relevant information isn't in the article, refer to the original study.}}

\twocol{0.75}{0.25}{
{\footnotesize
\begin{enumerate}
\item What are the cases?
\item What is (are) the response variable(s) in this study?
\item What is (are) the explanatory variable(s) in this study?
\item Does the study employ random sampling? How about random assignment?
\item Is this an observational study or an experiment? Explain your reasoning.
\item Can we establish a causal link between the explanatory and response variables?
\item Can the results of the study be generalized to the population at large?
\end{enumerate}
}
}
{
\begin{center}
\includegraphics[width=\textwidth]{figures/haters}
\end{center}
}

%---Note---%
\note{

\begin{itemize}

\item work on application exercise: we will talk about next time.  read study and answer a few questions.

\item let me walk you through how you are going to do that.

\item In the same teams.  Put up app exercise on Sakai.  Explain what to do.

\item I will tell you when you are free to leave.

\item Normally, I'd ask you to work with your team for a few minutes outside of class before wednesday.  It's fine if you don't finish today.

\end{itemize}

}

\end{frame}

%%%%%%%%%%%%%%%%%%%%%%%%%%%%%%%%%%%

\section{Summary}

%%%%%%%%%%%%%%%%%%%%%%%%%%%%%%%%%%%

\begin{frame}
\frametitle{Summary of main ideas}

\vfill

\begin{enumerate}

\item \nameref{mi1}

\item \nameref{mi2}

\item \nameref{mi3}

\item \nameref{mi4}

\item \nameref{mi5}

\item \nameref{mi6}

\end{enumerate}

\vfill

\end{frame}

%%%%%%%%%%%%%%%%%%%%%%%%%%%%%%%%%%%

\end{document}