% -*- TeX-engine: xetex; -*-
% Compile with XeLaTeX

%%%%%%%%%%%%%%%%%%%%%%%
% To do before class
%%%%%%%%%%%%%%%%%%%%%%%

% Send the Logistics/Week0Annoucnement (the night before).
% Send an email reminding students to bring a charged computer (the night before).

%%%%%%%%%%%%%%%%%%%%%%%
% Option 1: Slides: (comment for handouts)   %
%%%%%%%%%%%%%%%%%%%%%%%

\documentclass[slidestop,compress,mathserif,12pt,t,professionalfonts,xcolor=table]{beamer}

% solution stuff
\newcommand{\solnMult}[1]{
\only<1>{#1}
\only<2->{\red{\textbf{#1}}}
}
\newcommand{\soln}[1]{\textit{#1}}

%%%%%%%%%%%%%%%%%%%%%%%%%%%%%%%
% Option 2: Handouts, without solutions (post before class)    %
%%%%%%%%%%%%%%%%%%%%%%%%%%%%%%%

% \documentclass[11pt,containsverbatim,handout,xcolor=xelatex,dvipsnames,table]{beamer}

% % handout layout
% \usepackage{pgfpages}
% \pgfpagesuselayout{4 on 1}[letterpaper,landscape,border shrink=5mm]

% % solution stuff
% \newcommand{\solnMult}[1]{#1}
% \newcommand{\soln}[1]{}

% % % This breaks things for me for some reason.
% % tell pgfpages how to set page sizes in XeLaTeX
% %\renewcommand\pgfsetupphysicalpagesizes{%
% %   \pdfpagewidth\pgfphysicalwidth\pdfpageheight\pgfphysicalheight%
% %}

%%%%%%%%%%%%%%%%%%%%%%%%%%%%%%%%%%%%
% Option 3: Handouts, with solutions (may post after class if need be)    %
%%%%%%%%%%%%%%%%%%%%%%%%%%%%%%%%%%%%

% \documentclass[11pt,containsverbatim,handout,xcolor=xelatex,dvipsnames,table]{beamer}

% % handout layout
% \usepackage{pgfpages}
% \pgfpagesuselayout{4 on 1}[letterpaper,landscape,border shrink=5mm]

% % solution stuff
% \newcommand{\solnMult}[1]{\red{\textbf{#1}}}
% \newcommand{\soln}[1]{\textit{#1}}

% % % This breaks things for me for some reason.
% % % tell pgfpages how to set page sizes in XeLaTeX
% % \renewcommand\pgfsetupphysicalpagesizes{%
% %    \pdfpagewidth\pgfphysicalwidth\pdfpageheight\pgfphysicalheight%
% % }

%%%%%%%%%%%%%%%%%%%%%%%%%%%%%%%
% Option 4: Notes Only
%%%%%%%%%%%%%%%%%%%%%%%%%%%%%%%

% % See http://tex.stackexchange.com/questions/114219/add-notes-to-latex-beamer
% \documentclass[10pt,containsverbatim,xcolor=xelatex,dvipsnames,table,notes=only]{beamer}

% % handout layout
% % \usepackage{pgfpages}
% % \pgfpagesuselayout{1 on 1}[letterpaper, landscape, border shrink=5mm]

% % solution stuff
% \newcommand{\solnMult}[1]{#1}
% \newcommand{\soln}[1]{}

% % % Having a problem with this.
% % tell pgfpages how to set page sizes in XeLaTeX
% % \renewcommand\pgfsetupphysicalpagesizes{%
% %   \pdfpagewidth\pgfphysicalwidth\pdfpageheight\pgfphysicalheight%
% %}

%%%%%%%%%%
% Load style file, defaults  %
%%%%%%%%%%

\input{../../lec_style.tex}
% You cannot use numbers when defining variables.  Hence the use of letters, A, B, C, etc.

% Personal Info
\newcommand{\FirstName}{Mine}
\newcommand{\LastName}{\c{C}etinkaya-Rundel}
\newcommand{\OfficeHours}{MTWR 3-4pm.}
\newcommand{\OfficeHoursLocation}{Old Chem 213}

% Electronic Info
\newcommand{\PersonalSite}{http://stat.duke.edu/~mc301}
\newcommand{\CourseSite}{http://bitly.com/sta101sp15}
\newcommand{\Email}{mine@stat.duke.edu}

% TAs
\newcommand{\TAA}{Anthony Weishampel}
\newcommand{\TAB}{Fiamma Li}
\newcommand{\TAC}{Jialiang Mao}
\newcommand{\TAD}{Phillip Lee}

% Exam Dates
\newcommand{\ExamADate}{Wed, Feb 18}
\newcommand{\ExamBDate}{Wed, Mar 25}
\newcommand{\FinalDate}{Sat, May 2 (2-5pm)}

% Due Dates
\newcommand{\ClickerRegistrationDD}{Mon, Jan 26}
\newcommand{\GettingToKnowYouDD}{Friday, Jan 9, 11:59pm}
\newcommand{\ProblemSetADD}{Wed., 1/15}


% ALT ALT
% % You cannot use numbers when defining variables.  Hence the use of letters, A, B, C, etc.

% Personal Info
\renewcommand{\FirstName}{Jesse}
\renewcommand{\LastName}{Windle}
\renewcommand{\OfficeHours}{Tue, Thu 3:00pm-4:30pm}
\renewcommand{\OfficeHoursLocation}{Old Chem 211D}

% Electronic Info
\renewcommand{\PersonalSite}{http://stat.duke.edu/~jbw44/}
\renewcommand{\CourseSite}{http://bitly.com/windle2}
\renewcommand{\Email}{jbw44@stat.duke.edu}

% TAs
\renewcommand{\TAA}{David Clancy}
\renewcommand{\TAB}{Xinyi (Chris) Li}
\renewcommand{\TAC}{Tori Hall}
\renewcommand{\TAD}{Radhika Anand}

% Exam Dates
\renewcommand{\ExamADate}{Thu, Feb 19}
\renewcommand{\ExamBDate}{Thu, Mar 26}
\renewcommand{\FinalDate}{Mon, Apr 27 (9-Noon)}

% Due Dates
\renewcommand{\ClickerRegistrationDD}{Thu, Jan 15}
\renewcommand{\GettingToKnowYouDD}{Friday, Jan 9, 11:59pm}
\renewcommand{\ProblemSetADD}{Thu., 1/16}

%%%%%%%%%%%
% Cover slide info    %
%%%%%%%%%%%

\title{Unit 2: Probability and distributions}
\subtitle{2. Bayes' theorem and Bayesian inference}
\author{Sta 101 - Spring 2015}
\date{January 28, 2015}
\institute{Duke University, Department of Statistical Science}


%%%%%%%%%%%%%%%%%%%%%%%%%
% Begin document and set Helvetica Neue font   %
%%%%%%%%%%%%%%%%%%%%%%%%%

\begin{document}
\fontspec[Ligatures=TeX]{Helvetica Neue Light}

%%%%%%%%%%%%%%%%%%%%%%%%%%%%%%%%%%%%

% Title Page

\begin{frame}[plain]

\titlepage
\vfill
{\scriptsize \webLink{\PersonalSite}{Dr. \LastName{}} \hfill Slides posted at  \webLink{\CourseSite}{\CourseSite}}
\addtocounter{framenumber}{-1} 

\end{frame}

%%%%%%%%%%%%%%%%%%%%%%%%%%%%%%%%%%%%

\section{Housekeeping}

%%%%%%%%%%%%%%%%%%%%%%%%%%%%%%%%%%%%

\begin{frame}
\frametitle{Announcements}

\begin{itemize}

\item Review Project 1 assignment and start thinking about data you might want to find / collect for your project

\end{itemize}

%---Note---%
\note{
Talk about project.

You as a team are going to find your own data set.

The library has provided some data sets for you to use.

These tend to be survey data.

The library actually has office hours.

We can help you upload the data set.  But FINDING the data set is on you.

Go over due dates on Schedule.
}

\end{frame}

%%%%%%%%%%%%%%%%%%%%%%%%%%%%%%%%%%%%

\section{Main ideas}

%%%%%%%%%%%%%%%%%%%%%%%%%%%%%%%%%%%%

\subsection{Probability trees are useful for conditional probability calculations}
\label{mi1}

%%%%%%%%%%%%%%%%%%%%%%%%%%%%%%%%%%%%

\begin{frame}
\frametitle{1. Probability trees are useful for conditional probability calculations}

\begin{itemize}

\item Probability trees are useful for organizing information in conditional probability calculations

\item They're especially useful in cases where you know P(A $|$ B), along with some other information, and you're asked for P(B $|$ A)

\end{itemize}

%---Note---%
\note{

Assigned one problem with prob tree to prepare for today.

Prob. Trees: Not necesssary.  But a good way to organize information.  Very
useful when reversing the conditioning information.

Bayes Thoerem: using this theorem to actually make decisions.  

Learned one method: the p-value method.

GIVE A REAL WORLD EXAMPLE OF BAYES.

}

\end{frame}

%%%%%%%%%%%%%%%%%%%%%%%%%%%%%%%%%%%%

\subsection{Bayesian inference: start with a prior, collect data, calculate posterior, make a decision or iterate}
\label{mi2}

%%%%%%%%%%%%%%%%%%%%%%%%%%%%%%%%%%%%

\begin{frame}
\frametitle{2. Bayesian inference: start with a prior, collect data, calculate posterior, make a decision or iterate}

\vfill

We'll play a game to demonstrate this approach...

\vfill

\end{frame}

%%%%%%%%%%%%%%%%%%%%%%%%%%%%%%%%%%%%

\begin{frame}
\frametitle{Dice game}

\begin{itemize}

\item Two dice: 6-sided and 12-sided
\begin{itemize}
\item I keep one die in my left hand and one die on the right
\end{itemize}

\pause

\item Ultimate goal: come to a class consensus about whether the die on the left or the die on the right is the ``good die"

\pause

\item We will start with priors, collect data, and calculate posteriors, and make a decision or iterate until we're ready to make a decision

\end{itemize}

%---Note---%
\note{

INSERT GOOD DIE IS THE ONE WITH HIGHER PROB OF GETTING 4 or MORE.

COST OF MAKING A MISTAKE.

COST OF PLAYING THE GAME FOR TOO LONG: You have to study this at home.

NEED A VOLUNTEER.  NOTE Taker.

Have them WALK THROUGH DECISION PROCESS.

Go to 40-60.

Pass around candy.

One nice aspect of this is that probabilities go like your brain is thinking.

}

\end{frame}

%%%%%%%%%%%%%%%%%%%%%%%%%%%%%%%%%%%%

\begin{frame}
\frametitle{Prior probabilities}

\begin{itemize}

\item At each roll I tell you whether you won or not (win = $\ge 4$)
\begin{itemize}
\item P(win on 6-sided die) = 0.5 $\rightarrow$ bad die
\item P(win on 12-sided die) = 0.75 $\rightarrow$ good die
\end{itemize}

\pause

\item The two competing claims are
\begin{itemize}
\item[] $H_1$: Good die is in left hand
\item[] $H_2$: Good die is in right hand
\end{itemize}

\pause

\item Since initially you have no idea which is true, you can assign equal \hl{prior probabilities} to the hypotheses
\begin{itemize}
\item[] P($H_1$ is true) = 0.5 
\item[] P($H_2$ is true) = 0.5 
\end{itemize}

\end{itemize}

\end{frame}

%%%%%%%%%%%%%%%%%%%%%%%%%%%%%%%%%%%%

\begin{frame}
\frametitle{Rules of the game}

\begin{itemize}

\item You won't know which die I'm holding in which hand, left (L) or right (R). {\footnotesize left = YOUR left}

\pause

\item You pick die (L or R), I roll it, and I tell you if you win or not, where winning is getting a number $\ge$ 4. If you win, you get a piece of candy. If you lose, I get to keep the candy.

\pause

\item We'll play this multiple times with different contestants.

\pause

\item I will not swap the sides the dice are on at any point.

\pause

\item You get to pick how long you want play, but there are costs associated with playing longer.
\end{itemize}

\end{frame}

%%%%%%%%%%%%%%%%%%%%%%%%%%%%%%%%%%%%

\begin{frame}
\frametitle{Hypotheses and decisions}

\begin{center}
\renewcommand\arraystretch{1.25}
\begin{tabular}{l | c | c | }
  			&\multicolumn{2}{c|}{\hl{Truth}} \\
\cline{2-3}
\hl{Decision}		& {\small L good, R bad}		& {L bad, R good} \\
\hline
Pick L		& \hlGr{You get candy!}			& \red{You lose all the candy :(} \\
\hline
Pick R		& \red{You lose all the candy :(}	& \hlGr{You get candy!} \\
\hline
\end{tabular}
\end{center}

\vspace{0.75cm}

\hl{Sampling isn't free!} \\
At each trial you risk losing pieces of candy if you lose (the die comes up $<$ 4). Too many trials means you won't have much candy left. And if we spend too much class time and we may not get through all the material.
	
\end{frame}

%%%%%%%%%%%%%%%%%%%%%%%%%%%%%%%%%%%%

\begin{frame}
\frametitle{Data and decision making}

\begin{center}
\renewcommand\arraystretch{1.25}
\begin{tabular}{l | c | c}
		& Choice (L or R)	& Result (win or loss) \\
\hline
Roll 1	& 				& 				\\
\hline
Roll 2	& 				& 				\\
\hline
Roll 3	& 				& 				\\
\hline
Roll 4	& 				& 				\\
\hline
Roll 5	& 				& 				\\
\hline
Roll 6	& 				& 				\\
\hline
Roll 7	& 				& 				\\
\hline
...		& 				& 				\\
\end{tabular}
\end{center}

\disc{What is your decision? How did you make this decision?}

\end{frame}

%%%%%%%%%%%%%%%%%%%%%%%%%%%%%%%%%%%%

\begin{frame}
\frametitle{Posterior probability}

\begin{itemize}

\item \hl{Posterior probability} is the probability of the hypothesis given the observed data: P(hypothesis $|$ data)

\pause

\item Using Bayes' theorem
\begin{eqnarray*}
P(hypothesis~data) &=& \frac{P(hypothesis~and~data)}{P(data)} \\
\pause
&=& \frac{P(data~|~hypothesis) \times P(hypothesis)}{P(data)}
\end{eqnarray*}

\end{itemize}

\end{frame}

%%%%%%%%%%%%%%%%%%%%%%%%%%%%%%%%%%%%

\begin{frame}
\frametitle{}

\disc{Calculate the posterior probability for the hypothesis chosen in the first roll, and discuss how this might influence your decision for the next roll.}

% Set the overall layout of the tree
\tikzstyle{level 1}=[level distance=3.5cm, sibling distance=3.5cm]
\tikzstyle{level 2}=[level distance=3.5cm, sibling distance=1.5cm]

% Define styles for bags
\tikzstyle{bag} = [text width=4em, text centered]
\tikzstyle{end} = [minimum width=3pt, inner sep=0pt]

% Draw picture
\hspace{-1.5cm}
\begin{tikzpicture}[grow=right]
\node[bag] {}
    child {
        node[bag] {\emptybox{0.5cm}{0.25cm}      }  
            child {
                node[end, label=right:
                    {\emptybox{0.5cm}{0.25cm}      }] {}
                edge from parent
            }
            child {
                node[end, label=right:
                    {\emptybox{0.5cm}{0.25cm}      }] {}
                edge from parent
            }
            edge from parent
    }
    child {
        node[bag] {\emptybox{0.5cm}{0.25cm}      }        
            child {
                node[end, label=right:
                    {\emptybox{0.5cm}{0.25cm}      }] {}
                edge from parent
            }
            child {
                node[end, label=right:
                    {\emptybox{0.5cm}{0.25cm}      }] {}
                edge from parent
            }
            edge from parent
    };
\end{tikzpicture}


%---Note---%
\note{

Are your calculations making intuitive sense.

Show how to repeat.

Each person may have a different cost.  You may stop at different times.

Can I come up with a demo.

}

\end{frame}

%%%%%%%%%%%%%%%%%%%%%%%%%%%%%%%%%%%%

\subsection{Posterior probability and p-value do not mean the same thing}
\label{mi3}

%%%%%%%%%%%%%%%%%%%%%%%%%%%%%%%%%%%%

\begin{frame}
\frametitle{3. Posterior probability and p-value do not mean the same thing}

\begin{itemize}

\item $\hl{p-value}:$ P(observed or more extreme outcome $|$ null hypothesis is true)
\begin{itemize}
\item This is roughly P(data $|$ hypothesis)
\end{itemize}

\item $\hl{posterior}:$ P(hypothesis $|$ data)

\item Bayesian approach avoids the counter-intuitive Frequentist p-value for decision making, and more advanced Bayesian techniques offer flexibility not present in Frequentist models

\item \hl{Watch out!} \\
A good prior helps, a bad prior hurts, but the prior matters less the more data you have.

\end{itemize}

%---Note---%
\note{

GET PAPER about p-values and put up.

}

\end{frame}

%%%%%%%%%%%%%%%%%%%%%%%%%%%%%%%%%%%%

\begin{frame}
\frametitle{}

\vfill

\app{2.2 Bayesian inference for drug testing}{$\:$\\ See the course website for instructions. \\$\:$}

\vfill

\end{frame}

%%%%%%%%%%%%%%%%%%%%%%%%%%%%%%%%%%%%

\section{Summary}

%%%%%%%%%%%%%%%%%%%%%%%%%%%%%%%%%%%%

\begin{frame}
\frametitle{Summary of main ideas}

\vfill

\begin{enumerate}

\item \nameref{mi1}

\item \nameref{mi2}

\item \nameref{mi3}

\end{enumerate}

\vfill

\end{frame}

%%%%%%%%%%%%%%%%%%%%%%%%%%%%%%%%%%%

\end{document}