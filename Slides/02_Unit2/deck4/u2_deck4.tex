% -*- TeX-engine: xetex; -*-
% Compile with XeLaTeX

%%%%%%%%%%%%%%%%%%%%%%%
% To do before class
%%%%%%%%%%%%%%%%%%%%%%%

% Send the Logistics/Week0Annoucnement (the night before).
% Send an email reminding students to bring a charged computer (the night before).

%%%%%%%%%%%%%%%%%%%%%%%
% Option 1: Slides: (comment for handouts)   %
%%%%%%%%%%%%%%%%%%%%%%%

\documentclass[slidestop,compress,mathserif,12pt,t,professionalfonts,xcolor=table]{beamer}

% solution stuff
\newcommand{\solnMult}[1]{
\only<1>{#1}
\only<2->{\red{\textbf{#1}}}
}
\newcommand{\soln}[1]{\textit{#1}}

%%%%%%%%%%%%%%%%%%%%%%%%%%%%%%%
% Option 2: Handouts, without solutions (post before class)    %
%%%%%%%%%%%%%%%%%%%%%%%%%%%%%%%

% \documentclass[11pt,containsverbatim,handout,xcolor=xelatex,dvipsnames,table]{beamer}

% % handout layout
% \usepackage{pgfpages}
% \pgfpagesuselayout{4 on 1}[letterpaper,landscape,border shrink=5mm]

% % solution stuff
% \newcommand{\solnMult}[1]{#1}
% \newcommand{\soln}[1]{}

% % % This breaks things for me for some reason.
% % tell pgfpages how to set page sizes in XeLaTeX
% %\renewcommand\pgfsetupphysicalpagesizes{%
% %   \pdfpagewidth\pgfphysicalwidth\pdfpageheight\pgfphysicalheight%
% %}

%%%%%%%%%%%%%%%%%%%%%%%%%%%%%%%%%%%%
% Option 3: Handouts, with solutions (may post after class if need be)    %
%%%%%%%%%%%%%%%%%%%%%%%%%%%%%%%%%%%%

% \documentclass[11pt,containsverbatim,handout,xcolor=xelatex,dvipsnames,table]{beamer}

% % handout layout
% \usepackage{pgfpages}
% \pgfpagesuselayout{4 on 1}[letterpaper,landscape,border shrink=5mm]

% % solution stuff
% \newcommand{\solnMult}[1]{\red{\textbf{#1}}}
% \newcommand{\soln}[1]{\textit{#1}}

% % % This breaks things for me for some reason.
% % % tell pgfpages how to set page sizes in XeLaTeX
% % \renewcommand\pgfsetupphysicalpagesizes{%
% %    \pdfpagewidth\pgfphysicalwidth\pdfpageheight\pgfphysicalheight%
% % }

%%%%%%%%%%%%%%%%%%%%%%%%%%%%%%%
% Option 4: Notes Only
%%%%%%%%%%%%%%%%%%%%%%%%%%%%%%%

% % See http://tex.stackexchange.com/questions/114219/add-notes-to-latex-beamer
% \documentclass[10pt,containsverbatim,xcolor=xelatex,dvipsnames,table,notes=only]{beamer}

% % handout layout
% % \usepackage{pgfpages}
% % \pgfpagesuselayout{1 on 1}[letterpaper, landscape, border shrink=5mm]

% % solution stuff
% \newcommand{\solnMult}[1]{#1}
% \newcommand{\soln}[1]{}

% % % Having a problem with this.
% % tell pgfpages how to set page sizes in XeLaTeX
% % \renewcommand\pgfsetupphysicalpagesizes{%
% %   \pdfpagewidth\pgfphysicalwidth\pdfpageheight\pgfphysicalheight%
% %}

%%%%%%%%%%
% Load style file, defaults  %
%%%%%%%%%%

\input{../../lec_style.tex}
% You cannot use numbers when defining variables.  Hence the use of letters, A, B, C, etc.

% Personal Info
\newcommand{\FirstName}{Mine}
\newcommand{\LastName}{\c{C}etinkaya-Rundel}
\newcommand{\OfficeHours}{MTWR 3-4pm.}
\newcommand{\OfficeHoursLocation}{Old Chem 213}

% Electronic Info
\newcommand{\PersonalSite}{http://stat.duke.edu/~mc301}
\newcommand{\CourseSite}{http://bitly.com/sta101sp15}
\newcommand{\Email}{mine@stat.duke.edu}

% TAs
\newcommand{\TAA}{Anthony Weishampel}
\newcommand{\TAB}{Fiamma Li}
\newcommand{\TAC}{Jialiang Mao}
\newcommand{\TAD}{Phillip Lee}

% Exam Dates
\newcommand{\ExamADate}{Wed, Feb 18}
\newcommand{\ExamBDate}{Wed, Mar 25}
\newcommand{\FinalDate}{Sat, May 2 (2-5pm)}

% Due Dates
\newcommand{\ClickerRegistrationDD}{Mon, Jan 26}
\newcommand{\GettingToKnowYouDD}{Friday, Jan 9, 11:59pm}
\newcommand{\ProblemSetADD}{Wed., 1/15}


% ALT ALT
% % You cannot use numbers when defining variables.  Hence the use of letters, A, B, C, etc.

% Personal Info
\renewcommand{\FirstName}{Jesse}
\renewcommand{\LastName}{Windle}
\renewcommand{\OfficeHours}{Tue, Thu 3:00pm-4:30pm}
\renewcommand{\OfficeHoursLocation}{Old Chem 211D}

% Electronic Info
\renewcommand{\PersonalSite}{http://stat.duke.edu/~jbw44/}
\renewcommand{\CourseSite}{http://bitly.com/windle2}
\renewcommand{\Email}{jbw44@stat.duke.edu}

% TAs
\renewcommand{\TAA}{David Clancy}
\renewcommand{\TAB}{Xinyi (Chris) Li}
\renewcommand{\TAC}{Tori Hall}
\renewcommand{\TAD}{Radhika Anand}

% Exam Dates
\renewcommand{\ExamADate}{Thu, Feb 19}
\renewcommand{\ExamBDate}{Thu, Mar 26}
\renewcommand{\FinalDate}{Mon, Apr 27 (9-Noon)}

% Due Dates
\renewcommand{\ClickerRegistrationDD}{Thu, Jan 15}
\renewcommand{\GettingToKnowYouDD}{Friday, Jan 9, 11:59pm}
\renewcommand{\ProblemSetADD}{Thu., 1/16}

%%%%%%%%%%%
% Cover slide info    %
%%%%%%%%%%%

\title{Unit 2: Probability and distributions}
\subtitle{4. Binomial distribution}
\author{Sta 101 - Spring 2015}
\date{February 4, 2015}
\institute{Duke University, Department of Statistical Science}


%%%%%%%%%%%%%%%%%%%%%%%%%
% Begin document and set Helvetica Neue font   %
%%%%%%%%%%%%%%%%%%%%%%%%%

\begin{document}
\fontspec[Ligatures=TeX]{Helvetica Neue Light}

%%%%%%%%%%%%%%%%%%%%%%%%%%%%%%%%%%%%

% Title Page

\begin{frame}[plain]

\titlepage
\vfill
{\scriptsize \webLink{\PersonalSite}{Dr. \LastName{}} \hfill Slides posted at  \webLink{\CourseSite}{\CourseSite}}
\addtocounter{framenumber}{-1} 

\end{frame}

%%%%%%%%%%%%%%%%%%%%%%%%%%%%%%%%%%%%

\section{Housekeeping}

%%%%%%%%%%%%%%%%%%%%%%%%%%%%%%%%%%%%

\begin{frame}
\frametitle{Announcements}

Due by Friday 11:59pm:
\begin{itemize}

\item Peer evaluations

\item PA 2

\end{itemize}

\end{frame}

%%%%%%%%%%%%%%%%%%%%%%%%%%%%%%%%%%%%

\section{Main ideas}

%%%%%%%%%%%%%%%%%%%%%%%%%%%%%%%%%%%%

\subsection{Binomial distribution is used for calculating the probability of exact number of successes for a given number of trials}
\label{mi1}

%%%%%%%%%%%%%%%%%%%%%%%%%%%%%%%%%%%%

\begin{frame}
\frametitle{High-speed broadband connection at home in the US}

\begin{center}
\includegraphics[width=0.6\textwidth]{figures/pew_internet_access}
\end{center}

\pause

\begin{itemize}
\item Each person in the poll be thought of as a \hl{trial}
\pause
\item A person is labeled a \hl{success} if s/he has high-speed broadband connection at home, \hl{failure} if not
\pause
\item Since 70\% have high-speed broadband connection at home, \hl{probability of success} is \hl{p = 0.70}
\end{itemize}

\end{frame}

%%%%%%%%%%%%%%%%%%%%%%%%%%%%%%%%%%%

\begin{frame}
\frametitle{Considering many scenarios}

\disc{Suppose we randomly select three individuals from the US, what is the probability that exactly 1 has high-speed broadband connection at home?}
\pause
Let's call these people Anthony (A), Barry (B), Cam (C). Each one of the three scenarios below will satisfy the condition of ``exactly 1 of them says Yes": \\
\vspace{0.25cm}
\pause
\begin{changemargin}{+1.5cm}{+0cm}
{\footnotesize
\begin{enumerate}
\item[Scenario 1:] $\slot{0.70}{\text{(A) \red{yes}}} \times \slot{0.30}{\text{(B) not yes}} \times \slot{0.30}{\text{(C) not yes}} \approx 0.063$
\pause
\item[Scenario 2:]  $\slot{0.30}{\text{(B) not yes}} \times \slot{0.70}{\text{(A) \red{yes}}} \times \slot{0.30}{\text{(C) not yes}} \approx 0.063$
\pause
\item[Scenario 3:]  $\slot{0.30}{\text{(B) not yes}} \times \slot{0.30}{\text{(C) not yes}} \times \slot{0.70}{\text{(A) \red{yes}}} \approx 0.063$
\end{enumerate}
}
\end{changemargin}
\pause
The probability of exactly one 1 of 3 people saying Yes is the sum of all of these probabilities.
\[ 0.063 + 0.063 + 0.063 = 3 \times 0.063 = 0.189 \]

\end{frame}


%%%%%%%%%%%%%%%%%%%%%%%%%%%%%%%%%%%

\begin{frame}
\frametitle{Binomial distribution}

The question from the prior slide asked for the probability of given number of successes, \mathhl{k}, in a given number of trials, \mathhl{n}, ($k = 1$ success in $n = 3$ trials), and we calculated this probability as
\[ \#~of~scenarios \times P(single~scenario) \]

\pause

\begin{itemize}

\item $P(single~scenario) = p^k~(1-p)^{(n-k)}$ \\
{\tiny probability of success to the power of number of successes, probability of failure to the power of number of failures}

\pause

\item number of scenarios: ${n \choose k} = \frac{n!}{k! (n - k)!}$

\end{itemize}

\pause
$\:$ \\

The \hl{Binomial distribution} describes the probability of having exactly $k$ successes in $n$ independent trials (with only two possible outcomes) with probability of success $p$.

\end{frame}

%%%%%%%%%%%%%%%%%%%%%%%%%%%%%%%%%%%

\begin{frame}[fragile]
\frametitle{Binomial distribution (cont.)}

\[P(k~successes~in~n~trials) = {n \choose k}~p^k~(1-p)^{(n-k)} \] 

\pause

\Note{You can also use R for the calculation of number of scenarios:}
{\footnotesize
\begin{Verbatim}[frame=single, formatcom=\color{blue}]
> choose(5,3)
\end{Verbatim}
}
{\footnotesize
\begin{Verbatim}[frame=single, formatcom=\color{gray}]
[1] 10
\end{Verbatim}
}

\end{frame}

%%%%%%%%%%%%%%%%%%%%%%%%%%%%%%%%%%%

\begin{frame}
\frametitle{Properties of the choose function}

\clicker{Which of the following is false?}

\begin{enumerate}[(a)]
\item There are $n$ ways of getting 1 success in $n$ trials, ${n \choose 1} = n$.
\item There is only 1 way of getting $n$ successes in $n$ trials, ${n \choose n} = 1$.
\item There is only 1 way of getting $n$ failures in $n$ trials, ${n \choose 0} = 1$.
\item \solnMult{There are $n-1$ ways of getting $n-1$ successes in $n$ trials, ${n \choose n-1} = n-1$.}
\end{enumerate}

\end{frame}

%%%%%%%%%%%%%%%%%%%%%%%%%%%%%%%%%%%

\begin{frame}

\clicker{Which of the following is not a condition that needs to be met for the binomial distribution to be applicable?}

\begin{enumerate}[(a)]
\item the trials must be independent
\item the number of trials, $n$, must be fixed
\item each trial outcome must be classified as a \textit{success} or a \textit{failure}
\item \solnMult{the number of desired successes, $k$, must be greater than the number of trials}
\item the probability of success, $p$, must be the same for each trial
\end{enumerate}

\end{frame}

%%%%%%%%%%%%%%%%%%%%%%%%%%%%%%%%%%%

\begin{frame}
\frametitle{}

\clicker{According to the results of the Pew poll suggesting that 70\% of Americans have high-speed broadband connection at home, is the probability of exactly 2 out of 15 randomly sampled Americans having such connection at home pretty high or pretty low?}

\begin{enumerate}[(a)]
\item pretty high
\item \solnMult{pretty low}
\end{enumerate}

\end{frame}

%%%%%%%%%%%%%%%%%%%%%%%%%%%%%%%%%%%%

\begin{frame}

\clicker{According to the results of the Pew poll 70\% of Americans have high-speed broadband connection at home, what is the probability that exactly 2 out of 15 randomly sampled Americans have such connection at home?}

\begin{enumerate}[(a)]
\item $0.70^{2} \times 0.30^{13}$

\item ${2 \choose 15} \times 0.70^{2} \times 0.30^{13}$

\item \solnMult{${15 \choose 2} \times 0.70^{2} \times 0.30^{13}$} \soln{\red{\only<2>{$ = \frac{15!}{13! \times 2!} \times  0.70^{2} \times 0.30^{13} = 105 \times  0.70^{2} \times 0.30^{13} = 8.2e-06$}}}

\item ${15 \choose 2} \times 0.70^{13} \times 0.30^2$

\end{enumerate}

\end{frame}

%%%%%%%%%%%%%%%%%%%%%%%%%%%%%%%%%%%

\subsection{Expected value and standard deviation of the binomial can be calculated using its parameters n and p}
\label{mi2}

%%%%%%%%%%%%%%%%%%%%%%%%%%%%%%%%%%%%

\begin{frame}
\frametitle{Expected value and standard deviation of binomial}

\disc{According to the results of the Pew poll suggestion that 70\% of Americans have high-speed broadband connection at home, among a random sample of 100 Americans, how many would you expect to have such connection at home?}

\pause

\begin{itemize}

\item Easy enough, $100 \times 0.70 = 70$
\pause
\begin{itemize}
\item Or more formally, $\mu = np = 100 \times 0.44 = 44$
\end{itemize}

\pause

\item But this doesn't mean in every random sample of 100 Americans exactly 70 will have high-speed broadband connection at home. In some samples there will be fewer of those, and in others more. How much would we expect this value to vary?
\pause
\begin{itemize}
\item $\sigma = \sqrt{np(1-p)} = \sqrt{100 \times 0.70 \times 0.30} \approx  4.58$
\end{itemize}

\end{itemize}

\Note{Mean and standard deviation of a binomial might not always be whole numbers, and that is alright, these values represent what we would expect to see on average.}

\end{frame}

%%%%%%%%%%%%%%%%%%%%%%%%%%%%%%%%%%%

\subsection{Shape of the binomial distribution approaches normal when the S-F rule is met}
\label{mi3}

%%%%%%%%%%%%%%%%%%%%%%%%%%%%%%%%%%%%

\begin{frame}
\frametitle{Shape of the binomial distribution}

\vfill

\begin{center}
\webURL{http://bitly.com/dist_calc}
\end{center}
\vfill

\pause
\hl{S-F rule:} The sample size is considered large enough if the expected number of successes and failures are both at least 10
\[ np \ge 10 \qquad \text{ and } \qquad n(1-p) \ge 10 \]

\end{frame}

%%%%%%%%%%%%%%%%%%%%%%%%%%%%%%%%%%%

\begin{frame}
\frametitle{}

\clicker{Below are four pairs of Binomial distribution parameters. Which distribution's shape can be approximated by the normal distribution?}

\begin{enumerate}[(a)]
\item \solnMult{$n = 25, p = 0.45$}
\item $n = 100, p = 0.95$
\item $n = 150, p = 0.05$
\item $n = 500, p = 0.015$
\end{enumerate}

\end{frame}


%%%%%%%%%%%%%%%%%%%%%%%%%%%%%%%%%%%%

\begin{frame}

\vfill

\app{2.4 Binomial distribution}{See course website for details.}

\vfill

\end{frame}

%%%%%%%%%%%%%%%%%%%%%%%%%%%%%%%%%%%%

\begin{frame}
\frametitle{Binomial $\rightarrow$ normal}

\vfill

\disc{Why do we care?}

\vfill

\end{frame}

%%%%%%%%%%%%%%%%%%%%%%%%%%%%%%%%%%%%

\section{Summary}

%%%%%%%%%%%%%%%%%%%%%%%%%%%%%%%%%%%%

\begin{frame}
\frametitle{Summary of main ideas}

\vfill

\begin{enumerate}

\item \nameref{mi1}

\item \nameref{mi2}

\item \nameref{mi3}

\end{enumerate}

\vfill

\end{frame}

%%%%%%%%%%%%%%%%%%%%%%%%%%%%%%%%%%%

\end{document}