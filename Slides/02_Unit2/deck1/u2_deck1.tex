% -*- TeX-engine: xetex; -*-
% Compile with XeLaTeX

%%%%%%%%%%%%%%%%%%%%%%%
% To do before class
%%%%%%%%%%%%%%%%%%%%%%%

% Send the Logistics/Week0Annoucnement (the night before).
% Send an email reminding students to bring a charged computer (the night before).

%%%%%%%%%%%%%%%%%%%%%%%
% Option 1: Slides: (comment for handouts)   %
%%%%%%%%%%%%%%%%%%%%%%%

\documentclass[slidestop,compress,mathserif,12pt,t,professionalfonts,xcolor=table]{beamer}

% solution stuff
\newcommand{\solnMult}[1]{
\only<1>{#1}
\only<2->{\red{\textbf{#1}}}
}
\newcommand{\soln}[1]{\textit{#1}}

%%%%%%%%%%%%%%%%%%%%%%%%%%%%%%%
% Option 2: Handouts, without solutions (post before class)    %
%%%%%%%%%%%%%%%%%%%%%%%%%%%%%%%

% \documentclass[11pt,containsverbatim,handout,xcolor=xelatex,dvipsnames,table]{beamer}

% % handout layout
% \usepackage{pgfpages}
% \pgfpagesuselayout{4 on 1}[letterpaper,landscape,border shrink=5mm]

% % solution stuff
% \newcommand{\solnMult}[1]{#1}
% \newcommand{\soln}[1]{}

% % % This breaks things for me for some reason.
% % tell pgfpages how to set page sizes in XeLaTeX
% %\renewcommand\pgfsetupphysicalpagesizes{%
% %   \pdfpagewidth\pgfphysicalwidth\pdfpageheight\pgfphysicalheight%
% %}

%%%%%%%%%%%%%%%%%%%%%%%%%%%%%%%%%%%%
% Option 3: Handouts, with solutions (may post after class if need be)    %
%%%%%%%%%%%%%%%%%%%%%%%%%%%%%%%%%%%%

% \documentclass[11pt,containsverbatim,handout,xcolor=xelatex,dvipsnames,table]{beamer}

% % handout layout
% \usepackage{pgfpages}
% \pgfpagesuselayout{4 on 1}[letterpaper,landscape,border shrink=5mm]

% % solution stuff
% \newcommand{\solnMult}[1]{\red{\textbf{#1}}}
% \newcommand{\soln}[1]{\textit{#1}}

% % % This breaks things for me for some reason.
% % % tell pgfpages how to set page sizes in XeLaTeX
% % \renewcommand\pgfsetupphysicalpagesizes{%
% %    \pdfpagewidth\pgfphysicalwidth\pdfpageheight\pgfphysicalheight%
% % }

%%%%%%%%%%%%%%%%%%%%%%%%%%%%%%%
% Option 4: Notes Only
%%%%%%%%%%%%%%%%%%%%%%%%%%%%%%%

% % See http://tex.stackexchange.com/questions/114219/add-notes-to-latex-beamer
% \documentclass[10pt,containsverbatim,xcolor=xelatex,dvipsnames,table,notes=only]{beamer}

% % handout layout
% % \usepackage{pgfpages}
% % \pgfpagesuselayout{1 on 1}[letterpaper, landscape, border shrink=5mm]

% % solution stuff
% \newcommand{\solnMult}[1]{#1}
% \newcommand{\soln}[1]{}

% % % Having a problem with this.
% % tell pgfpages how to set page sizes in XeLaTeX
% % \renewcommand\pgfsetupphysicalpagesizes{%
% %   \pdfpagewidth\pgfphysicalwidth\pdfpageheight\pgfphysicalheight%
% %}

\usepackage{outlines}

%%%%%%%%%%
% Load style file, defaults  %
%%%%%%%%%%

\input{../../lec_style.tex}
% You cannot use numbers when defining variables.  Hence the use of letters, A, B, C, etc.

% Personal Info
\newcommand{\FirstName}{Mine}
\newcommand{\LastName}{\c{C}etinkaya-Rundel}
\newcommand{\OfficeHours}{MTWR 3-4pm.}
\newcommand{\OfficeHoursLocation}{Old Chem 213}

% Electronic Info
\newcommand{\PersonalSite}{http://stat.duke.edu/~mc301}
\newcommand{\CourseSite}{http://bitly.com/sta101sp15}
\newcommand{\Email}{mine@stat.duke.edu}

% TAs
\newcommand{\TAA}{Anthony Weishampel}
\newcommand{\TAB}{Fiamma Li}
\newcommand{\TAC}{Jialiang Mao}
\newcommand{\TAD}{Phillip Lee}

% Exam Dates
\newcommand{\ExamADate}{Wed, Feb 18}
\newcommand{\ExamBDate}{Wed, Mar 25}
\newcommand{\FinalDate}{Sat, May 2 (2-5pm)}

% Due Dates
\newcommand{\ClickerRegistrationDD}{Mon, Jan 26}
\newcommand{\GettingToKnowYouDD}{Friday, Jan 9, 11:59pm}
\newcommand{\ProblemSetADD}{Wed., 1/15}


% ALT ALT
% % You cannot use numbers when defining variables.  Hence the use of letters, A, B, C, etc.

% Personal Info
\renewcommand{\FirstName}{Jesse}
\renewcommand{\LastName}{Windle}
\renewcommand{\OfficeHours}{Tue, Thu 3:00pm-4:30pm}
\renewcommand{\OfficeHoursLocation}{Old Chem 211D}

% Electronic Info
\renewcommand{\PersonalSite}{http://stat.duke.edu/~jbw44/}
\renewcommand{\CourseSite}{http://bitly.com/windle2}
\renewcommand{\Email}{jbw44@stat.duke.edu}

% TAs
\renewcommand{\TAA}{David Clancy}
\renewcommand{\TAB}{Xinyi (Chris) Li}
\renewcommand{\TAC}{Tori Hall}
\renewcommand{\TAD}{Radhika Anand}

% Exam Dates
\renewcommand{\ExamADate}{Thu, Feb 19}
\renewcommand{\ExamBDate}{Thu, Mar 26}
\renewcommand{\FinalDate}{Mon, Apr 27 (9-Noon)}

% Due Dates
\renewcommand{\ClickerRegistrationDD}{Thu, Jan 15}
\renewcommand{\GettingToKnowYouDD}{Friday, Jan 9, 11:59pm}
\renewcommand{\ProblemSetADD}{Thu., 1/16}

%%%%%%%%%%%
% Cover slide info    %
%%%%%%%%%%%

\title{Unit 2: Probability and distributions}
\subtitle{1. Probability and conditional probability}
\author{Sta 101 - Spring 2015}
\date{January 26, 2015}
% ALT ALT
% \date{January 27, 2015}
\institute{Duke University, Department of Statistical Science}


%%%%%%%%%%%%%%%%%%%%%%%%%
% Begin document and set Helvetica Neue font   %
%%%%%%%%%%%%%%%%%%%%%%%%%

\begin{document}
\fontspec[Ligatures=TeX]{Helvetica Neue Light}

%%%%%%%%%%%%%%%%%%%%%%%%%%%%%%%%%%%%

% Title Page

\begin{frame}[plain]

\titlepage
\vfill
{\scriptsize \webLink{\PersonalSite}{Dr. \LastName{}} \hfill Slides posted at  \webLink{\CourseSite}{\CourseSite}}
\addtocounter{framenumber}{-1} 

\end{frame}

%%%%%%%%%%%%%%%%%%%%%%%%%%%%%%%%%%%%

\section{Housekeeping}

%%%%%%%%%%%%%%%%%%%%%%%%%%%%%%%%%%%%

\begin{frame}
\frametitle{Announcements}

\begin{itemize}

\item Please review the regrade policy on the course syllabus: \\
{\footnotesize \textit{Regrade requests must be made within 3 days of when the assignment is returned, and must be submitted in writing. These will be honored if points were tallied incorrectly, or if you feel your answer is correct but it was marked wrong. No regrade will be made to alter the number of points deducted for a mistake. There will be no grade changes after the final exam.}}

\item Any regrade requests should be submitted to me, I will regrade the entire assignment.

\end{itemize}

%---Note---
\note{

}

\end{frame}

%%%%%%%%%%%%%%%%%%%%%%%%%%%%%%%%%%%%

\section{Readiness assessment}

%%%%%%%%%%%%%%%%%%%%%%%%%%%%%%%%%%%%

\begin{frame}
\frametitle{Readiness assessment}

\begin{itemize}

\item 15 minutes individual -- turn your clicker over when you're done

\item 10 minutes team -- put your team name on the front of the scratch off
  sheet + Lab Time + put \textbf{only} the names of the members who are present today on the back

\end{itemize}

%---Note---
\note{

\begin{outline}

\1 Stuff to clarify, perhaps beforehand.

\2 Sit to the sides.
\2 They will be tempted to talk during RA.  Don't talk during individual portion.
\2 How are scores determined, it is a weighted average.

\1 People will do pretty well

\1 Two pretty good to go over.

\2 Q2. Do Venn diagram representation.  How many both?  How many premeds?  How
many not both?  Same for social sciences.  What is the question asking?  GIVEN.
Know the denomenator out of 50.

\2 Q3.  What is min?  What is max?  Probs. only occur between zero and one.
Sample space has to add up to what percent.  

\3 Go through

What is wrong with the first option?  Adds up to more that 100.

What is wrong with B?  It does not add up to 100.

C is the proper one.

What is rong with D? can't have a negative probability.

\1 Two not so good.

\2 Q6: Which, on its own is the least useful method if to determine if normal.

Okay to check for percentiles

QQ plot okay.

Median/mode is worst option.  Why?  Doesn't tell you about the shape.  Two
symmetric multimodal.

\2 Q7: Bring up answer key to show learning objectives.

Draw a left skewed distribution.

Where is median, closer to hump or tail?

Where is mean?  Going to be at least a little closer to the tail.

Why will mean be less than?  Lower observations will be pulling it down.  (Can
do dot plot to show.)

Write down formula for z-score

\end{outline}

}

\end{frame}

%%%%%%%%%%%%%%%%%%%%%%%%%%%%%%%%%%%%

\section{Main ideas}

%%%%%%%%%%%%%%%%%%%%%%%%%%%%%%%%%%%%

\subsection{Disjoint and independent do not mean the same thing}
\label{mi1}

%%%%%%%%%%%%%%%%%%%%%%%%%%%%%%%%%%%%

\begin{frame}
\frametitle{1. Disjoint and independent do not mean the same thing}

\begin{itemize}

\item \hl{Disjoint (mutually exclusive) events} cannot happen at the same time
\begin{itemize}
\item A voter cannot register as a Democrat and a Republican at the same time
\item But s/he might be a Republican and a Moderate at the same time -- \hl{non-disjoint events}
\item For disjoint A and B: $P(A~and~B) = 0$
\end{itemize}

\pause

\item If A and B are \hl{independent events}, having information on A does not tell us anything about B (and vice versa)
\begin{itemize}
\item If A and B are independent: 
\begin{itemize}
\item $P(A~|~B) = P(A)$
\item $P(A~and~B) = P(A) \times P(B)$
\end{itemize}
\end{itemize}

\end{itemize}

%---Note---
\note{

\begin{outline}
\1  3 main ideas

\2 First idea: disjoint and independent do not mean the same thing.

\end{outline}

Let's look at the definitions of disjoint and independent.

\textbf{Disjoint}:

\begin{outline}

\1 Disjoint events are events that cannot happen at the same time.

\1 Disjoint events are also called \textbf{mutually exclusive}.

\1 If disjoint events cannot happen at the same time, the probability that both
occur is zero.

\1 Suppose I consider the population of US voters.

\2 In this setup the cases of interest are people.

\1 Now suppose I reached into this population and randomly sampled a person.

\1 One event: This person is a Democrat.

\2 And then I could ask, what is the probability of this event, and as we saw
last time, this would be the proportion of voters that are registered as Democrats.

\1 Another event: This person is a Republican.

\2 And I could also ask, what is the probability of this event.  And this would
also relate back to proportions.

\1 But these two events are mutually exclusive.  You can't be a registered
Democrate and a registered Republican at the same time.  So the probability that
the person is a Democrate and a Republican is zero.

\1 Let's consider yet another event: This person is a moderate.

\2 It is totally possible that a person is both a Republican and a moderate.  So
these two events are not disjoint.

\2 Disjointness is all about if two events can occur simultaneously.

\end{outline}

\textbf{Independence}

\begin{outline}

\1 independence has to do with one event informing another.

\1 What is the probability that I roll a 1,2,3? 1/2

\1 What is the probability that I roll a 1? 1/6

\1 Now, what if I roll this die, but don't show you what landed up.  I just say
that I rolled a 1,2,3, what is the probability that I rolled a 1? 1/3

\1 So if I don't know anything, then the probability is 1/6, but if I know that
the roll resulted in a 1,2,3, then that tells me something about the probability
of having rolled a 1.

\1 independent: two events are independent if having information on one doesn't
tell us anything about the other.

\2 ASK: can anyone think of two events that are obviously independent of each
other.  any examples?

\3 two coins example.  Bring up if they don't say.

\3 Go through conditional notation using coin example.

\end{outline}

}

\end{frame}

%%%%%%%%%%%%%%%%%%%%%%%%%%%%%%%%%%%%

\subsection{Application of the addition rule depends on disjointness of events}
\label{mi2}

%%%%%%%%%%%%%%%%%%%%%%%%%%%%%%%%%%%%

\begin{frame}
\frametitle{2. Application of the addition rule depends on disjointness of events}

\begin{itemize}

\item \hl{General addition rule:} P(A or B) = P(A) + P(B) - P(A and B)

\item A or B = either A or B or both

\end{itemize}

\vspace{0.5cm}

\pause

\twocol{0.5}{0.5}{
\textbf{disjoint events:} \\
P(A or B) \\
= P(A) + P(B) - P(A and B) \\
= 0.4 + 0.3 - 0 = 0.7
\begin{center}
\includegraphics[width = 0.9\textwidth]{figures/venn/venn_disjoint}
\end{center}
}
{
\pause
\textbf{non-disjoint events:} \\
P(A or B) \\
= P(A) + P(B) - P(A and B) \\
= 0.4 + 0.3 - 0.02 = 0.68
\begin{center}
\includegraphics[width = 0.9\textwidth]{figures/venn/venn_non_disjoint}
\end{center}
}

%---Note---
\note{

\begin{outline}

\1 This is the general addition rule.

\1 We need to watch out for this because this may be slightly different than how
we use the word OR in English.

For instance, I might ask, do you want to go to a movie or a play on Friday
night.  What I mean when I ask that is do you want to go to either a movie or a
play, but not both.

But the definition of OR always includes the possibility of both events
occuring.  

\textbf{Does either A occur or B occur or both.}

\1 It is possible to visualize the general addition rule.  

\2 When two events are disjoint we draw two blobs that don't overlap to indicate
that they are mutually exclusive and write the probability of each occuring
within the blob.

\2 When two events are not disjoint, we draw two blobs that do overlap to
indicate that it is possible for both events to occur.

The probability of A is still 0.4, the probability of B is still 0.3.  If we
just add 0.4 to 0.3 we double count the overlaps of the two events.  But we just
want to single count it.  So we need to subtract the amount that is in the
intersection.

\end{outline}

}

\end{frame}

%%%%%%%%%%%%%%%%%%%%%%%%%%%%%%%%%%%%

\subsection{Bayes' theorem works for all types of events}
\label{mi3}

%%%%%%%%%%%%%%%%%%%%%%%%%%%%%%%%%%%%

\begin{frame}
\frametitle{3. Bayes' theorem works for all types of events}

\begin{itemize}

\item \hl{Bayes' theorem:} $P(A~|~B) = \frac{P(A~and~B)}{P(B)}$

\pause

\item ... can be rewritten as: $P(A~and~B) = P(A~|~B) \times P(B)$

\end{itemize}

\pause

\vspace{0.5cm}

\twocol{0.5}{0.5}{
\textbf{disjoint events:}

\begin{itemize}
\item We know P(A $|$ B) = 0, since if B happened A could not have happened
\pause
\item P(A and B) \\
= P(A $|$ B) $\times$ P(B) \\
\pause
= \red{0} $\times$ P(B) = 0
\end{itemize}
}
{
\pause

\textbf{independent events:}

\begin{itemize}
\item We know P(A $|$ B) = P(A), since knowing B doesn't tell us anything about A
\pause
\item P(A and B) \\
= P(A $|$ B) $\times$ P(B) \\
\pause
= \red{P(A)} $\times$ P(B)
\end{itemize}
}

%---Note---
\note{

\begin{outline}

\1 First, a bit about terminology.

\2 This is a conditional distribution because we are asking what is the
probability that A occurs given that we know B occurs.

\2 This is a joint probability, it is asking what is the probability that BOTH A
and B jointly occur.

\2 And this a a marginal probability, it is asking what is the probability that
B occurs, without reference to A.

\2 So the probability of A knowing B is the probability of A and B occuring
divided by the probability of B occuring.

\1 We can rewrite this as...

\1 Let's see how this relates back to disjoint and independent sets.

\1 What if A and B are disjoint:

\2 Say a family member is about to have a baby and the A represents the event of
having a girl while B represent the event of having a boy.

\2 Then the probability of having a girl given that the baby is a boy is zero.

\2 We can relate this back to what we said earlier, what is the probability that
the baby is a boy and the baby is a girl?

\1 What if A and B are independent?

\1 This is like what we said earlier.  Suppose a I roll a die twice.  What I get
on the first roll doesn't affect the second roll.  So the probability of rolling
a 4 on the second roll and a 1 on the first roll is just the probability of
rolling a 4 on the second roll times rolling a 1 on the first roll.

\end{outline}

}

\end{frame}

%%%%%%%%%%%%%%%%%%%%%%%%%%%%%%%%%%%%

\begin{frame}
\frametitle{}

\vfill

\app{2.1 Probability and conditional probability}{$\:$\\ See the course website for instructions. \\$\:$}

\vfill

%---Note---
\note{

\begin{outline}

\1 State beforehand

\2 8 questions, brief.

\2 Show your work since we will review.

\2 If you get done, start reading about project I.

\2 Read through each question as you go along.  Bring up sakai side by side app ex.

\1 Go over table a bit.

\1 The questions

\2 2.  We are looking for people with BOTH of these attributes.  AND.

\2 3.  GIVEN.

\2 4. What is the definition of marginal probability.  Grab number from MARGINS
of the table.  Not JOINT.  Not CONDITIONAL.

p-value is a CONDITIONAL probability.

We will talk more about this on Wednesday.

\2 5. OR

\2 6. What is the difference between 5 and 6?  There is no difference.
Probability vs. PERCENT.

\2 7. 

\2 8. Bring up h-plot.

\2 9. Extra question: what is the response, what is the explanatory?  Probably go
from year to decision to vote.

\end{outline}

}

\end{frame}

%%%%%%%%%%%%%%%%%%%%%%%%%%%%%%%%%%%%

\section{Summary}

%%%%%%%%%%%%%%%%%%%%%%%%%%%%%%%%%%%%

\begin{frame}
\frametitle{Summary of main ideas}

\vfill

\begin{enumerate}

\item \nameref{mi1}

\item \nameref{mi2}

\item \nameref{mi3}

\end{enumerate}

\vfill

\end{frame}

%%%%%%%%%%%%%%%%%%%%%%%%%%%%%%%%%%%

\end{document}