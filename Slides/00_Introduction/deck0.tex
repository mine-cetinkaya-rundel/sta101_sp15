% -*- TeX-engine: xetex; -*-
% Compile with XeLaTeX

%%%%%%%%%%%%%%%%%%%%%%%
% To do before class
%%%%%%%%%%%%%%%%%%%%%%%

% Send the Logistics/Week0Annoucnement (the night before).
% Send an email reminding students to bring a charged computer (the night before).

%%%%%%%%%%%%%%%%%%%%%%%
% Option 1: Slides: (comment for handouts)   %
%%%%%%%%%%%%%%%%%%%%%%%

\documentclass[slidestop,compress,mathserif,12pt,t,professionalfonts,xcolor=table]{beamer}

% solution stuff
\newcommand{\solnMult}[1]{
\only<1>{#1}
\only<2->{\red{\textbf{#1}}}
}
\newcommand{\soln}[1]{\textit{#1}}

%%%%%%%%%%%%%%%%%%%%%%%%%%%%%%%
% Option 2: Handouts, without solutions (post before class)    %
%%%%%%%%%%%%%%%%%%%%%%%%%%%%%%%

% \documentclass[11pt,containsverbatim,handout,xcolor=xelatex,dvipsnames,table]{beamer}

% % handout layout
% \usepackage{pgfpages}
% \pgfpagesuselayout{4 on 1}[letterpaper,landscape,border shrink=5mm]

% % solution stuff
% \newcommand{\solnMult}[1]{#1}
% \newcommand{\soln}[1]{}

% % % This breaks things for me for some reason.
% % tell pgfpages how to set page sizes in XeLaTeX
% %\renewcommand\pgfsetupphysicalpagesizes{%
% %   \pdfpagewidth\pgfphysicalwidth\pdfpageheight\pgfphysicalheight%
% %}

%%%%%%%%%%%%%%%%%%%%%%%%%%%%%%%%%%%%
% Option 3: Handouts, with solutions (may post after class if need be)    %
%%%%%%%%%%%%%%%%%%%%%%%%%%%%%%%%%%%%

% \documentclass[11pt,containsverbatim,handout,xcolor=xelatex,dvipsnames,table]{beamer}

% % handout layout
% \usepackage{pgfpages}
% \pgfpagesuselayout{4 on 1}[letterpaper,landscape,border shrink=5mm]

% % solution stuff
% \newcommand{\solnMult}[1]{\red{\textbf{#1}}}
% \newcommand{\soln}[1]{\textit{#1}}

% % % This breaks things for me for some reason.
% % % tell pgfpages how to set page sizes in XeLaTeX
% % \renewcommand\pgfsetupphysicalpagesizes{%
% %    \pdfpagewidth\pgfphysicalwidth\pdfpageheight\pgfphysicalheight%
% % }

%%%%%%%%%%%%%%%%%%%%%%%%%%%%%%%
% Option 4: Notes Only
%%%%%%%%%%%%%%%%%%%%%%%%%%%%%%%

% % See http://tex.stackexchange.com/questions/114219/add-notes-to-latex-beamer
% \documentclass[10pt,containsverbatim,xcolor=xelatex,dvipsnames,table,notes=only]{beamer}

% % handout layout
% % \usepackage{pgfpages}
% % \pgfpagesuselayout{1 on 1}[letterpaper, landscape, border shrink=5mm]

% % solution stuff
% \newcommand{\solnMult}[1]{#1}
% \newcommand{\soln}[1]{}

% % % Having a problem with this.
% % tell pgfpages how to set page sizes in XeLaTeX
% % \renewcommand\pgfsetupphysicalpagesizes{%
% %   \pdfpagewidth\pgfphysicalwidth\pdfpageheight\pgfphysicalheight%
% %}

%%%%%%%%%%
% Load style file, defaults  %
%%%%%%%%%%

\input{../lec_style.tex}
% You cannot use numbers when defining variables.  Hence the use of letters, A, B, C, etc.

% Personal Info
\newcommand{\FirstName}{Mine}
\newcommand{\LastName}{\c{C}etinkaya-Rundel}
\newcommand{\OfficeHours}{MTWR 3-4pm.}
\newcommand{\OfficeHoursLocation}{Old Chem 213}

% Electronic Info
\newcommand{\PersonalSite}{http://stat.duke.edu/~mc301}
\newcommand{\CourseSite}{http://bitly.com/sta101sp15}
\newcommand{\Email}{mine@stat.duke.edu}

% TAs
\newcommand{\TAA}{Anthony Weishampel}
\newcommand{\TAB}{Fiamma Li}
\newcommand{\TAC}{Jialiang Mao}
\newcommand{\TAD}{Phillip Lee}

% Exam Dates
\newcommand{\ExamADate}{Wed, Feb 18}
\newcommand{\ExamBDate}{Wed, Mar 25}
\newcommand{\FinalDate}{Sat, May 2 (2-5pm)}

% Due Dates
\newcommand{\ClickerRegistrationDD}{Mon, Jan 26}
\newcommand{\GettingToKnowYouDD}{Friday, Jan 9, 11:59pm}
\newcommand{\ProblemSetADD}{Wed., 1/15}


% ALT ALT
% % You cannot use numbers when defining variables.  Hence the use of letters, A, B, C, etc.

% Personal Info
\renewcommand{\FirstName}{Jesse}
\renewcommand{\LastName}{Windle}
\renewcommand{\OfficeHours}{Tue, Thu 3:00pm-4:30pm}
\renewcommand{\OfficeHoursLocation}{Old Chem 211D}

% Electronic Info
\renewcommand{\PersonalSite}{http://stat.duke.edu/~jbw44/}
\renewcommand{\CourseSite}{http://bitly.com/windle2}
\renewcommand{\Email}{jbw44@stat.duke.edu}

% TAs
\renewcommand{\TAA}{David Clancy}
\renewcommand{\TAB}{Xinyi (Chris) Li}
\renewcommand{\TAC}{Tori Hall}
\renewcommand{\TAD}{Radhika Anand}

% Exam Dates
\renewcommand{\ExamADate}{Thu, Feb 19}
\renewcommand{\ExamBDate}{Thu, Mar 26}
\renewcommand{\FinalDate}{Mon, Apr 27 (9-Noon)}

% Due Dates
\renewcommand{\ClickerRegistrationDD}{Thu, Jan 15}
\renewcommand{\GettingToKnowYouDD}{Friday, Jan 9, 11:59pm}
\renewcommand{\ProblemSetADD}{Thu., 1/16}

%%%%%%%%%%%
% Cover slide info    %
%%%%%%%%%%%

\title{Data Analysis and Statistical Inference}
\subtitle{Introduction}
\author{Sta 101 - Spring 2015}
\date{January 7, 2015}
% ALT ALT
% \date{January 8, 2015}
\institute{Duke University, Department of Statistical Science}


%%%%%%%%%%%%%%%%%%%%%%%%%
% Begin document and set Helvetica Neue font   %
%%%%%%%%%%%%%%%%%%%%%%%%%

\begin{document}
\fontspec[Ligatures=TeX]{Helvetica Neue Light}

%%%%%%%%%%%%%%%%%%%%%%%%%%%%%%%%%%%

% Title Page

\begin{frame}[plain]

\titlepage
\vfill
{\scriptsize \webLink{\PersonalSite}{Dr. \LastName{}} \hfill Slides posted at  \webLink{\CourseSite}{\CourseSite}}
\addtocounter{framenumber}{-1} 

\end{frame}

%%%%%%%%%%%%%%%%%%%%%%%%%%%%%%%%%%%

\section{Course info}

% THIS SECTION CONTAINS COURSE/SYLLABUS INFO
% GO OVER IT QUICKLY AS IT IS LIKELY INFORMATION OVERLOAD
% BUT USEFUL FOR STUDENTS TO HEAR ON THE FIRST DAY
% ESTIMATE - 20 MINUTES

%%%%%%%%%%%%%%%%%%%%%%%%%%%%%%%%%%%

\subsection{General info}

%%%%%%%%%%%%%%%%%%%%%%%%%%%%%%%%%%%

\begin{frame}
\frametitle{Teaching team}

\begin{itemize}

\item Professor: Dr. \FirstName{} \LastName{} - \mail{\Email{}}

\item TAs:
\begin{itemize}
\item \TAA{}
\item \TAB{}
\item \TAC{}
\item \TAD{}
\end{itemize}

\end{itemize}

%---NOTE---%
\note{

  \begin{itemize}
    \item State your name and title.
    \item State what your research interests are.
    \item State any other useful/interesting information about yourself.
    \item Introduce the TAs.
    \item Mention that you sent an email describing the materials and list of to-dos.
  \end{itemize}

}

\end{frame}

%%%%%%%%%%%%%%%%%%%%%%%%%%%%%%%%%%%

\begin{frame}
\frametitle{Required materials}

\begin{itemize}

\item OpenIntro Statistics, 2nd Edition

\item i$>$clicker2 - See Google Doc for a list of students selling used clickers (link emailed)

\item (optional) Calculator
% ALT ALT
% \item Calculator (just something that can do square roots)

\end{itemize}

%---Note---%
\note{

  \begin{itemize}
  \item Mention book
  \item SHOW Clicker.  iClicker2 not another version.
  \item Google doc for used clickers on syllabus.
  \item Make sure to take your name off the list if you purchase a clicker.
  \item Calculator: can get for under \$6 on Amazon (prime even).
  \end{itemize}

}

\end{frame}

%%%%%%%%%%%%%%%%%%%%%%%%%%%%%%%%%%%

\begin{frame}
\frametitle{Webpage}

\vfill

\centering
{\Large 
\webURL{\CourseSite}
}

\vfill

%---Note---%
\note{

  \begin{itemize}
  \item Please bring a computer: use for in-class work.
  \item You can read notes.
  \item Okay to use for class-related purposes, not for surfing the web.
  \item If you watch a cat video, it will draw the attention of those around you.  That is distracting.  And I will know...
  \item If not on task I or TA will tap you on your shoulder and ask you to stop.
  \end{itemize}

}

\end{frame}

%%%%%%%%%%%%%%%%%%%%%%%%%%%%%%%%%%%

\begin{frame}
\frametitle{Grading}

\begin{center}
\rowcolors{1}{}{custom_lightGray}
\renewcommand\arraystretch{1.25}
{\scriptsize
\begin{tabular}{ r | l }
\textbf{Component} & \textbf{Weight} \\
Attendance \& participation + peer evaluation	& 7.5\% \\
Problem sets							& 10\%  \\ 
Labs									& 10\% \\    
Readiness assessments					& 10\%   \\  
Performance assessments 				& 2.5\%  \\  
Project 1								& 5\%   \\   
Project 2								& 10\% \\   
Midterm 1								& 10\% \\    
Midterm 2 							& 10\% \\    
Final 								& 25\%     
\end{tabular}
}
\end{center}

\begin{itemize}

\item Grades may be curved at the end of the semester. 

\item Cumulative numerical averages of 90 - 100 are guaranteed at least an A-, 80 - 89 at least a B-, and 70 - 79 at least a C-, however the exact ranges for letter grades will be determined after the final exam. 
% ALT ALT
% \item The exact ranges for letter grades will be determined after the final exam.

\item The more evidence there is that the class has mastered the material, the more generous the curve will be.

\end{itemize}



\end{frame}

%%%%%%%%%%%%%%%%%%%%%%%%%%%%%%%%%%%

\subsection{Goals}

%%%%%%%%%%%%%%%%%%%%%%%%%%%%%%%%%%%

\begin{frame}
\frametitle{Course goals and objectives}

{\footnotesize
\begin{itemize}[<alert@+>]
\item Recognize the importance of data collection, identify limitations in data collection methods, and determine how they affect the scope of inference.
\item Use statistical software to summarize data numerically and visually, and to perform data analysis.
\item Have a conceptual understanding of the unified nature of statistical inference.
\item Apply estimation and testing methods to analyze single variables or the relationship between two variables in order to understand natural phenomena and make data-based decisions.
\item Model numerical response variables using a single or multiple explanatory variables.
\item Interpret results correctly, effectively, and in context without relying on statistical jargon.
\item Critique data-based claims and evaluate data-based decisions.
\item Complete two research projects: one that focuses on statistical inference and one that focuses on modeling. 
\end{itemize}
}

%---Note---&
\note{
  Need to add note.
}

\end{frame}

%%%%%%%%%%%%%%%%%%%%%%%%%%%%%%%%%%%

\begin{frame}
\frametitle{Learning units and course outline}

{\footnotesize
\begin{itemize}[<+->]
\item \hl{Unit 1 - Intro to data:} Observational studies and non-causal inference, principles of experimental design and causal inference, exploratory data analysis, and introduction to simulation-based statistical inference
\item \hl{Unit 2 - Probability \& distributions:} Basics of probability and chance processes, Bayesian perspective in statistical inference, the normal and binomial distributions
\item \hl{Unit 3 - Framework for inference:} CLT, sampling distributions, and introduction to theoretical inference
\begin{itemize}
\item Midterm 1
\end{itemize}
\item \hl{Unit 4 - Statistical inference for numerical variables}
\item \hl{Unit 5 - Statistical inference for categorical variables}
\begin{itemize}
\item Project 1 \& Midterm 2
\end{itemize}
\item \hl{Unit 6 - Simple linear regression:} Bivariate correlation and causality, introduction to modeling
\item \hl{Unit 7 - Multiple linear regression:} More advanced modeling with multiple predictors
\begin{itemize}
\item Project 2 \& Final
\end{itemize}
\end{itemize}
}

%---Note---%
\note{
  \begin{itemize}
    \item We can think of breaking this down into chunks.
      \begin{itemize}
        \item Pictures and other summaries of data.
        \item Mathematics behind statistics.
        \item A common framework for rigorously answering research questions.
      \end{itemize}
  \end{itemize}
}

\end{frame}

%%%%%%%%%%%%%%%%%%%%%%%%%%%%%%%%%%%

\subsection{Course structure and components}

%%%%%%%%%%%%%%%%%%%%%%%%%%%%%%%%%%%

\begin{frame}
\frametitle{Course structure}

\begin{itemize}[<alert@+>]
\item Set of learning objectives and required and suggested readings, videos, etc. for each unit
\item Prior to beginning the unit, watch the videos and/or complete the readings and familiarize yourselves with the learning objectives
\item Begin a new unit with a readiness assessment: individual, then team 
\item Class time: split between lecture, discussion/application, and lab
\item Complement your learning with problem sets
\item Wrap up a unit with a performance assessment
\end{itemize}

%---Note---%
\note{
  \begin{itemize}
    \item Go over process from start to finish.
    \item In the remaining classes I will talk less and you will do more.
  \end{itemize}
}

\end{frame}

%%%%%%%%%%%%%%%%%%%%%%%%%%%%%%%%%%%

\begin{frame}
\frametitle{Teams}

\begin{itemize}
\item Highly functional teams of learners based on survey and pre-test

\item Team members first point of contact

\item Application exercises, labs, team readiness assessments, projects

\item Study together, but anything that is not explicitly a team assignment must be your own work

\item Peer evaluations to ensure that all team members contribute to the success of the group and to address any potential issues early on
\begin{itemize}
\item If you feel that there are issues within your team, you are encouraged to discuss it with your team members and to bring it to my or your TA's attention ASAP (don't wait till things get worse)
\end{itemize}

\end{itemize}

%---Note---%
\note{

  \begin{itemize}
  \item Describe how teams are chosen.
  \item \; \; We try to find commonalities: year, age, location.
  \item \; \; And complementaries: calculus experience, statistics experience, programming experience.
  \item Rely on team members, but make sure on individual assignments to try first before asking for help.
  \item Peer evaluations.
  \end{itemize}

}

\end{frame}

%%%%%%%%%%%%%%%%%%%%%%%%%%%%%%%%%%%

\begin{frame}
\frametitle{Clickers}

\red{Objective:} Two-way communication and instant feedback

\begin{itemize}
\item Readiness assessments (graded for accuracy)

\item Questions throughout lecture (graded for participation)
\begin{itemize}
\item to get credit for the day you must respond to at least 75\% of the questions
\item up to three unexcused late arrivals or absences will not affect your clicker grade
\end{itemize}

\item Register your clicker
\begin{itemize}
\item \webURL{https://www1.iclicker.com/register-clicker} (Student ID = Net ID)
\item grading starts \ClickerRegistrationDD{}
\end{itemize}

\end{itemize}

%---Note---%
\note{
  \begin{itemize}
  \item Go over how they will be used for readiness assessments.
  \item Used to answer questions in lecture
  \item I will use them to figure out if we need to talk more about something.
  \item Register your clicker - quick demo?
  \end{itemize}
}

\end{frame}

%%%%%%%%%%%%%%%%%%%%%%%%%%%%%%%%%%%

\begin{frame}
\frametitle{Attendance \& participation}

\red{Objective:} Make you an active participant and help me pace the class 

\begin{itemize}

\item Attendance and participation during class, as well as your activity on Piazza make up a non-insignificant portion of your grade in this class

\item Might sometimes call on you during the class discussion, however it is your responsibility to be an active participant without being called on

\end{itemize}

%---Note---%
\note{
  You may get called on.
}

\end{frame}

%%%%%%%%%%%%%%%%%%%%%%%%%%%%%%%%%%%

\begin{frame}
\frametitle{Problem sets (PS)}

\red{Objective:} Help you develop a more in-depth understanding of the material and help you prepare for exams and projects

\begin{itemize}

\item Questions from the textbook

\item Show \emph{all} your work to receive credit

\item \hl{Required format:} Use one of the following, no other submission types will be accepted
\begin{itemize}
\item Type your answers in the text box on Sakai and attach any plots/images as separate files, properly named
\item Attach a PDF (\emph{not} Word, Google Doc, etc.) of your answers
\end{itemize}

\item Welcomed and encouraged to work with others, but turn in your own work

\item No make-ups, excused absences (e.g. STINF) do not excuse homework

\item Lowest PS score will be dropped

\end{itemize}

%---Note---%
\note{
  \begin{itemize}
  \item Do not submit a word document.
  \item Save often and keep a backup of your work.
  \item Try to do the problems on your own first before talking to others.
  \item STINF: 24 hour extension.
  \end{itemize}
}

\end{frame}

%%%%%%%%%%%%%%%%%%%%%%%%%%%%%%%%%%%

\begin{frame}[fragile]
\frametitle{Labs}

\red{Objective:} Give you hands on experience with data analysis using statistical software and provide you with tools for the projects

\begin{itemize}

\item Work in teams: author / discussants

\item Must be present in lab session to get credit

\item Lowest lab score will be dropped

\end{itemize}

\vfill

\pause

\activity{Get started with R/RStudio}{
\begin{itemize}
    \setlength{\itemsep}{0pt}
    \setlength{\parskip}{0pt}
\item Go to the course website, \webURL{\CourseSite}, click on the RStudio link (top right)
\begin{itemize}
\item {\footnotesize Make sure you're on the Duke network, not visitor}
\end{itemize}
\item Log in using your Net ID and password
\item In the Console, generate a random number between 1 and 5, and introduce yourself to that many people sitting around you:
\Rcode{sample(1:5, size = 1)}
\end{itemize}
}

%---Note---%
\note{
  \begin{itemize}
  \item Reiterate that you MUST be in lab to get credit.
  \end{itemize}

  \begin{itemize}
  \item Go to course website (create a clicker question?)
  \item Click on Rstudio.
  \item We may crash the system.  Post on Piazza if this ever occurs and I will try to take care of it.
  \item Explain what is going on in Rstudio a bit.
  \item You tell R what to do.  It will yell at your crypticall if you get it wrong. (CASE/spelling).
  \item Sample: %\texttt{sample(1:5, size=1)}
    \begin{itemize}
    \item Does everyone get the same answer?
    \item No? Why not?  (Because it is random)
    \end{itemize}
  \item Take 2 minutes to introduce yourself.
  \end{itemize}
}

\end{frame}

%%%%%%%%%%%%%%%%%%%%%%%%%%%%%%%%%%%

\begin{frame}
\frametitle{Readiness assessments (RA)}

\red{Objective:} Encourage you to watch the videos and/or complete the reading assignment and review the learning objectives prior to coming to class as well as evaluate your conceptual understanding of the unit's material

\begin{itemize}

\item 10 multiple choice questions, at the beginning of a unit

\item Conceptual questions addressing the learning objectives of the new unit, assessing familiarity and reasoning, not mastery

\item Take the individual RA using clickers, then re-take in teams

\item Individual RA score 3/4 of grade, team RA score 1/4 \& your input during the team portion will factor into your participation grade

\item Lowest RA score will be dropped

\end{itemize}

%---Note---%
\note{
  \begin{itemize}
    \item The questions aren't intended to be super difficult, but you will need to study.
    \item We have one coming up this Tuesday.  Make sure to study to get an idea of what you need to do for the future RA.
    \item Watching a set of videos is roughly 1.5-2 hours.
    \item Get you clickers by then so you can practice using them.
    \item Look at the learning objectives.
    \item Questions will pop-up to check in, but not recorded.
  \end{itemize}
}

\end{frame}

%%%%%%%%%%%%%%%%%%%%%%%%%%%%%%%%%%%

\begin{frame}
\frametitle{Performance assessments (PA)}

\red{Objective:} Evaluate your mastery of the material by the end of a unit and give you instant feedback on your performance.

\begin{itemize}

\item 10 multiple choice questions, at the end of a unit

\item Taken individually on Sakai

\item Lowest PA score will be dropped

\end{itemize}

%---Note---%
\note{
Unlike RA, this is trying to gauge if you have mastered the material.
}

\end{frame}

%%%%%%%%%%%%%%%%%%%%%%%%%%%%%%%%%%%

\begin{frame}
\frametitle{Projects}

\red{Objective:} Give you independent applied research experience using real data and statistical methods

\begin{itemize}

\item Project 1: For a parameter of interest to you, you will describe the relevant data, compute a confidence interval and conduct a hypothesis test, and summarize your findings in a written, fully reproducible, data analysis report

\item Project 2: Use all (relevant) techniques learned in this class to analyze a dataset provided by me, and share your results in a poster session

\item Must complete both projects and score at least 30\% of the points on each project in order to pass this class

\end{itemize}

%---Note---%
\note{
  \begin{itemize}
    \item The first project you will be coming up with a research question and data set.
    \item Keep an eye out for interesting data.
    \item The second project you will work on data set provided by me.
  \end{itemize}
}

\end{frame}

%%%%%%%%%%%%%%%%%%%%%%%%%%%%%%%%%%%

\begin{frame}
\frametitle{Exams}

\begin{center}
\rowcolors{1}{}{custom_lightGray}
\renewcommand\arraystretch{1.25}
{\footnotesize
\begin{tabular}{ r | l }
Midterm 1								& \ExamADate \\    
Midterm 2 							& \ExamBDate \\    
Final 								& \FinalDate   
\end{tabular}
}
\end{center}

\begin{itemize}

\item Exam dates cannot be changed, no make-up exams will be given

\item If you cannot take the exams on these dates you should drop this class

\item Calculator + cheat sheet allowed

\end{itemize}

%---Note---%
\note{
  \begin{itemize}
    \item Cousins aren't allowed to have weddings, brother is not allowed to graduate.
    \item If medical issues come up, you need to work through your dean.
    \item Final on Mon. Apr. 27, 9am-noon.
  \end{itemize}
}

\end{frame}

%%%%%%%%%%%%%%%%%%%%%%%%%%%%%%%%%%%

\subsection{Support}

%%%%%%%%%%%%%%%%%%%%%%%%%%%%%%%%%%%

\begin{frame}
\frametitle{Email \& Piazza}

\begin{itemize}

\item I will regularly send announcements by email, so make sure to check your email  daily

\item Any \emph{non-personal} questions related to the material covered in class, problem sets, labs, projects, etc. should be posted on Piazza forum

\item Before posting a new question please make sure to check if your question has already been answered, and answer others' questions

\item Use informative titles for your posts

\item It is more efficient to answer most statistical questions ``in person" so make use of OH

\end{itemize}

%---Notes---%
\note{

  \begin{itemize}
    \item Regularly check your email.  I will be sending out important communications.
    \item All content related questions should go on piazza.
    \item Title your question informatively.  ``Confused'' or ``Question'' is not informative.
  \end{itemize}
  
}

\end{frame}

%%%%%%%%%%%%%%%%%%%%%%%%%%%%%%%%%%%

\begin{frame}
\frametitle{Office Hours}

\begin{itemize}
\item Prof. \LastName{}: \OfficeHours{}
\item TAs: TBD
\end{itemize}

\end{frame}

%%%%%%%%%%%%%%%%%%%%%%%%%%%%%%%%%%%

\begin{frame}
\frametitle{Students with disabilities}

Students with disabilities who believe they may need accommodations in this class are encouraged to contact the \webLink{http://www.access.duke.edu/students/requesting/index.php}{Student Disability Access Office} at (919) 668-1267 as soon as possible to better ensure that such accommodations can be made

\vfill

\ct{\webURL{http://www.access.duke.edu/students/requesting/index.php}}

%---Note---%
\note{
  If you need extra time contact the student disability access office immediately so that they can contact me.
}

\end{frame}

%%%%%%%%%%%%%%%%%%%%%%%%%%%%%%%%%%%

\subsection{Policies}

%%%%%%%%%%%%%%%%%%%%%%%%%%%%%%%%%%%

\begin{frame}
\frametitle{Late work policy}

\begin{itemize}

\item Late work policy for problem sets and labs reports:
\begin{itemize}
\item next day: lose 30\% of points (within 24 hours of due date)
\item later than next day: lose all points
\end{itemize}

\item Late work policy for projects: 20\% off for each day late

\end{itemize}

\end{frame}

%%%%%%%%%%%%%%%%%%%%%%%%%%%%%%%%%%%

\begin{frame}
\frametitle{Regrade policy}

Regrade requests must be made \hl{within 3 days} of when the assignment is returned, and must be submitted to me in writing 

\begin{itemize}

\item These will be honored if points were tallied incorrectly, or if you feel your answer is correct but it was marked wrong

\item No regrade will be made to alter the number of points deducted for a mistake

\item There will be no grade changes after the final exam

\end{itemize}

\end{frame}

%%%%%%%%%%%%%%%%%%%%%%%%%%%%%%%%%%%

\begin{frame}
\frametitle{Make up policy}

\begin{itemize}

\item No make-up for attendance, individual and team readiness assessments, labs, problem sets, projects, or exams

\item If the midterm exam must be missed due to a documented medical excuse, absence must be officially excused \hl{in advance}, in which case the missing exam score will be imputed using the final exam score

\item The final exam must be taken at the stated time

\item You must take the final exam and turn in the projects in order to pass this course

\end{itemize}

%---Note---%
\note{
  If you have any issue come up, you must go through your dean.
}

\end{frame}

%%%%%%%%%%%%%%%%%%%%%%%%%%%%%%%%%%%

\begin{frame}
\frametitle{Other policies}

\begin{itemize}

\item Clickers may not be shared, and the clicker registered to a person may only be used by that person, failure to abide by this will result in a 0 clicker grade for everyone involved

\item Use of disallowed materials (textbook, class notes, web references, any form of communication with classmates or other persons, etc.) during exams will not be tolerated

\end{itemize}

%---Note---%
\note{
  You get three days to miss your clicker participation.
}

\end{frame}

%%%%%%%%%%%%%%%%%%%%%%%%%%%%%%%%%%%%

\begin{frame}
\frametitle{Academic Dishonesty}

Any form of academic dishonesty will result in an immediate 0 on the given assignment and will be reported to the Office of Student Conduct. Additional penalties may also be assessed if deemed appropriate. If you have any questions about whether something is or is not allowed, ask me beforehand.

Some examples:

\begin{itemize}

\item Use of disallowed materials (including any form of communication with classmates or accessing the web) during exams and readiness assessments

\item Plagiarism of any kind

\item Use of outside answer keys or solution manuals for the homework

\end{itemize}

\end{frame}

%%%%%%%%%%%%%%%%%%%%%%%%%%%%%%%%%%%%

\subsection{Tips for success}

%%%%%%%%%%%%%%%%%%%%%%%%%%%%%%%%%%%%

\begin{frame}
\frametitle{Tips for success}

{\footnotesize
\begin{itemize}[<alert@+>]
\item Complete the reading before a new unit begins, and then review again after the unit is over.
\item Be an active participant during lectures and labs.
\item Ask questions - during class or office hours, or by email. Ask me, your TAs, and your classmates.
\item Do the problem sets - start early and make sure you attempt and understand all questions.
\item Start your projects early and and allow adequate time to complete them.
\item Give yourself plenty of time time to prepare a good cheat sheet for exams. This requires going through the material and taking the time to review the concepts that you're not comfortable with.
\item Do not procrastinate - don't let a unit go by with unanswered questions as it will just make the following unit's material even more difficult to follow. 
\end{itemize}
}

\end{frame}

%%%%%%%%%%%%%%%%%%%%%%%%%%%%%%%%%%%%

\subsection{To do}

%%%%%%%%%%%%%%%%%%%%%%%%%%%%%%%%%%%%

\begin{frame}
\frametitle{To do}

\begin{itemize}

\item Download or purchase the textbook

\item Obtain and register your clicker
\begin{itemize}
\item \webURL{https://www1.iclicker.com/register-clicker} (Student ID = Net ID)
\end{itemize}

\item Complete the following by \GettingToKnowYouDD{}
\begin{itemize}
\item Pretest
% ALT ALT
% \item Pretest ({\color{red}Due Sunday at 11:59pm})
\item Getting to know you survey
\item Performance assessment 0 (on course policies etc., not graded, for practice with the quiz module on Sakai)
\end{itemize}

\item Read the syllabus and let me know if you have any questions

\item Watch/Read/Review the resources for Unit 1

\end{itemize}

%---Note---%
\note{
  Go over to do.
}

\end{frame}

%%%%%%%%%%%%%%%%%%%%%%%%%%%%%%%%%%%%

\section{Example: Baby names}

%%%%%%%%%%%%%%%%%%%%%%%%%%%%%%%%%%%%

\begin{frame}
\frametitle{Baby names in the US}

\begin{itemize}

\item Each year the Social Security Administration collects and releases data on the how many babies are given a certain name

\item They released these data for years 1880 onwards for each gender

\item For privacy reasons they restrict the list of names to those with at least 5 occurrences

\end{itemize}

%---Note---%
\note{

  \begin{itemize}
    \item Go through points.
    \item ... so Blue Iny probably isn't on the list.
  \end{itemize}

}

\end{frame}

%%%%%%%%%%%%%%%%%%%%%%%%%%%%%%%%%%%%

\begin{frame}
\frametitle{Top 10 baby names for 2013}

\begin{center}
\includegraphics[width=0.8\textwidth]{figures/babynames2013}
\end{center}

\ct{\webURL{http://www.ssa.gov/oact/babynames}}

%---Note---%
\note{

  \begin{itemize}
    \item Top Baby names of 2013
    \item Go over female names.
    \item Ask if there is anyone with these names.
    \item Do the same for men.
    \item Tell them that they are all ahead of the fashion curve for names.
  \end{itemize}

}

\end{frame}

%%%%%%%%%%%%%%%%%%%%%%%%%%%%%%%%%%%%

\begin{frame}
\frametitle{Name voyager}

\begin{center}
\includegraphics[width=0.8\textwidth]{figures/namevoyager_mine}
\end{center}

\ct{\webURL{http://www.babynamewizard.com/voyager}}

%---Note---%
\note{

  \begin{itemize}
    \item My friend that created these slides is named Mine.
    \item Closest thing is Minerva.
    \item Really popular back in the day.  Now there are none.
    \item Does anyone here know a Minerva?
    \item If you want to child to have a unique name, think about Minerva.
  \end{itemize}

}

\end{frame}

%%%%%%%%%%%%%%%%%%%%%%%%%%%%%%%%%%%%

\begin{frame}
\frametitle{Names and ages}

\begin{center}
\includegraphics[width=0.8\textwidth]{figures/nameage538}
\end{center}

\ct{\webURL{http://fivethirtyeight.com/features/how-to-tell-someones-age-when-all-you-know-is-her-name}}

%---Note---%
\note{
\small

  \begin{itemize}
    \item Name is just one piece of information.
    \item It is often interesting to look at two pieces of information together. What if we combine name and year?
    \item Is anyone familiar with 538?  What is it?
    \item Nate Silver started this blog back in the 2008 election of Obama vs. McCain.  538 refers to the number of votes in the electoral college.  Since then it has branched out into all sorts of ``data news''.
    \item This article... How to Tell Someone's age... tries to figure out what we can tell about someone's age, given their name.
    \item If I told you that you were about to meet Gertrude, what would you guess her age to be?  60, 70?
    \item You can use SSD data and actuarial tables, which tell you about how long someone is going to live to figure out the distribution of ages for a given name.
  \end{itemize}

}

\end{frame}

%%%%%%%%%%%%%%%%%%%%%%%%%%%%%%%%%%%%

% THE FOLLOWING 4 SLIDES CONTAIN IMAGES EDITED FOR THIS COURSE
% EITHER UPDATE IMAGES WITH NAME COUNTS FROM YOUR COURSE
% OR OMIT THESE IMAGES AND JUST DISCUSS THE ARTICLE GENERALLY

\begin{frame}
\frametitle{}

\vspace{-0.45cm}

\begin{center}
\includegraphics[width=0.7\textwidth]{figures/popnamesclass538_female}
\end{center}

%---Note---%
\note{
\small

  \begin{itemize}
  \item Here we have a plot of names, their interquartile ranges, and the their medians.
  \item Basically we have a data set in which each observation is the name of an individual and her age.
  \item Does anyone know what a percentile is?
  \item There are actually a couple definitions of percentile.
  \item All of them ultimately have to do with expressing how many observations in our data set our below a certain threshold.
  \item We take all the Anna's and order them by age from youngest to oldest.
  \item We then look at the cutoff age for a certain fraction of the observations.
  \item For the name Anna, the 25th percentile is 14 years.
  \item That means that 25\% of the Anna's are 14 years of age or younger.
  \item For the name Anna, the 75th percentile is 62 years.
  \item That means that 75\% of the Anna's are 62 years of age or younger.
  \item The median is the 50\% percentile.
  \item 50\% of Anna's are 31 years of age or younger.
  \item Women of wisdom.
  \end{itemize}

}

\end{frame}

%%%%%%%%%%%%%%%%%%%%%%%%%%%%%%%%%%%%

\begin{frame}
\frametitle{}

\vspace{-0.45cm}

\begin{center}
\includegraphics[width=0.7\textwidth]{figures/popnamesclass538_male}
\end{center}

  \begin{itemize}
  \item If I didn't know anything about you, I would guess that...
  \item Men of maturity.  
  \item I don't know you that well so that is probably up for debate.
  \end{itemize}

\end{frame}

%%%%%%%%%%%%%%%%%%%%%%%%%%%%%%%%%%%%

\begin{frame}
\frametitle{}

\begin{center}
\includegraphics[width=\textwidth]{figures/youngnamesclass538_female}
\end{center}

%---Notes---%
\note{

  Your parents were ahead of their time in naming you.

  All of these other people are just copycats.
  
  For Madison the 75th percentile doesn't even get up to 15.

}

\end{frame}

%%%%%%%%%%%%%%%%%%%%%%%%%%%%%%%%%%%%

\begin{frame}
\frametitle{}

\begin{center}
\includegraphics[width=\textwidth]{figures/youngnamesclass538_male}
\end{center}

\end{frame}

%%%%%%%%%%%%%%%%%%%%%%%%%%%%%%%%%%%%

\section{Example: Geotagged data}

%%%%%%%%%%%%%%%%%%%%%%%%%%%%%%%%%%%%

\begin{frame}
\frametitle{}

\clicker{Do you geotag your posts on social networking sites, like Facebook, 
Twitter, Instagram, etc.?}

\begin{enumerate}[(a)]
\item yes
\item no
\end{enumerate}

%---Note---%
\note{
  \begin{itemize}
  \item Explain how clickers work.
  \item Do the countdown.
  \item Look at your results.
  \item Why don't you geotag your posts.
  \item Hey, look, I'm visiting friends, rob me!
  \end{itemize}
}

\end{frame}

%%%%%%%%%%%%%%%%%%%%%%%%%%%%%%%%%%%%

\begin{frame}
\frametitle{Maps based on clicker tags}

\twocol{0.65}{0.35}
{
\begin{center}
\includegraphics[width=\textwidth]{figures/flickr_ny}
\end{center}
}
{
\begin{itemize}
\item[] \textcolor{red}{tourists}
\item[] \textcolor{blue}{local}
\item[] \textcolor{yellow}{both}
\end{itemize}
}

\ct{\webURL{http://aaronstraupcope.com}}

%---Note---%
\note{
\small

  \begin{itemize}
  \item Is anyone here from the New York Region?  What is this a map of?
  \item This image was created using data from Flickr, which is where people posted pictures before instagram and Facebook.
  \item Along with the pictures and geotagging info, there is text posted by the photographer.
  \item Based on that text we can categorize the photographer as a tourist or native.
  \item Where are the tourists hanging out? Broadway, Times Square, South Street Port
  \item Where are the locals: anywhere but Times Square.  At home.
  \item Staten Island is a crap shoot.
  \item Why are there no pictures in the upper middle right?  Riker prison.  You can't take pictures in prison.
  \item Why is this data useful: targetted marketing.  Serve up adds based on WHO you are and WHERE you are.
  \end{itemize}

}

\end{frame}

%%%%%%%%%%%%%%%%%%%%%%%%%%%%%%%%%%%%

\section{Why study statistics?}

%%%%%%%%%%%%%%%%%%%%%%%%%%%%%%%%%%%%

\begin{frame}
\frametitle{Why study statistics?}

\begin{center}
\includegraphics[width=0.55\textwidth]{figures/lottery}
\end{center}

%---Note---%
\note{
  
  \begin{itemize}
    \item I will not be showing you how to win the lottery.
    \item If I knew how to win the lottery using statistics, I wouldn't be here.
    \item If anything, I will be trying to convince you not to play the lottery.
  \end{itemize}

}

\end{frame}

%%%%%%%%%%%%%%%%%%%%%%%%%%%%%%%%%%%%

\begin{frame}
\frametitle{Why study statistics?}

\begin{center}
\includegraphics[width=0.7\textwidth]{figures/why_stats}
\end{center}

\vspace{-0.25cm}

\ct{\webURL{http://thisisstatistics.org}}

%---Note---%
\note{

  \begin{itemize}
    \item This website has very good information on waht you can do with statistics.
    \item Data scienctist.  Fancy name for statistician.  Hot job in silicon valley.
    \item Policy, medicine, biology, psychology.
    \item Go over growth: growing faster --- more jobs.
    \item Go over wages: wages growing faster.
  \end{itemize}

}

\end{frame}

%%%%%%%%%%%%%%%%%%%%%%%%%%%%%%%%%%%%

\section{Class survey [time permitting]}

%%%%%%%%%%%%%%%%%%%%%%%%%%%%%%%%%%%%

\begin{frame}
\frametitle{}

\activity{Class survey}

{
\begin{itemize}
    \setlength{\itemsep}{0pt}
    \setlength{\parskip}{0pt}
\item One of your first tasks in this class is to help design a survey. This survey will be completed \emph{anonymously}. It will (ideally) have information on \emph{variables you are interested in}. When writing your question consider whether you would feel comfortable answering it on an anonymous survey.

\item Work with 3-4 classmates to come up with a survey question, and add it to Google Doc linked below. Make sure that the wording of the question is clear, and (if categorical) the answer choices make sense.
\begin{center}
\webURL{http://bit.ly/sta101sp15_ClassSurvey}
\end{center}

\item Before adding a question check to make sure that it hasn't already been added. If your question is already there, but you can suggest a clearer / better wording, add it as ``alternative wording'' underneath the original question.
\end{itemize}
}

%---Note---%
\note{

  \begin{itemize}
    \item Work in informal teams of 3-4.
    \item Go over examples of types of questions.
    \item Come up with questions.
    \item Do not put questions you (or another) would be uncomfortable answering.
    \item But try to keep the questions interesting.
    \item We will try to compile a set of reasonable questions for you all to answer.
    \item You will fill out the survey and then we will use that data for examples during the semester.
  \end{itemize}

}

\end{frame}

%%%%%%%%%%%%%%%%%%%%%%%%%%%%%%%%%%%%

\end{document}