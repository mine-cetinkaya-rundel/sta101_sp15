% -*- TeX-engine: xetex; -*-
% Compile with XeLaTeX

%%%%%%%%%%%%%%%%%%%%%%%
% To do before class
%%%%%%%%%%%%%%%%%%%%%%%

% Send the Logistics/Week0Annoucnement (the night before).
% Send an email reminding students to bring a charged computer (the night before).

%%%%%%%%%%%%%%%%%%%%%%%
% Option 1: Slides: (comment for handouts)   %
%%%%%%%%%%%%%%%%%%%%%%%
%
%\documentclass[slidestop,compress,mathserif,12pt,t,professionalfonts,xcolor=table]{beamer}
%
%% solution stuff
%\newcommand{\solnMult}[1]{
%\only<1>{#1}
%\only<2->{\red{\textbf{#1}}}
%}
%\newcommand{\soln}[1]{\textit{#1}}

%%%%%%%%%%%%%%%%%%%%%%%%%%%%%%%
% Option 2: Handouts, without solutions (post before class)    %
%%%%%%%%%%%%%%%%%%%%%%%%%%%%%%%

 \documentclass[11pt,containsverbatim,handout,xcolor=xelatex,dvipsnames,table]{beamer}

 % handout layout
 \usepackage{pgfpages}
 \pgfpagesuselayout{4 on 1}[letterpaper,landscape,border shrink=5mm]

 % solution stuff
 \newcommand{\solnMult}[1]{#1}
 \newcommand{\soln}[1]{}

 % % This breaks things for me for some reason.
 % tell pgfpages how to set page sizes in XeLaTeX
 \renewcommand\pgfsetupphysicalpagesizes{%
    \pdfpagewidth\pgfphysicalwidth\pdfpageheight\pgfphysicalheight%
}

%%%%%%%%%%%%%%%%%%%%%%%%%%%%%%%%%%%%
% Option 3: Handouts, with solutions (may post after class if need be)    %
%%%%%%%%%%%%%%%%%%%%%%%%%%%%%%%%%%%%

% \documentclass[11pt,containsverbatim,handout,xcolor=xelatex,dvipsnames,table]{beamer}

% % handout layout
% \usepackage{pgfpages}
% \pgfpagesuselayout{4 on 1}[letterpaper,landscape,border shrink=5mm]

% % solution stuff
% \newcommand{\solnMult}[1]{\red{\textbf{#1}}}
% \newcommand{\soln}[1]{\textit{#1}}

% % % This breaks things for me for some reason.
% % % tell pgfpages how to set page sizes in XeLaTeX
% % \renewcommand\pgfsetupphysicalpagesizes{%
% %    \pdfpagewidth\pgfphysicalwidth\pdfpageheight\pgfphysicalheight%
% % }

%%%%%%%%%%%%%%%%%%%%%%%%%%%%%%%
% Option 4: Notes Only
%%%%%%%%%%%%%%%%%%%%%%%%%%%%%%%

% % See http://tex.stackexchange.com/questions/114219/add-notes-to-latex-beamer
% \documentclass[10pt,containsverbatim,xcolor=xelatex,dvipsnames,table,notes=only]{beamer}

% % handout layout
% % \usepackage{pgfpages}
% % \pgfpagesuselayout{1 on 1}[letterpaper, landscape, border shrink=5mm]

% % solution stuff
% \newcommand{\solnMult}[1]{#1}
% \newcommand{\soln}[1]{}

% % % Having a problem with this.
% % tell pgfpages how to set page sizes in XeLaTeX
% % \renewcommand\pgfsetupphysicalpagesizes{%
% %   \pdfpagewidth\pgfphysicalwidth\pdfpageheight\pgfphysicalheight%
% %}

%%%%%%%%%%
% Load style file, defaults  %
%%%%%%%%%%

\input{../../lec_style.tex}
% You cannot use numbers when defining variables.  Hence the use of letters, A, B, C, etc.

% Personal Info
\newcommand{\FirstName}{Mine}
\newcommand{\LastName}{\c{C}etinkaya-Rundel}
\newcommand{\OfficeHours}{MTWR 3-4pm.}
\newcommand{\OfficeHoursLocation}{Old Chem 213}

% Electronic Info
\newcommand{\PersonalSite}{http://stat.duke.edu/~mc301}
\newcommand{\CourseSite}{http://bitly.com/sta101sp15}
\newcommand{\Email}{mine@stat.duke.edu}

% TAs
\newcommand{\TAA}{Anthony Weishampel}
\newcommand{\TAB}{Fiamma Li}
\newcommand{\TAC}{Jialiang Mao}
\newcommand{\TAD}{Phillip Lee}

% Exam Dates
\newcommand{\ExamADate}{Wed, Feb 18}
\newcommand{\ExamBDate}{Wed, Mar 25}
\newcommand{\FinalDate}{Sat, May 2 (2-5pm)}

% Due Dates
\newcommand{\ClickerRegistrationDD}{Mon, Jan 26}
\newcommand{\GettingToKnowYouDD}{Friday, Jan 9, 11:59pm}
\newcommand{\ProblemSetADD}{Wed., 1/15}


% ALT ALT
% % You cannot use numbers when defining variables.  Hence the use of letters, A, B, C, etc.

% Personal Info
\renewcommand{\FirstName}{Jesse}
\renewcommand{\LastName}{Windle}
\renewcommand{\OfficeHours}{Tue, Thu 3:00pm-4:30pm}
\renewcommand{\OfficeHoursLocation}{Old Chem 211D}

% Electronic Info
\renewcommand{\PersonalSite}{http://stat.duke.edu/~jbw44/}
\renewcommand{\CourseSite}{http://bitly.com/windle2}
\renewcommand{\Email}{jbw44@stat.duke.edu}

% TAs
\renewcommand{\TAA}{David Clancy}
\renewcommand{\TAB}{Xinyi (Chris) Li}
\renewcommand{\TAC}{Tori Hall}
\renewcommand{\TAD}{Radhika Anand}

% Exam Dates
\renewcommand{\ExamADate}{Thu, Feb 19}
\renewcommand{\ExamBDate}{Thu, Mar 26}
\renewcommand{\FinalDate}{Mon, Apr 27 (9-Noon)}

% Due Dates
\renewcommand{\ClickerRegistrationDD}{Thu, Jan 15}
\renewcommand{\GettingToKnowYouDD}{Friday, Jan 9, 11:59pm}
\renewcommand{\ProblemSetADD}{Thu., 1/16}

%%%%%%%%%%%
% Cover slide info    %
%%%%%%%%%%%

\title{Unit 3: Foundations for inference}
\subtitle{2. Confidence intervals and hypothesis tests}
\author{Sta 101 - Spring 2015}
\date{February 11, 2015}
\institute{Duke University, Department of Statistical Science}

%%%%%%%%%%%
% Begin document   %
%%%%%%%%%%%

\begin{document}

%%%%%%%%%%%%%%%%%%%%%%%%%%%%%%%%%%%%

% Title Page

\begin{frame}[plain]

\titlepage
\vfill
{\scriptsize \webLink{\PersonalSite}{Dr. \LastName{}} \hfill Slides posted at  \webLink{\CourseSite}{\CourseSite}}
\addtocounter{framenumber}{-1} 

\end{frame}

%%%%%%%%%%%%%%%%%%%%%%%%%%%%%%%%%%%%

\section{Housekeeping}

%%%%%%%%%%%%%%%%%%%%%%%%%%%%%%%%%%%%

\begin{frame}
\frametitle{Announcements}

\begin{itemize}

\item No new PS this week, but review questions provided on new material that will be on the exam next week

\item Sample midterm + review questions posted

\item PA3 will be posted on Monday after class and will be due that evening, so budget your time accordingly for that short turnaround!

\end{itemize}

\end{frame}

%%%%%%%%%%%%%%%%%%%%%%%%%%%%%%%%%%%%

\section{Main ideas}

%%%%%%%%%%%%%%%%%%%%%%%%%%%%%%%%%%%%

\subsection{Statistical inference methods based on the CLT depend on the same conditions as the CLT}
\label{mi1}

%%%%%%%%%%%%%%%%%%%%%%%%%%%%%%%%%%%%


\subsection{Use confidence intervals to estimate population parameters}
\label{mi2}

%%%%%%%%%%%%%%%%%%%%%%%%%%%%%%%%%%%%

\begin{frame}

\vfill

\app{3.1 Confidence interval for a single mean}{See course website for details.}

\vfill

\end{frame}

%%%%%%%%%%%%%%%%%%%%%%%%%%%%%%%%%%%%

\subsection{Critical value depends on the confidence level}
\label{mi3}

%%%%%%%%%%%%%%%%%%%%%%%%%%%%%%%%%%%%

\begin{frame}

\clicker{What is the critical value ($Z^\star$) for a confidence interval at the 91\% confidence level?}

\twocol{0.4}{0.6}{
\begin{enumerate}[(a)]
\item $Z^\star = 1.34$
\item $Z^\star = 1.65$
\item \solnMult{$Z^\star = 1.70$}
\item $Z^\star = 1.96$
\item $Z^\star = 2.33$
\end{enumerate}
}
{
\pause
\soln{
\includegraphics[width=\textwidth]{figures/conf_level/conf_level}
}
}

\end{frame}

%%%%%%%%%%%%%%%%%%%%%%%%%%%%%%%%%%%%

\begin{frame}
\frametitle{Common misconceptions about confidence intervals}

\begin{enumerate}

\item \textcolor{gray}{The confidence level of a confidence interval is the probability that a given interval contains the true population parameter.} \\
\textit{This is incorrect, CIs are part of the frequentist paradigm and as such the population parameter is fixed but unknown. Consequently, the probability any given CI contains the true value must be 0 or 1 (it does or does not).} \\
$\:$ \\

\pause

\item  \textcolor{gray}{A narrower confidence interval is always better.}\\
\textit{This is incorrect since the width is a function of both the confidence level and the standard error.} \\
$\:$ \\

\pause

\item   \textcolor{gray}{A wider interval means less confidence.} \\
\textit{This is incorrect since it is possible to make very precise statements with very little confidence.} \\

\end{enumerate}

\end{frame}

%%%%%%%%%%%%%%%%%%%%%%%%%%%%%%%%%%%%

\subsection{Use hypothesis tests to make decisions about population parameters}
\label{mi4}

%%%%%%%%%%%%%%%%%%%%%%%%%%%%%%%%%%%%

\begin{frame}

\vfill

\app{3.2 Hypothesis testing for a single mean}{See course website for details.}

\vfill

\end{frame}

%%%%%%%%%%%%%%%%%%%%%%%%%%%%%%%%%%%%

\begin{frame}

\clicker{Which of the following is the correct interpretation of the p-value?}

\begin{enumerate}[(a)]
\item The probability that average GPA of Duke students has changed since 2001.
\item The probability that average GPA of Duke students has not changed since 2001.
\item The probability that average GPA of Duke students has not changed since 2001, if in fact a random sample of 63 Duke students this year have an average GPA of 3.58 or higher.
\item The probability that a random sample of 63 Duke students have an average GPA of 3.58 or higher, if in fact the average GPA has not changed since 2001.
\item \solnMult{The probability that a random sample of 63 Duke students have an average GPA of 3.58 or higher or 3.16 or lower, if in fact the average GPA has not changed since 2001.}
\end{enumerate}

\end{frame}

%%%%%%%%%%%%%%%%%%%%%%%%%%%%%%%%%%%%

\begin{frame}
\frametitle{Common misconceptions about hypothesis testing}

\begin{enumerate}

\item \textcolor{gray}{P-value is the probability that the null hypothesis is true} \\
\textit{This is incorrect, p-value is the probability of the observed or more extreme outcome if in fact the null hypothesis is correct. It is the conditional probability of the observed data (or something more extreme), conditioned on the null hypothesis being correct.} \\
$\:$ \\

\pause

\item  \textcolor{gray}{A high p-value confirms the null hypothesis.}\\
\textit{This is incorrect, a high p-value means the data do not provide convincing evidence for the alternative hypothesis and hence that the null hypothesis can't be rejected.} \\
$\:$ \\

\pause

\item   \textcolor{gray}{A low p-value confirms the alternative hypothesis.} \\
\textit{This is incorrect, a low p-value means the data provide convincing evidence for the alternative hypothesis, but not necessarily that it is confirmed.} \\

\end{enumerate}

\end{frame}

%%%%%%%%%%%%%%%%%%%%%%%%%%%%%%%%%%%%

\begin{frame}
\frametitle{Recap: Hypothesis testing framework}

\begin{enumerate}

\item Set the hypotheses.

\item Check assumptions and conditions.

\item Calculate a \hl{test statistic} and a p-value.

\item Make a decision, and interpret it in context of the research question.

\end{enumerate}

\end{frame}

%%%%%%%%%%%%%%%%%%%%%%%%%%%%%%%%%%%

\begin{frame}
\frametitle{Recap: Hypothesis testing for a population mean}

\begin{enumerate}

\item Set the hypotheses
\begin{itemize}
\item $H_0: \mu = null~value$
\item $H_A: \mu <$ or $>$ or $\ne null~value$
\end{itemize}

\item Check assumptions and conditions
\begin{itemize}
\item Independence: random sample/assignment, 10\% condition when sampling without replacement
\item Normality: nearly normal population or $n \ge 30$, no extreme skew
\end{itemize}

\item Calculate a \hl{test statistic} and a p-value (draw a picture!)
\[ Z = \frac{\bar{x} - \mu}{SE},~where~SE = \frac{s}{\sqrt{n}} \]

\item Make a decision, and interpret it in context of the research question
\begin{itemize}
\item If p-value $< \alpha$, reject $H_0$, data provide evidence for $H_A$
\item If p-value $> \alpha$, do not reject $H_0$, data do not provide evidence for $H_A$
\end{itemize}

\end{enumerate}

\end{frame}

%%%%%%%%%%%%%%%%%%%%%%%%%%%%%%%%%%%%

\section{Summary}

%%%%%%%%%%%%%%%%%%%%%%%%%%%%%%%%%%%%

\begin{frame}
\frametitle{Summary of main ideas}

\vfill

\begin{enumerate}

\item \nameref{mi1}

\item \nameref{mi2}

\item \nameref{mi3}

\item \nameref{mi4}

\end{enumerate}

\vfill

\end{frame}

%%%%%%%%%%%%%%%%%%%%%%%%%%%%%%%%%%%

\end{document}