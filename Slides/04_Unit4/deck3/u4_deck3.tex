% -*- TeX-engine: xetex; eval: (auto-fill-mode 0); eval: (visual-line-mode 1); -*-
% Compile with XeLaTeX

%%%%%%%%%%%%%%%%%%%%%%%
% To do before class
%%%%%%%%%%%%%%%%%%%%%%%

% Send the Logistics/Week0Annoucnement (the night before).
% Send an email reminding students to bring a charged computer (the night before).

%%%%%%%%%%%%%%%%%%%%%%%
% Option 1: Slides: (comment for handouts)   %
%%%%%%%%%%%%%%%%%%%%%%%

\documentclass[slidestop,compress,mathserif,12pt,t,professionalfonts,xcolor=table]{beamer}

% solution stuff
\newcommand{\solnMult}[1]{
\only<1>{#1}
\only<2->{\red{\textbf{#1}}}
}
\newcommand{\soln}[1]{\textit{#1}}

\newcommand{\solnMultOn}[3]{
\only<#1>{#3}
\only<{#2}->{\red{\textbf{#3}}}
}

%%%%%%%%%%%%%%%%%%%%%%%%%%%%%%%
% Option 2: Handouts, without solutions (post before class)    %
%%%%%%%%%%%%%%%%%%%%%%%%%%%%%%%

% \documentclass[11pt,containsverbatim,handout,xcolor=xelatex,dvipsnames,table]{beamer}

% % handout layout
% \usepackage{pgfpages}
% \pgfpagesuselayout{4 on 1}[letterpaper,landscape,border shrink=5mm]

% % solution stuff
% \newcommand{\solnMult}[1]{#1}
% \newcommand{\soln}[1]{}
% \newcommand{\solnMultOn}[3]{#3}

% % % This breaks things for me for some reason.
% % tell pgfpages how to set page sizes in XeLaTeX
% %\renewcommand\pgfsetupphysicalpagesizes{%
% %   \pdfpagewidth\pgfphysicalwidth\pdfpageheight\pgfphysicalheight%
% %}

%%%%%%%%%%%%%%%%%%%%%%%%%%%%%%%%%%%%
% Option 3: Handouts, with solutions (may post after class if need be)    %
%%%%%%%%%%%%%%%%%%%%%%%%%%%%%%%%%%%%

% \documentclass[11pt,containsverbatim,handout,xcolor=xelatex,dvipsnames,table]{beamer}

% % handout layout
% \usepackage{pgfpages}
% \pgfpagesuselayout{4 on 1}[letterpaper,landscape,border shrink=5mm]

% % solution stuff
% \newcommand{\solnMult}[1]{\red{\textbf{#1}}}
% \newcommand{\soln}[1]{\textit{#1}}

% % % This breaks things for me for some reason.
% % % tell pgfpages how to set page sizes in XeLaTeX
% % \renewcommand\pgfsetupphysicalpagesizes{%
% %    \pdfpagewidth\pgfphysicalwidth\pdfpageheight\pgfphysicalheight%
% % }

%%%%%%%%%%%%%%%%%%%%%%%%%%%%%%%
% Option 4: Notes Only
%%%%%%%%%%%%%%%%%%%%%%%%%%%%%%%

% % See http://tex.stackexchange.com/questions/114219/add-notes-to-latex-beamer
% \documentclass[10pt,containsverbatim,xcolor=xelatex,dvipsnames,table,notes=only]{beamer}

% % handout layout
% % \usepackage{pgfpages}
% % \pgfpagesuselayout{1 on 1}[letterpaper, landscape, border shrink=5mm]

% % solution stuff
% \newcommand{\solnMult}[1]{#1}
% \newcommand{\soln}[1]{}

% % % Having a problem with this.
% % tell pgfpages how to set page sizes in XeLaTeX
% % \renewcommand\pgfsetupphysicalpagesizes{%
% %   \pdfpagewidth\pgfphysicalwidth\pdfpageheight\pgfphysicalheight%
% %}

%%%%%%%%%%
% Load style file, defaults  %
%%%%%%%%%%

\input{../../lec_style.tex}
% You cannot use numbers when defining variables.  Hence the use of letters, A, B, C, etc.

% Personal Info
\newcommand{\FirstName}{Mine}
\newcommand{\LastName}{\c{C}etinkaya-Rundel}
\newcommand{\OfficeHours}{MTWR 3-4pm.}
\newcommand{\OfficeHoursLocation}{Old Chem 213}

% Electronic Info
\newcommand{\PersonalSite}{http://stat.duke.edu/~mc301}
\newcommand{\CourseSite}{http://bitly.com/sta101sp15}
\newcommand{\Email}{mine@stat.duke.edu}

% TAs
\newcommand{\TAA}{Anthony Weishampel}
\newcommand{\TAB}{Fiamma Li}
\newcommand{\TAC}{Jialiang Mao}
\newcommand{\TAD}{Phillip Lee}

% Exam Dates
\newcommand{\ExamADate}{Wed, Feb 18}
\newcommand{\ExamBDate}{Wed, Mar 25}
\newcommand{\FinalDate}{Sat, May 2 (2-5pm)}

% Due Dates
\newcommand{\ClickerRegistrationDD}{Mon, Jan 26}
\newcommand{\GettingToKnowYouDD}{Friday, Jan 9, 11:59pm}
\newcommand{\ProblemSetADD}{Wed., 1/15}


% ALT ALT
% You cannot use numbers when defining variables.  Hence the use of letters, A, B, C, etc.

% Personal Info
\renewcommand{\FirstName}{Jesse}
\renewcommand{\LastName}{Windle}
\renewcommand{\OfficeHours}{Tue, Thu 3:00pm-4:30pm}
\renewcommand{\OfficeHoursLocation}{Old Chem 211D}

% Electronic Info
\renewcommand{\PersonalSite}{http://stat.duke.edu/~jbw44/}
\renewcommand{\CourseSite}{http://bitly.com/windle2}
\renewcommand{\Email}{jbw44@stat.duke.edu}

% TAs
\renewcommand{\TAA}{David Clancy}
\renewcommand{\TAB}{Xinyi (Chris) Li}
\renewcommand{\TAC}{Tori Hall}
\renewcommand{\TAD}{Radhika Anand}

% Exam Dates
\renewcommand{\ExamADate}{Thu, Feb 19}
\renewcommand{\ExamBDate}{Thu, Mar 26}
\renewcommand{\FinalDate}{Mon, Apr 27 (9-Noon)}

% Due Dates
\renewcommand{\ClickerRegistrationDD}{Thu, Jan 15}
\renewcommand{\GettingToKnowYouDD}{Friday, Jan 9, 11:59pm}
\renewcommand{\ProblemSetADD}{Thu., 1/16}

%%%%%%%%%%%
% Cover slide info    %
%%%%%%%%%%%

\title{Unit 4: Inference for numerical data}
\subtitle{3. ANOVA}
\author{Sta 101 - Spring 2015}
\date{March 2, 2015}
\institute{Duke University, Department of Statistical Science}

%%%%%%%%%%%
% Begin document   %
%%%%%%%%%%%

\newcommand{\mainideaA}{It is difficult to simultaneously compare many groups.}
\newcommand{\mainideaB}{ANOVA is useful for testing if there is \emph{some} difference
  between the means of many different groups.}
\newcommand{\mainideaC}{The test is based on comparing between group to within group variation.}

\begin{document}

%%%%%%%%%%%%%%%%%%%%%%%%%%%%%%%%%%%%

% Title Page

\begin{frame}[plain]

\titlepage
\vfill
{\scriptsize \webLink{\PersonalSite}{Dr. \LastName{}} \hfill Slides posted at  \webLink{\CourseSite}{\CourseSite}}
\addtocounter{framenumber}{-1} 

\end{frame}

%%%%%%%%%%%%%%%%%%%%%%%%%%%%%%%%%%%%

\section{Housekeeping}

%%%%%%%%%%%%%%%%%%%%%%%%%%%%%%%%%%%%

\begin{frame}
\frametitle{Announcements}

\begin{itemize}

\item 

\end{itemize}

\end{frame}

%%%%%%%%%%%%%%%%%%%%%%%%%%%%%%%%%%%%

\section{Main ideas}

%%%%%%%%%%%%%%%%%%%%%%%%%%%%%%%%%%%%

\subsection{\mainideaA}
\label{mi1}

%%%%%%%%%%%%%%%%%%%%%%%%%%%%%%%%%%%%

\begin{frame}
  \frametitle{1. \mainideaA}

  \begin{center}
  {\Large NEWS FLASH!}

  \hspace{12pt}

  Jelly beans rumored to affect acne!!!
  \end{center}

  \pause

  \disc{
    How would you check this rumor? Imagine that doctors can assign an "acne
    score" to patients on a 0-100 scale.

    \begin{itemize}
      \item What would your research question be?
      \item How would you conduct your study?
      \item What statistical test would you use?
      \end{itemize}
    }

    \hfill \\

    \pause
   \soln{
     Use an independent samples t-test:
     \[
     H_0: \mu_{\textmd{jelly beans}} = \mu_{\textmd{placebo}}
     \]
     \[
     H_A: \mu_{\textmd{jelly beans}} \neq \mu_{\textmd{placebo}}
     \]
   }

\end{frame}

%%%%%%%%%%%%%%%%%%%%%%%%%%%%%%%%%%%%

\begin{frame}
  \frametitle{1. \mainideaA}

  \vfill
  
  \begin{center}
  \url{http://imgs.xkcd.com/comics/significant.png}
  \end{center}

\end{frame}

%%%%%%%%%%%%%%%%%%%%%%%%%%%%%%%%%%%%

\begin{frame}
  \frametitle{1. \mainideaA}

  \textbf{Suppose} $\alpha = 0.05$.

  \disc{What is the probability of correctly failing to reject
    \[
    H_0: \mu_{\textmd{purple}} = \mu_{\textmd{placebo}} \; ?
    \]
    }

    \pause

    \hfill \\

    \clicker{If all the tests are independent and if no color of Jelly bean has any link to acne, what is the
      probability of making at least one type I error in the 20 trials?}
      \begin{enumerate}[(a)]
      \item 5\%
      \item 36\%
      \item \solnMultOn{2}{3}{64\%}
      \item 95\%
      \end{enumerate}


\end{frame}

%%%%%%%%%%%%%%%%%%%%%%%%%%%%%%%%%%%%

\subsection{\mainideaB}
\label{mi2}

%%%%%%%%%%%%%%%%%%%%%%%%%%%%%%%%%%%%

\begin{frame}
  \frametitle{2. \mainideaB}

Null hypothesis for "F-Test" (the test associated with ANOVA):
\[
H_0: \mu_{\textmd{placebo}} = \mu_{\textmd{purple}} = \mu_{\textmd{brown}} =
\ldots = \mu_{\textmd{peach}} = \mu_{\textmd{orange}} .
\]

\pause

\clicker{Which of the following is a correct version of the alternative
  hypothesis?}

\begin{enumerate}[(a)]
\item For any two groups, including the placebo group, no two group means are the same.
\item For any two groups, not including the placebo group, no two group means are the
  same.
\item Amongst the jelly bean groups, there are at least two groups that have different group means.
\item \solnMultOn{2}{3}{Amongst all groups, there are at least two groups that have different group means.}
\end{enumerate}

\end{frame}

%%%%%%%%%%%%%%%%%%%%%%%%%%%%%%%%%%%%

\begin{frame}
  \frametitle{2. \mainideaB}

The practical implication of this alternative is: ``At least one color of
jelly bean is linked to acne.''

\pause

\hfill \\

At least two of the group means are not the same: \pause
\begin{enumerate}
\item $\mu_{\textmd{placebo}} \neq \mu_{\textmd{color}}$ for some color of jelly
  bean, or \pause
\item $\mu_{\textmd{A}} \neq \mu_{\textmd{B}}$ for two colors, A and B. \pause

\vspace{6pt}

Then 

\vspace{3pt}

\begin{enumerate}
\item $\mu_{\textmd{A}} \neq \mu_{\textmd{placebo}}$, or \pause

\item $\mu_{\textmd{A}} = \mu_{\textmd{placebo}}$.  Thus, $\mu_{\textmd{B}} \neq
  \mu_{\textmd{A}} = \mu_{\textmd{placebo}}$.
\end{enumerate}

\end{enumerate}

\end{frame}

%%%%%%%%%%%%%%%%%%%%%%%%%%%%%%%%%%%%

\subsection{\mainideaC}
\label{mi3}

%%%%%%%%%%%%%%%%%%%%%%%%%%%%%%%%%%%%

\begin{frame}
  \frametitle{3. \mainideaC}

\centering
\(
\sum {\color{blue} \mid}^2 \Big/ \sum {\color{green} \mid}^2
\)

  \includegraphics[scale=0.6]{figures/anova-middle-ground-jelly-bean.png}

\end{frame}

%-------------------------------------------------------------------------------
\begin{frame}
\frametitle{Relatively large WITHIN group variation: little apparent difference}

\centering
\(
\sum {\color{blue} \mid}^2 \Big/ \sum {\color{green} \mid}^2
\)

    \includegraphics[scale=0.6]{Figures/anova-lots-of-within-group-jelly-bean.png}

\end{frame}

%-------------------------------------------------------------------------------
\begin{frame}
\frametitle{Relatively large BETWEEN group variation: there may be a difference}

\centering
\(
\sum {\color{blue} \mid}^2 \Big/ \sum {\color{green} \mid}^2
\)
    
    \includegraphics[scale=0.6]{Figures/anova-lots-of-between-group-jelly-bean.png}

\end{frame}

%%%%%%%%%%%%%%%%%%%%%%%%%%%%%%

\begin{frame}
  \frametitle{3. The F-test is based on comparing between group to within group variation}

For historical reasons, we use a modification of this ratio called the $F$-statistic:
\[
F = \frac{\sum {\color{blue} \mid}^2 \; \; / \; \; (j-1)}{\sum {\color{green} \mid}^2 \; \; / \; \; (n-j)}
\hspace{12pt} = \hspace{12pt} \frac{MSG}{MSE}.
\]

$j$: \# of groups; $n$: \# of obs.

\pause

\begin{center}
\small
\begin{tabular}{lllrrr}
  \hline
  & Df  & Sum Sq & Mean Sq & F value & Pr($>$F) \\ 
  \hline
  Between & j-1 & $\sum {\color{blue}\mid}^2$  & MSG & $F_{obs}$ & $p_{obs}$ \\ 
  Within  & n-j & $\sum {\color{green}\mid}^2$ & MSE &		 &  \\ 
   \hline
Total			& n-1 & $\sum ({\color{blue}\mid} + {\color{green}\mid})^2$    &                &                &
\end{tabular}
\end{center}

\pause

\[
p_{obs} = p(W > F_{obs} \mid H_0) \pause = p(W > F_{obs} \mid W \sim \textmd{F-dist}_{j-1, n-j})
\]

\centering
\url{bitly.com/dist_calc}

\end{frame}

%%%%%%%%%%%%%%%%%%%%%%%%%%%%%%

\begin{frame}

\centering
\includegraphics[scale=0.6,angle=270]{figures/cartoon.png}

\end{frame}

%%%%%%%%%%%%%%%%%%%%%%%%%%%%%%

\begin{frame}

Say $\alpha = 0.05$.

\clicker{What is the most accurate statement of the results?}
\begin{enumerate}[(a)]
\item At least one color of jelly bean is linked to acne.
\item At least one color of jelly bean is not linked to acne.
\item \solnMultOn{1}{2}{There is little evidence that any color of jelly bean is linked to acne.}
\item Jelly beans definitely do not cause acne.
\end{enumerate}

\pause \pause

\clicker{For the $F$-test, what is the probability of incorrectly rejecting the null?}
\begin{enumerate}[(a)]
\item \solnMultOn{3}{4}{5\%}
\item 36\%
\item 64\%
\item 95\%
\end{enumerate}

\end{frame}

%%%%%%%%%%%%%%%%%%%%%%%%%%%%%%

\begin{frame}

\vfill

\app{4.4 ANOVA - Pt 1}{See the course webpage for details.}

\vfill

\end{frame}

%%%%%%%%%%%%%%%%%%%%%%%%%%%%%%%%%%

\section{Summary}

%%%%%%%%%%%%%%%%%%%%%%%%%%%%%%%%%%%%

\begin{frame}
\frametitle{Summary of main ideas}

\vfill

\begin{enumerate}

\item \nameref{mi1}

\item \nameref{mi2}

\item \nameref{mi3}

\end{enumerate}

\vfill

\end{frame}

%%%%%%%%%%%%%%%%%%%%%%%%%%%%%%%%%%%

\end{document}