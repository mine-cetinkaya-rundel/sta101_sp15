\documentclass[slidestop,compress,mathserif]{beamer}

%%% To get handouts:
%\documentclass[11pt,containsverbatim,handout]{beamer}
%% include when making handouts
%\usepackage{pgfpages}
%\pgfpagesuselayout{4 on 1}[letterpaper,landscape,border shrink=5mm]

%%% To get rid of solutions on handouts:
\newcommand{\soln}[1]{\textit{#1}}				% For slides
%\newcommand{\soln}[1]{ }	% For handouts

\newcommand{\solnGr}[1]{#1}
%\newcommand{\solnGr}{ }

\input{../../lec_style.tex}

%%%%%%%%%%%%%%%%%%%%%

\title[U4 - L3: ANOVA]{Unit 4: Inference for Numerical Data \\ Lecture 3: Comparing many means via ANOVA}
\author{Statistics 101}
\date{October 21, 2014}
\institute{Mine \c{C}etinkaya-Rundel}

%%%%%%%%%%%%%%%%%%%%%

\begin{document}

%%%%%%%%%%%%%%%%%%%%%

% Title Page

\begin{frame}[plain]

\titlepage

\end{frame}

%%%%%%%%%%%%%%%%%%%%%%%%%%%%%%%%%%%%

\section{Housekeeping}

%%%%%%%%%%%%%%%%%%%%%%%%%%%%%%%%%%%%

\begin{frame}
\frametitle{Announcements}

\begin{itemize}

\item Extra credit on MT

\item Project proposal feedback

\item PA 4 due Friday at 5pm (extended)

\end{itemize}

\end{frame}

%%%%%%%%%%%%%%%%%%%%%%%%%%%%%%%%%%%%

\section{Main ideas}

%%%%%%%%%%%%%%%%%%%%%%%%%%%%%%%%%%%

\subsection{(1) Roadmap for inference for numerical data}

%%%%%%%%%%%%%%%%%%%%%%%%%%%%%%%%%%%

\begin{frame}
\frametitle{(1) Roadmap for inference for numerical data}

\begin{center}
\includegraphics[width=\textwidth]{figures/num_data_inf/num_data_inf}
\end{center}

\end{frame}

%%%%%%%%%%%%%%%%%%%%%%%%%%%%%%%%%%%

\subsection{(2) ANOVA for comparing many means}

%%%%%%%%%%%%%%%%%%%%%%%%%%%%%%%%%%%

\begin{frame}
\frametitle{(2) ANOVA for comparing many means}

Different framework than before: instead of comparing point estimate to null value, compare variability within and between groups.

\pause

\begin{itemize}
\item[\hl{$Z$/$T$ test}] Compare means from \hl{two} groups to see whether they are so far apart that the observed difference cannot reasonably be attributed to sampling variability. The test statistic is a ratio.
\[ H_0: \mu_1 = \mu_2 \qquad
Z / T = \frac{(\bar{x}_1 - \bar{x}_2) - (\mu_1 - \mu_2)}{SE(\bar{x}_1 - \bar{x}_2)} \]

\pause

\item[\hl{ANOVA}] Compare the means from \hl{more than two} groups to see whether they are so far apart that the observed differences cannot all reasonably be attributed to sampling variability. The test statistic is a ratio.
\[ H_0: \mu_1 = \mu_2 = \cdots = \mu_k \qquad F = \frac{\text{variability bet. groups}}{\text{variability w/in groups}} \]
\end{itemize}

\end{frame}

%%%%%%%%%%%%%%%%%%%%%%%%%%%%%%%%%%%

\subsection{(3) Within group variability vs. between group variability} 

%%%%%%%%%%%%%%%%%%%%%%%%%%%%%%%%%%%

\begin{frame}
\frametitle{(3) Within group variability vs. between group variability}

\twocol{0.5}{0.5}
{
\begin{center}
Very little variability \green{within groups} \\
{\small (observations within a group alike)} \\
$\:$ \\
+ \\
\pause
lots of variability \orange{between groups} \\
{\small (groups very different \\
from each other)} \\
$\downarrow$ \\
\pause
likely significant ANOVA
\end{center}
}
{
\begin{center}
\pause
Lots of variability \green{within groups} \\
{\small (observations within a group \\
all over the place)} \\
+ \\
\pause
little variability \orange{between groups} \\
{\small (groups not so different \\
from each other)} \\
$\downarrow$ \\
\pause
likely \emph{not} significant ANOVA
\end{center}
}

\end{frame}

%%%%%%%%%%%%%%%%%%%%%%%%%%%%%%%%%%%%

\begin{frame}

\clicker{Order the plots with respect to \green{within groups} variability (low to high).}

\begin{center}
\includegraphics[width=\textwidth]{figures/within_between/within_between}
\end{center}

\begin{multicols}{2}
\begin{enumerate}[(a)]
\item Plot 1, Plot 2, Plot 3
\item Plot 1, Plot 3, Plot 1
\solnMult{Plot 2, Plot 1, Plot 3}
\item Plot 2, Plot 3, Plot 1
\item Plot 3, Plot 1, Plot 2
\item[]
\end{enumerate}
\end{multicols}

\end{frame}

%%%%%%%%%%%%%%%%%%%%%%%%%%%%%%%%%%%%

\begin{frame}

\clicker{Order the plots with respect to \orange{between groups} variability (low to high).}

\begin{center}
\includegraphics[width=\textwidth]{figures/within_between/within_between}
\end{center}

\begin{multicols}{2}
\begin{enumerate}[(a)]
\item Plot 1, Plot 2, Plot 3
\item Plot 1, Plot 3, Plot 1
\item Plot 2, Plot 1, Plot 3
\item Plot 2, Plot 3, Plot 1
\solnMult{Plot 3, Plot 1, Plot 2}
\item[]
\end{enumerate}
\end{multicols}

\end{frame}

%%%%%%%%%%%%%%%%%%%%%%%%%%%%%%%%%%%%

\subsection{(4) First ANOVA, then multiple comparisons}

%%%%%%%%%%%%%%%%%%%%%%%%%%%%%%%%%%%

\begin{frame}
\frametitle{(4) First ANOVA, then multiple comparisons}

\begin{itemize}

\item ANOVA: $H_A$:  At least one pair of means are different from each other.

\pause

\item If $H_0$ is rejected, and data provide evidence for $H_A$, we still don't know which groups have differing means.

\pause

\item Conduct tests that compare each pair of means to each other: \hl{multiple comparisons}.

\pause

\item Each one of these tests might potentially commit a Type 1 error, with a likelihood of $\alpha$.

\pause

\item Therefore, in order to avoid inflating the Type 1 error rate, run each of these tests at a modified (lower) significance level. 
\[ \alpha^\star = \frac{\alpha}{number~of~tests} \]

\end{itemize}

\end{frame}

%%%%%%%%%%%%%%%%%%%%%%%%%%%%%%%%%%%%

\begin{frame}
\frametitle{}

%add some clicker questions

\end{frame}

%%%%%%%%%%%%%%%%%%%%%%%%%%%%%%%%%%%%

\section{Application exercise}

%%%%%%%%%%%%%%%%%%%%%%%%%%%%%%%%%%%%

\begin{frame}
\frametitle{}

\vfill

\app{4.2 ANOVA - Part 1}{See course website for instructions}

\vfill

\end{frame}

%%%%%%%%%%%%%%%%%%%%%%%%%%%%%%%%%%%


\begin{frame}
\frametitle{}

\vfill

\app{4.3 ANOVA - Part 2}{See course website for instructions}

\vfill

\end{frame}

%%%%%%%%%%%%%%%%%%%%%%%%%%%%%%%%%%%

\section{Recap}

%%%%%%%%%%%%%%%%%%%%%%%%%%%%%%%%%%%

\begin{frame}
\frametitle{Conditions for ANOVA}

\begin{enumerate}

\item Independence: random sampling / assignment + 10\% rule

\item Approximately normal -- can be relaxed if $n$ is large

\item Constant variance -- can be relaxed if $n$ is consistent

\end{enumerate}

\end{frame}

%%%%%%%%%%%%%%%%%%%%%%%%%%%%%%%%%%%

\begin{frame}
\frametitle{Misc. notes on ANOVA}

\begin{itemize}

\item No such things as a confidence interval for ANOVA, since there is no parameter of interest to estimate.

\item You could use a Z test for multiple comparisons if the sample sizes are large enough, but now that you're familiar with the T test (which can be used for small and large samples) you might as well use that for all inference on single means or comparing two means.

\item The F distribution is always positive since the F statistic is the ratio of two variabilities (sums of squares). Also, we always shade above the observed F statistic since evidence for the alternative hypothesis means a higher ratio of the between to within variability.

\end{itemize}

\end{frame}

%%%%%%%%%%%%%%%%%%%%%%%%%%%%%%%%%%%

\end{document}