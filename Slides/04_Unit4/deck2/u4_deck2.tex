% -*- TeX-engine: xetex; eval: (auto-fill-mode 0); eval: (visual-line-mode 1); *-
% Compile with XeLaTeX

%%%%%%%%%%%%%%%%%%%%%%%
% To do before class
%%%%%%%%%%%%%%%%%%%%%%%

% Send the Logistics/Week0Annoucnement (the night before).
% Send an email reminding students to bring a charged computer (the night before).

%%%%%%%%%%%%%%%%%%%%%%%
% Option 1: Slides: (comment for handouts)   %
%%%%%%%%%%%%%%%%%%%%%%%

\documentclass[slidestop,compress,mathserif,12pt,t,professionalfonts,xcolor=table]{beamer}

% solution stuff
\newcommand{\solnMult}[1]{
\only<1>{#1}
\only<2->{\red{\textbf{#1}}}
}
\newcommand{\soln}[1]{\textit{#1}}

%%%%%%%%%%%%%%%%%%%%%%%%%%%%%%%
% Option 2: Handouts, without solutions (post before class)    %
%%%%%%%%%%%%%%%%%%%%%%%%%%%%%%%

% \documentclass[11pt,containsverbatim,handout,xcolor=xelatex,dvipsnames,table]{beamer}

% % handout layout
% \usepackage{pgfpages}
% \pgfpagesuselayout{4 on 1}[letterpaper,landscape,border shrink=5mm]

% % solution stuff
% \newcommand{\solnMult}[1]{#1}
% \newcommand{\soln}[1]{}

% % % This breaks things for me for some reason.
% % tell pgfpages how to set page sizes in XeLaTeX
% %\renewcommand\pgfsetupphysicalpagesizes{%
% %   \pdfpagewidth\pgfphysicalwidth\pdfpageheight\pgfphysicalheight%
% %}

%%%%%%%%%%%%%%%%%%%%%%%%%%%%%%%%%%%%
% Option 3: Handouts, with solutions (may post after class if need be)    %
%%%%%%%%%%%%%%%%%%%%%%%%%%%%%%%%%%%%

% \documentclass[11pt,containsverbatim,handout,xcolor=xelatex,dvipsnames,table]{beamer}

% % handout layout
% \usepackage{pgfpages}
% \pgfpagesuselayout{4 on 1}[letterpaper,landscape,border shrink=5mm]

% % solution stuff
% \newcommand{\solnMult}[1]{\red{\textbf{#1}}}
% \newcommand{\soln}[1]{\textit{#1}}

% % % This breaks things for me for some reason.
% % % tell pgfpages how to set page sizes in XeLaTeX
% % \renewcommand\pgfsetupphysicalpagesizes{%
% %    \pdfpagewidth\pgfphysicalwidth\pdfpageheight\pgfphysicalheight%
% % }

%%%%%%%%%%%%%%%%%%%%%%%%%%%%%%%
% Option 4: Notes Only
%%%%%%%%%%%%%%%%%%%%%%%%%%%%%%%

% % See http://tex.stackexchange.com/questions/114219/add-notes-to-latex-beamer
% \documentclass[10pt,containsverbatim,xcolor=xelatex,dvipsnames,table,notes=only]{beamer}

% % handout layout
% % \usepackage{pgfpages}
% % \pgfpagesuselayout{1 on 1}[letterpaper, landscape, border shrink=5mm]

% % solution stuff
% \newcommand{\solnMult}[1]{#1}
% \newcommand{\soln}[1]{}

% % % Having a problem with this.
% % tell pgfpages how to set page sizes in XeLaTeX
% % \renewcommand\pgfsetupphysicalpagesizes{%
% %   \pdfpagewidth\pgfphysicalwidth\pdfpageheight\pgfphysicalheight%
% %}

%%%%%%%%%%
% Load style file, defaults  %
%%%%%%%%%%

\input{../../lec_style.tex}
% You cannot use numbers when defining variables.  Hence the use of letters, A, B, C, etc.

% Personal Info
\newcommand{\FirstName}{Mine}
\newcommand{\LastName}{\c{C}etinkaya-Rundel}
\newcommand{\OfficeHours}{MTWR 3-4pm.}
\newcommand{\OfficeHoursLocation}{Old Chem 213}

% Electronic Info
\newcommand{\PersonalSite}{http://stat.duke.edu/~mc301}
\newcommand{\CourseSite}{http://bitly.com/sta101sp15}
\newcommand{\Email}{mine@stat.duke.edu}

% TAs
\newcommand{\TAA}{Anthony Weishampel}
\newcommand{\TAB}{Fiamma Li}
\newcommand{\TAC}{Jialiang Mao}
\newcommand{\TAD}{Phillip Lee}

% Exam Dates
\newcommand{\ExamADate}{Wed, Feb 18}
\newcommand{\ExamBDate}{Wed, Mar 25}
\newcommand{\FinalDate}{Sat, May 2 (2-5pm)}

% Due Dates
\newcommand{\ClickerRegistrationDD}{Mon, Jan 26}
\newcommand{\GettingToKnowYouDD}{Friday, Jan 9, 11:59pm}
\newcommand{\ProblemSetADD}{Wed., 1/15}


% ALT ALT
% % You cannot use numbers when defining variables.  Hence the use of letters, A, B, C, etc.

% Personal Info
\renewcommand{\FirstName}{Jesse}
\renewcommand{\LastName}{Windle}
\renewcommand{\OfficeHours}{Tue, Thu 3:00pm-4:30pm}
\renewcommand{\OfficeHoursLocation}{Old Chem 211D}

% Electronic Info
\renewcommand{\PersonalSite}{http://stat.duke.edu/~jbw44/}
\renewcommand{\CourseSite}{http://bitly.com/windle2}
\renewcommand{\Email}{jbw44@stat.duke.edu}

% TAs
\renewcommand{\TAA}{David Clancy}
\renewcommand{\TAB}{Xinyi (Chris) Li}
\renewcommand{\TAC}{Tori Hall}
\renewcommand{\TAD}{Radhika Anand}

% Exam Dates
\renewcommand{\ExamADate}{Thu, Feb 19}
\renewcommand{\ExamBDate}{Thu, Mar 26}
\renewcommand{\FinalDate}{Mon, Apr 27 (9-Noon)}

% Due Dates
\renewcommand{\ClickerRegistrationDD}{Thu, Jan 15}
\renewcommand{\GettingToKnowYouDD}{Friday, Jan 9, 11:59pm}
\renewcommand{\ProblemSetADD}{Thu., 1/16}

%%%%%%%%%%%
% Cover slide info    %
%%%%%%%%%%%

\title{Unit 4: Inference for numerical data}
\subtitle{2. Comparing means}
\author{Sta 101 - Spring 2015}
\date{February 25, 2015}
% ALT ALT
% \date{February 26, 2015}
\institute{Duke University, Department of Statistical Science}

%%%%%%%%%%%
% Begin document   %
%%%%%%%%%%%

\begin{document}

%%%%%%%%%%%%%%%%%%%%%%%%%%%%%%%%%%%%

% Title Page

\begin{frame}[plain]

\titlepage
\vfill
{\scriptsize \webLink{\PersonalSite}{Dr. \LastName{}} \hfill Slides posted at  \webLink{\CourseSite}{\CourseSite}}
\addtocounter{framenumber}{-1} 

\end{frame}

%%%%%%%%%%%%%%%%%%%%%%%%%%%%%%%%%%%%

\section{Housekeeping}

%%%%%%%%%%%%%%%%%%%%%%%%%%%%%%%%%%%%

\begin{frame}
\frametitle{Announcements}

\begin{itemize}

\item 
% ALT ALT 
% \item Individually or in teams, complete App Ex 4.2 and 4.3 by Tuesday, March 2.
%   App Ex 4.2 and 4.3 will not be turned in.

\end{itemize}

\end{frame}

%%%%%%%%%%%%%%%%%%%%%%%%%%%%%%%%%%%%

\section{Main ideas}

%%%%%%%%%%%%%%%%%%%%%%%%%%%%%%%%%%%%

\subsection{When comparing means of two groups, ask if paired or independent}
\label{mi1}

%%%%%%%%%%%%%%%%%%%%%%%%%%%%%%%%%%%%

\begin{frame}
\frametitle{1. When comparing means of two groups, ask if paired or independent}

\begin{itemize}

\item dependent (paired) groups (e.g. pre/post weights of subjects in a weight loss study, twin studies, etc.)
\[ SE_{\bar{x}_{diff}} = \frac{s_{diff}}{\sqrt{n_{diff}}} \]

\item independent groups (e.g. grades of students across two sections)
\[ SE_{\bar{x}_1 - \bar{x}_2} = \sqrt{ \frac{s_1^2}{n_1} + \frac{s_2^2}{n_2} } \]

\end{itemize}

%---Note---%
\note{

- Go through some examples of where you have two groups.
  - Twin studies for instance.

- Overall SE increases when looking at two different groups.

- Lemonade and cup.  Uncertainty in both.  Makes uncertainty in difference
greater.

- Comment on types of data.

}

\end{frame}

%%%%%%%%%%%%%%%%%%%%%%%%%%%%%%%%%%%%

\subsection{T corrects for uncertainty introduced by plugging in $s$ for $\sigma$}
\label{mi2}

%%%%%%%%%%%%%%%%%%%%%%%%%%%%%%%%%%%%

\begin{frame}
\frametitle{2. T corrects for uncertainty introduced by plugging in $s$ for $\sigma$}

\begin{itemize}

\item Essential when $n$ is small ($n < 30$) since $s$ is more likely to be not be a good estimate for $\sigma$ when $n$ is small than when $n$ is large

\pause

\item Could be used when $n$ is large as well

\pause

\item Also has a bell shape, but its tails are \hl{thicker} than the normal model's
\begin{itemize}
\item Observations are more likely to fall beyond two SDs from the mean than under the normal distribution.
\end{itemize}

\pause

\item Extra thick tails are helpful for mitigating the effect of a less reliable estimate for the standard error of the sampling distribution (since $n$ is small)

\end{itemize}

\begin{center}
\includegraphics[width=0.5\textwidth]{figures/tDistCompareToNormalDist/tDistCompareToNormalDist}
\end{center}

%---Note---%
\note{

- Corrects for certainty when plugging in $s$ for $\sigma$.

- Question from student: when would you ever have small $n$?

- Think about squishing down distribution.

- Go through shape, center, symmetry, etc.

}

\end{frame}

%%%%%%%%%%%%%%%%%%%%%%%%%%%%%%%%%%%%

\begin{frame}
\frametitle{T distribution}

\begin{itemize}

\item Always centered at zero, like the standard normal ($z$) distribution

\pause

\item Has a single parameter: \hl{degrees of freedom} (\mathhl{df})
\begin{itemize}
\item one sample: $df = n - 1$
\item two (independent) samples: $df = min(n_1 - 1, n_2 - 1)$
\end{itemize}

\end{itemize}

\begin{center}
\includegraphics[width=0.8\textwidth]{figures/tDistConvergeToNormalDist/tDistConvergeToNormalDist}
\end{center}

\pause

\disc{What happens to shape of the T distribution as $df$ increases?}

\soln{\pause Approaches normal.}

\end{frame}

%%%%%%%%%%%%%%%%%%%%%%%%%%%%%%%%%%%%

\begin{frame}

\clicker{Under the T distribution with 5 degrees freedom, \_\_\_\_\_\_\_\_\_\_ of the data are within one standard deviation of the mean?}

\begin{enumerate}[(a)]
\item 68\%
\item \solnMult{less than 68\%}
\item more than 68\%
\end{enumerate}

\end{frame}

%%%%%%%%%%%%%%%%%%%%%%%%%%%%%%%%%%%%

\begin{frame}
\frametitle{}

\vfill

\app{4.2 Comparing means, Pt 1}{See the course webpage for details.}

\vfill

\end{frame}

%%%%%%%%%%%%%%%%%%%%%%%%%%%%%%%%%%

\begin{frame}
\frametitle{}

\vfill

\app{4.3 Comparing means, Pt 2}{See the course webpage for details.}

\vfill

\end{frame}

%%%%%%%%%%%%%%%%%%%%%%%%%%%%%%%%%%

\section{Summary}

%%%%%%%%%%%%%%%%%%%%%%%%%%%%%%%%%%%%

\begin{frame}
\frametitle{Summary of main ideas}

\vfill

\begin{enumerate}

\item \nameref{mi1}

\item \nameref{mi2}

\end{enumerate}

\vfill

\end{frame}

%%%%%%%%%%%%%%%%%%%%%%%%%%%%%%%%%%%

\end{document}