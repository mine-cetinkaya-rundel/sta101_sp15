% -*- TeX-engine: xetex; -*-
% Compile with XeLaTeX

%%%%%%%%%%%%%%%%%%%%%%%
% To do before class
%%%%%%%%%%%%%%%%%%%%%%%

% Send the Logistics/Week0Annoucnement (the night before).
% Send an email reminding students to bring a charged computer (the night before).

%%%%%%%%%%%%%%%%%%%%%%%
% Option 1: Slides: (comment for handouts)   %
%%%%%%%%%%%%%%%%%%%%%%%

\documentclass[slidestop,compress,mathserif,12pt,t,professionalfonts,xcolor=table]{beamer}

% solution stuff
\newcommand{\solnMult}[1]{
\only<1>{#1}
\only<2->{\red{\textbf{#1}}}
}
\newcommand{\soln}[1]{\textit{#1}}

%%%%%%%%%%%%%%%%%%%%%%%%%%%%%%%
% Option 2: Handouts, without solutions (post before class)    %
%%%%%%%%%%%%%%%%%%%%%%%%%%%%%%%

% \documentclass[11pt,containsverbatim,handout,xcolor=xelatex,dvipsnames,table]{beamer}

% % handout layout
% \usepackage{pgfpages}
% \pgfpagesuselayout{4 on 1}[letterpaper,landscape,border shrink=5mm]

% % solution stuff
% \newcommand{\solnMult}[1]{#1}
% \newcommand{\soln}[1]{}

% % % This breaks things for me for some reason.
% % tell pgfpages how to set page sizes in XeLaTeX
% %\renewcommand\pgfsetupphysicalpagesizes{%
% %   \pdfpagewidth\pgfphysicalwidth\pdfpageheight\pgfphysicalheight%
% %}

%%%%%%%%%%%%%%%%%%%%%%%%%%%%%%%%%%%%
% Option 3: Handouts, with solutions (may post after class if need be)    %
%%%%%%%%%%%%%%%%%%%%%%%%%%%%%%%%%%%%

% \documentclass[11pt,containsverbatim,handout,xcolor=xelatex,dvipsnames,table]{beamer}

% % handout layout
% \usepackage{pgfpages}
% \pgfpagesuselayout{4 on 1}[letterpaper,landscape,border shrink=5mm]

% % solution stuff
% \newcommand{\solnMult}[1]{\red{\textbf{#1}}}
% \newcommand{\soln}[1]{\textit{#1}}

% % % This breaks things for me for some reason.
% % % tell pgfpages how to set page sizes in XeLaTeX
% % \renewcommand\pgfsetupphysicalpagesizes{%
% %    \pdfpagewidth\pgfphysicalwidth\pdfpageheight\pgfphysicalheight%
% % }

%%%%%%%%%%%%%%%%%%%%%%%%%%%%%%%
% Option 4: Notes Only
%%%%%%%%%%%%%%%%%%%%%%%%%%%%%%%

% % See http://tex.stackexchange.com/questions/114219/add-notes-to-latex-beamer
% \documentclass[10pt,containsverbatim,xcolor=xelatex,dvipsnames,table,notes=only]{beamer}

% % handout layout
% % \usepackage{pgfpages}
% % \pgfpagesuselayout{1 on 1}[letterpaper, landscape, border shrink=5mm]

% % solution stuff
% \newcommand{\solnMult}[1]{#1}
% \newcommand{\soln}[1]{}

% % % Having a problem with this.
% % tell pgfpages how to set page sizes in XeLaTeX
% % \renewcommand\pgfsetupphysicalpagesizes{%
% %   \pdfpagewidth\pgfphysicalwidth\pdfpageheight\pgfphysicalheight%
% %}

%%%%%%%%%%
% Load style file, defaults  %
%%%%%%%%%%

\input{../../lec_style.tex}
% You cannot use numbers when defining variables.  Hence the use of letters, A, B, C, etc.

% Personal Info
\newcommand{\FirstName}{Mine}
\newcommand{\LastName}{\c{C}etinkaya-Rundel}
\newcommand{\OfficeHours}{MTWR 3-4pm.}
\newcommand{\OfficeHoursLocation}{Old Chem 213}

% Electronic Info
\newcommand{\PersonalSite}{http://stat.duke.edu/~mc301}
\newcommand{\CourseSite}{http://bitly.com/sta101sp15}
\newcommand{\Email}{mine@stat.duke.edu}

% TAs
\newcommand{\TAA}{Anthony Weishampel}
\newcommand{\TAB}{Fiamma Li}
\newcommand{\TAC}{Jialiang Mao}
\newcommand{\TAD}{Phillip Lee}

% Exam Dates
\newcommand{\ExamADate}{Wed, Feb 18}
\newcommand{\ExamBDate}{Wed, Mar 25}
\newcommand{\FinalDate}{Sat, May 2 (2-5pm)}

% Due Dates
\newcommand{\ClickerRegistrationDD}{Mon, Jan 26}
\newcommand{\GettingToKnowYouDD}{Friday, Jan 9, 11:59pm}
\newcommand{\ProblemSetADD}{Wed., 1/15}


% ALT ALT
% % You cannot use numbers when defining variables.  Hence the use of letters, A, B, C, etc.

% Personal Info
\renewcommand{\FirstName}{Jesse}
\renewcommand{\LastName}{Windle}
\renewcommand{\OfficeHours}{Tue, Thu 3:00pm-4:30pm}
\renewcommand{\OfficeHoursLocation}{Old Chem 211D}

% Electronic Info
\renewcommand{\PersonalSite}{http://stat.duke.edu/~jbw44/}
\renewcommand{\CourseSite}{http://bitly.com/windle2}
\renewcommand{\Email}{jbw44@stat.duke.edu}

% TAs
\renewcommand{\TAA}{David Clancy}
\renewcommand{\TAB}{Xinyi (Chris) Li}
\renewcommand{\TAC}{Tori Hall}
\renewcommand{\TAD}{Radhika Anand}

% Exam Dates
\renewcommand{\ExamADate}{Thu, Feb 19}
\renewcommand{\ExamBDate}{Thu, Mar 26}
\renewcommand{\FinalDate}{Mon, Apr 27 (9-Noon)}

% Due Dates
\renewcommand{\ClickerRegistrationDD}{Thu, Jan 15}
\renewcommand{\GettingToKnowYouDD}{Friday, Jan 9, 11:59pm}
\renewcommand{\ProblemSetADD}{Thu., 1/16}

%%%%%%%%%%%
% Cover slide info    %
%%%%%%%%%%%

\title{Unit 4: Inference for numerical data}
\subtitle{1. Bootstrap intervals}
\author{Sta 101 - Spring 2015}
\date{February 23, 2015}
% ALT LAT
% \date{February 24, 2015}
\institute{Duke University, Department of Statistical Science}

%%%%%%%%%%%
% Begin document   %
%%%%%%%%%%%

\begin{document}

%%%%%%%%%%%%%%%%%%%%%%%%%%%%%%%%%%%%

% Title Page

\begin{frame}[plain]

\titlepage
\vfill
{\scriptsize \webLink{\PersonalSite}{Dr. \LastName{}} \hfill Slides posted at  \webLink{\CourseSite}{\CourseSite}}
\addtocounter{framenumber}{-1} 

\end{frame}

%%%%%%%%%%%%%%%%%%%%%%%%%%%%%%%%%%%%

\section{Housekeeping}

%%%%%%%%%%%%%%%%%%%%%%%%%%%%%%%%%%%%

\begin{frame}
\frametitle{Announcements}

\begin{itemize}

\item MT corrections extra credit: Work \textbf{as a team} to write up a collective exam corrections document that discusses all questions missed by any member of the team. Your corrections should show full work and explain reasoning, even for the multiple choice questions. Due next Tuesday (March 3) in lab. \textbf{Extra credit:} +2 points on the exam.

\item No lab due tomorrow. Check lab authorship document for updated lab authorship assignments.

\item Project proposals due Friday, 11:59pm

% ALT ALT

% \item Exam 1:

% \begin{itemize}[<+->]
% \item Hand back Exam 1 in lab tomorrow (2/25).
% \item You must return Exam 1 in the following lab (3/4).
% \item You may make a copy for your own personal use.  You may not share that
%   copy with anyone.
% \item Regrade requests must be made \emph{in writing} and turned in with Exam 1.
% \item Regrade requests are only for problems that have been graded incorrectly.
% \item I reserve the right to regrade the entire exam.
% \end{itemize}

% \pause

% \item Project proposalas due Friday, 11:59pm.

% \begin{itemize}
% \item Schedule a time to meet right now.
% \end{itemize}

\end{itemize}

%---Note---%
\note{

Readiness assessment:

Go over: 1, 4

1: Think about each.

4: What does paired data mean?  (We have two observations of the same case.)

Comment on Project Proposal.  Finding the data is the most challenging part.

}

\end{frame}

%%%%%%%%%%%%%%%%%%%%%%%%%%%%%%%%%%%%

\section{Bootstrapping}

%%%%%%%%%%%%%%%%%%%%%%%%%%%%%%%%%%%

\begin{frame}
\frametitle{Rotten horrors}

\twocol{0.4}{0.6}
{
\includegraphics[width = 0.8\textwidth]{figures/movies/rotten_tomatoes} \\
is a movie aggregator, where the audience is also able to review and score the movies. We want to estimate the average audience score of horror movies on RottenTomatoes.com. We start with a random sample of 20 horror movies.
}
{
\includegraphics[width = \textwidth]{figures/movies/horror_data}
}

%---Note---%
\note{

- Who here is NOT familiar with rotten tomoatoes?

}

\end{frame}

%%%%%%%%%%%%%%%%%%%%%%%%%%%%%%%%%%%

\begin{frame}[fragile]
\frametitle{Data}

{\scriptsize
\begin{verbatim}
                                  title audience_score
 1                              Patrick             52
 2                           Demon Seed             43
 3                            Tormented             34
 4                        Under the Bed             12
 5                Phantasm IV: Oblivion             41
 6                  Fright Night Part 2             42
 7                House of 1000 Corpses             65
 8                          Creepshow 2             46
 9                         The Forsaken             44
10         All the Boys Love Mandy Lane             34
11 Jason Lives: Friday the 13th Part VI             57
12                       Vampire's Kiss             48
13              The Witches of Eastwick             60
14                      Yellowbrickroad             28
15                          Dying Breed             27
16                               Carrie             73
17             Whoever Slew Auntie Roo?             56
18                          The Mangler             23
19                               Primal             29
20          The Twilight Saga: New Moon             65
\end{verbatim}
}


%---Note---%
\note{

- Does anyone recognize any of these?

}

\end{frame}

%%%%%%%%%%%%%%%%%%%%%%%%%%%%%%%%%%%

\begin{frame}
\frametitle{First look}

\disc{{\small The histogram below shows the distribution of the audience scores of these movies (ranging from 0 to 100). The median score in the sample is 43.5. Can we apply CLT based methods we have learned so far to construct a confidence interval for the \underline{median} RottenTomatoes score of horror movies. Why or why not?}}

\begin{center}
\includegraphics[width = 0.8\textwidth]{figures/movies/horror_hist}
\end{center}

%---Note---%
\note{

- We do not know about confidence intervals for medians.

}

\end{frame}

%%%%%%%%%%%%%%%%%%%%%%%%%%%%%%%%%%%

\begin{frame}
\frametitle{Bootstrapping}

\begin{itemize}

\item An alternative approach to constructing confidence intervals is \hl{bootstrapping}. 

\pause

\item This term comes from the phrase ``pulling oneself up by one's bootstraps", which is a metaphor for accomplishing an impossible task without any outside help. 

\pause

\item In this case the \sout{im}possible task is estimating a population parameter, and we'll accomplish it using data from only the given sample.

\end{itemize}

\hfill \includegraphics[width = 0.25\textwidth]{figures/boot}

\end{frame}

%%%%%%%%%%%%%%%%%%%%%%%%%%%%%%%%%%%%%

\begin{frame}
\frametitle{Bootstrapping}

\begin{itemize}

\item Bootstrapping works as follows:
\pause
\begin{enumerate}[(1)]
\item take a bootstrap sample - a random sample taken with replacement from the original sample, of the same size as the original sample
\pause
\item calculate the bootstrap statistic - a statistic such as mean, median, proportion, etc. computed on the bootstrap samples
\pause
\item repeat steps (1) and (2) many times to create a bootstrap distribution - a distribution of bootstrap statistics
\end{enumerate}

\pause

\item The XX\% bootstrap confidence interval can be estimated by
\begin{itemize}
\pause
\item the cutoff values for the middle XX\% of the bootstrap distribution,
\item[]
\pause
\item[] OR
\item[]
\pause
\item $\bar{x}_{boot} \pm z^\star SE_{boot}$
\end{itemize}

\end{itemize}

\end{frame}

%%%%%%%%%%%%%%%%%%%%%%%%%%%%%%%%%%%%%

\begin{frame}[fragile]
\frametitle{Bootstrap sample 1}

{\small \textbf{(1) Take a bootstrap sample:}}
\pause
{\tiny
\begin{verbatim}
                                  title audience_score
 1                       Vampire's Kiss             48
 2                Phantasm IV: Oblivion             41
 3                House of 1000 Corpses             65
 4                          Dying Breed             27
 5             Whoever Slew Auntie Roo?             56
 6                         The Forsaken             44
 7          The Twilight Saga: New Moon             65
 8          The Twilight Saga: New Moon             65
 9             Whoever Slew Auntie Roo?             56
10          The Twilight Saga: New Moon             65
11                          The Mangler             23
12                          Dying Breed             27
13                          Creepshow 2             46
14                House of 1000 Corpses             65
15             Whoever Slew Auntie Roo?             56
16                            Tormented             34
17 Jason Lives: Friday the 13th Part VI             57
18                       Vampire's Kiss             48
19                               Primal             29
20              The Witches of Eastwick             60
\end{verbatim}
}

\pause

{\small \textbf{(2) Calculate the median of the bootstrap sample:}} \\
\pause
{\footnotesize
23, 27, 27, 29, 34, 41, 44, 46, 48, \red{48, 56}, 56, 56, 57, 60, 65, 65, 65, 65, 65 \\
median = (48 + 56) / 2 = 52 \\
}

\pause

{\small
\textbf{(3) Record this value}
}

%---Note---%
\note{

Sample WITH replacement.

}

\end{frame}

%%%%%%%%%%%%%%%%%%%%%%%%%%%%%%%%%%%%%

\begin{frame}[fragile]
\frametitle{Bootstrap sample 2}

{\small \textbf{(1) Take another bootstrap sample:}}
\pause
{\tiny
\begin{verbatim}
                                  title audience_score
 1                  Fright Night Part 2             42
 2                               Carrie             73
 3                         The Forsaken             44
 4                          The Mangler             23
 5                               Primal             29
 6                              Patrick             52
 7 Jason Lives: Friday the 13th Part VI             57
 8                          The Mangler             23
 9                       Vampire's Kiss             48
10         All the Boys Love Mandy Lane             34
11          The Twilight Saga: New Moon             65
12         All the Boys Love Mandy Lane             34
13                      Yellowbrickroad             28
14                       Vampire's Kiss             48
15                            Tormented             34
16                          The Mangler             23
17                Phantasm IV: Oblivion             41
18                              Patrick             52
19                House of 1000 Corpses             65
20          The Twilight Saga: New Moon             65
\end{verbatim}
}

\pause

{\small \textbf{(2) Calculate the median of the bootstrap sample:}} \\
\pause
{\footnotesize
23, 23, 23, 28, 29, 34, 34, 34, 41, \red{42, 44}, 48, 48, 52, 52, 57, 65, 65, 65, 73 \\
median = (42 + 44) / 2 = 43 \\
}

\pause

{\small
\textbf{(3) Record this value}
}

\end{frame}

%%%%%%%%%%%%%%%%%%%%%%%%%%%%%%%%%%%%%

\begin{frame}[fragile]
\frametitle{Bootstrap sample 3}

{\small \textbf{(1) Take another bootstrap sample:}}
\pause
{\tiny
\begin{verbatim}
                                  title audience_score
 1                            Tormented             34
 2              The Witches of Eastwick             60
 3              The Witches of Eastwick             60
 4              The Witches of Eastwick             60
 5                          The Mangler             23
 6              The Witches of Eastwick             60
 7                              Patrick             52
 8                Phantasm IV: Oblivion             41
 9                      Yellowbrickroad             28
10 Jason Lives: Friday the 13th Part VI             57
11                      Yellowbrickroad             28
12 Jason Lives: Friday the 13th Part VI             57
13                  Fright Night Part 2             42
14                               Primal             29
15                  Fright Night Part 2             42
16             Whoever Slew Auntie Roo?             56
17                  Fright Night Part 2             42
18                  Fright Night Part 2             42
19                        Under the Bed             12
20                Phantasm IV: Oblivion             41
\end{verbatim}
}

\pause

{\small \textbf{(2) Calculate the median of the bootstrap sample:}} \\
\pause
{\footnotesize
12, 23, 28, 28, 29, 34, 41, 41, 42, \red{42, 42}, 42, 52, 56, 57, 57, 60, 60, 60, 60 \\
median = (42 + 42) / 2 = 42 \\
}

\pause

{\small
\textbf{(3) Record this value}
}

\end{frame}

%%%%%%%%%%%%%%%%%%%%%%%%%%%%%%%%%%%%%

\begin{frame}
\frametitle{Many more bootstrap samples}

\vfill

... repeat

\vfill

\end{frame}

%%%%%%%%%%%%%%%%%%%%%%%%%%%%%%%%%%%%%

\begin{frame}
\frametitle{}

\clicker{The dot plot  below is the bootstrap distribution of medians constructed using 100 simulations. What does each dot on the dot plot represent?}

\begin{center}
\includegraphics[width = 0.6\textwidth]{figures/movies/horror_boot_med_dot}
\end{center}

\begin{enumerate}[(a)]
\item Score of a horror movie in the original sample
\item Score of a horror movie in the population
\item \solnMult{Median from one bootstrap sample from the original sample}
\item Median from one sample from the population
\end{enumerate}

%---Note---%
\note{

What does each dot on this dotplot represent.

}

\end{frame}

%%%%%%%%%%%%%%%%%%%%%%%%%%%%%%%%%%

\begin{frame}
\frametitle{}

\clicker{The dot plot  below shows the distribution of 100 bootstrap medians. Estimate the 90\% bootstrap confidence interval for the median RT score of horror movies using the percentile method.}

\only<1>{
\begin{center}
\includegraphics[width = 0.8\textwidth]{figures/movies/horror_boot_med_dot}
\end{center}
}

\soln{\only<2->{
\begin{center}
\includegraphics[width = 0.8\textwidth]{figures/movies/horror_boot_med_dot_soln}
\end{center}
}}

\begin{multicols}{2}
\begin{enumerate}[(a)]
\item (29, 58.5)
\item (34, 57)
\item \solnMult{(37.5, 52)}
\item (40, 49.5)
\end{enumerate}
\end{multicols}

%---Note---%
\note{

Percentile method.  Maybe ask, where are 90\% of the sample medians?

Talk to partners and retry.

Looking for cutoff of middle 90\% of distribution.

}

\end{frame}

%%%%%%%%%%%%%%%%%%%%%%%%%%%%%%%%%

\begin{frame}
\frametitle{Botstrap interval, standard error}

\disc{The dot plot  below shows the distribution of 100 bootstrap medians. The median of the original sample is 43.5 and the bootstrap standard error is 4.88. Estimate the 90\% bootstrap confidence interval for the median RT score of horror movies using the standard error method.}

\begin{center}
\includegraphics[width = 0.75\textwidth]{figures/movies/horror_boot_med_dot}
\end{center}

\pause

\soln{\[ 43.5 \pm (1.65 \times 4.88) = (35.45, 51.55) \] }

%---Note---%
\note{

You can also compute the standard error.  Think of this as just given one some
sense of unertainty in estimate.

}

\end{frame}

%%%%%%%%%%%%%%%%%%%%%%%%%%%%%%%%%

\begin{frame}
\frametitle{Bootstrap vs. sampling distributions}

\vfill

\app{4.1 Bootstrap intervals}{See the course webpage for details.}

\vfill

\end{frame}

%%%%%%%%%%%%%%%%%%%%%%%%%%%%%%%%%%

\section{Randomization testing for a single numerical variable}
% ALT ALT
% \section{Bootstrap testing for a single numerical variable}

%%%%%%%%%%%%%%%%%%%%%%%%%%%%%%%%%%%

\begin{frame}
\frametitle{Randomization testing for a mean}
% ALT ALT
% \frametitle{Bootstrap testing for a mean}

\begin{itemize}

\item This is very similar to bootstrapping, i.e. we randomly sample with replacement from the sample, but this time we shift the bootstrap distribution to be \underline{centered at the null value}. 

\pause

\item The p-value is then defined as the proportion of simulations that yield a sample statistic at least as favorable to the alternative hypothesis as the observed sample statistic.

\end{itemize}

\end{frame}

%%%%%%%%%%%%%%%%%%%%%%%%%%%%%%%%%%%

\begin{frame}
\frametitle{}

\disc{Do these data provide convincing evidence that the median audience score of horror movies is greater than 40? Remember that the median of the original sample was 43.5.}

\begin{center}
\includegraphics[width = 0.75\textwidth]{figures/movies/horror_boot_med_test_dot}
\end{center}

\pause

\twocol{0.3}{0.7}{
\begin{itemize}
\item[$H_0:$] $median = 40$
\item[$H_A:$] $median > 40$
\end{itemize}
}
{
\pause
p-value: proportion of simulations where the simulated bootstrap sample median is at least as extreme as the one observed (43.5). $\rightarrow$ 20 / 100 = 0.20
}

\end{frame}

%%%%%%%%%%%%%%%%%%%%%%%%%%%%%%%%%%%

\section{Bootstrapping for categorical data}

%%%%%%%%%%%%%%%%%%%%%%%%%%%%%%%%%%%

\begin{frame}
\frametitle{}

\disc{Describe how you would construct a bootstrap interval for a proportion.}

\end{frame}

%%%%%%%%%%%%%%%%%%%%%%%%%%%%%%%%%%%

% ALT ALT

\end{document}