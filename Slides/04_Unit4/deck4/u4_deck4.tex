% -*- TeX-engine: xetex; eval: (auto-fill-mode 0); eval: (visual-line-mode 1); -*-
% Compile with XeLaTeX

%%%%%%%%%%%%%%%%%%%%%%%
% To do before class
%%%%%%%%%%%%%%%%%%%%%%%

% Send the Logistics/Week0Annoucnement (the night before).
% Send an email reminding students to bring a charged computer (the night before).

%%%%%%%%%%%%%%%%%%%%%%%
% Option 1: Slides: (comment for handouts)   %
%%%%%%%%%%%%%%%%%%%%%%%

\documentclass[slidestop,compress,mathserif,12pt,t,professionalfonts,xcolor=table]{beamer}

% solution stuff
\newcommand{\solnMult}[1]{
\only<1>{#1}
\only<2->{\red{\textbf{#1}}}
}
\newcommand{\soln}[1]{\textit{#1}}

\newcommand{\solnMultOn}[3]{
\only<#1>{#3}
\only<{#2}->{\red{\textbf{#3}}}
}

%%%%%%%%%%%%%%%%%%%%%%%%%%%%%%%
% Option 2: Handouts, without solutions (post before class)    %
%%%%%%%%%%%%%%%%%%%%%%%%%%%%%%%

% \documentclass[11pt,containsverbatim,handout,xcolor=xelatex,dvipsnames,table]{beamer}

% % handout layout
% \usepackage{pgfpages}
% \pgfpagesuselayout{4 on 1}[letterpaper,landscape,border shrink=5mm]

% % solution stuff
% \newcommand{\solnMult}[1]{#1}
% \newcommand{\soln}[1]{}
% \newcommand{\solnMultOn}[3]{#3}

% % % This breaks things for me for some reason.
% % tell pgfpages how to set page sizes in XeLaTeX
% %\renewcommand\pgfsetupphysicalpagesizes{%
% %   \pdfpagewidth\pgfphysicalwidth\pdfpageheight\pgfphysicalheight%
% %}

%%%%%%%%%%%%%%%%%%%%%%%%%%%%%%%%%%%%
% Option 3: Handouts, with solutions (may post after class if need be)    %
%%%%%%%%%%%%%%%%%%%%%%%%%%%%%%%%%%%%

% \documentclass[11pt,containsverbatim,handout,xcolor=xelatex,dvipsnames,table]{beamer}

% % handout layout
% \usepackage{pgfpages}
% \pgfpagesuselayout{4 on 1}[letterpaper,landscape,border shrink=5mm]

% % solution stuff
% \newcommand{\solnMult}[1]{\red{\textbf{#1}}}
% \newcommand{\soln}[1]{\textit{#1}}

% % % This breaks things for me for some reason.
% % % tell pgfpages how to set page sizes in XeLaTeX
% % \renewcommand\pgfsetupphysicalpagesizes{%
% %    \pdfpagewidth\pgfphysicalwidth\pdfpageheight\pgfphysicalheight%
% % }

%%%%%%%%%%%%%%%%%%%%%%%%%%%%%%%
% Option 4: Notes Only
%%%%%%%%%%%%%%%%%%%%%%%%%%%%%%%

% % See http://tex.stackexchange.com/questions/114219/add-notes-to-latex-beamer
% \documentclass[10pt,containsverbatim,xcolor=xelatex,dvipsnames,table,notes=only]{beamer}

% % handout layout
% % \usepackage{pgfpages}
% % \pgfpagesuselayout{1 on 1}[letterpaper, landscape, border shrink=5mm]

% % solution stuff
% \newcommand{\solnMult}[1]{#1}
% \newcommand{\soln}[1]{}

% % % Having a problem with this.
% % tell pgfpages how to set page sizes in XeLaTeX
% % \renewcommand\pgfsetupphysicalpagesizes{%
% %   \pdfpagewidth\pgfphysicalwidth\pdfpageheight\pgfphysicalheight%
% %}

%%%%%%%%%%
% Load style file, defaults  %
%%%%%%%%%%

\input{../../lec_style.tex}
% You cannot use numbers when defining variables.  Hence the use of letters, A, B, C, etc.

% Personal Info
\newcommand{\FirstName}{Mine}
\newcommand{\LastName}{\c{C}etinkaya-Rundel}
\newcommand{\OfficeHours}{MTWR 3-4pm.}
\newcommand{\OfficeHoursLocation}{Old Chem 213}

% Electronic Info
\newcommand{\PersonalSite}{http://stat.duke.edu/~mc301}
\newcommand{\CourseSite}{http://bitly.com/sta101sp15}
\newcommand{\Email}{mine@stat.duke.edu}

% TAs
\newcommand{\TAA}{Anthony Weishampel}
\newcommand{\TAB}{Fiamma Li}
\newcommand{\TAC}{Jialiang Mao}
\newcommand{\TAD}{Phillip Lee}

% Exam Dates
\newcommand{\ExamADate}{Wed, Feb 18}
\newcommand{\ExamBDate}{Wed, Mar 25}
\newcommand{\FinalDate}{Sat, May 2 (2-5pm)}

% Due Dates
\newcommand{\ClickerRegistrationDD}{Mon, Jan 26}
\newcommand{\GettingToKnowYouDD}{Friday, Jan 9, 11:59pm}
\newcommand{\ProblemSetADD}{Wed., 1/15}


% ALT ALT
% You cannot use numbers when defining variables.  Hence the use of letters, A, B, C, etc.

% Personal Info
\renewcommand{\FirstName}{Jesse}
\renewcommand{\LastName}{Windle}
\renewcommand{\OfficeHours}{Tue, Thu 3:00pm-4:30pm}
\renewcommand{\OfficeHoursLocation}{Old Chem 211D}

% Electronic Info
\renewcommand{\PersonalSite}{http://stat.duke.edu/~jbw44/}
\renewcommand{\CourseSite}{http://bitly.com/windle2}
\renewcommand{\Email}{jbw44@stat.duke.edu}

% TAs
\renewcommand{\TAA}{David Clancy}
\renewcommand{\TAB}{Xinyi (Chris) Li}
\renewcommand{\TAC}{Tori Hall}
\renewcommand{\TAD}{Radhika Anand}

% Exam Dates
\renewcommand{\ExamADate}{Thu, Feb 19}
\renewcommand{\ExamBDate}{Thu, Mar 26}
\renewcommand{\FinalDate}{Mon, Apr 27 (9-Noon)}

% Due Dates
\renewcommand{\ClickerRegistrationDD}{Thu, Jan 15}
\renewcommand{\GettingToKnowYouDD}{Friday, Jan 9, 11:59pm}
\renewcommand{\ProblemSetADD}{Thu., 1/16}

%%%%%%%%%%%
% Cover slide info    %
%%%%%%%%%%%

\title{Unit 4: Inference for numerical data}
\subtitle{3. ANOVA}
\author{Sta 101 - Spring 2015}
\date{March 4, 2015}
\institute{Duke University, Department of Statistical Science}

%%%%%%%%%%%
% Main ideas %
%%%%%%%%%%%

\newcommand{\mainideaA}{You can use the $F$-test to compare grouping by 2 variables vs. grouping by 1 variable}

\newcommand{\bonferroni}{If you want to test many hypotheses simulteneously, use
  the Bonferroni correction.}

%%%%%%%%%%%
% Begin document   %
%%%%%%%%%%%

\begin{document}

%%%%%%%%%%%%%%%%%%%%%%%%%%%%%%%%%%%%

% Title Page

\begin{frame}[plain]

\titlepage
\vfill
{\scriptsize \webLink{\PersonalSite}{Dr. \LastName{}} \hfill Slides posted at  \webLink{\CourseSite}{\CourseSite}}
\addtocounter{framenumber}{-1} 

\end{frame}

%%%%%%%%%%%%%%%%%%%%%%%%%%%%%%%%%%%%

\section{Housekeeping}

%%%%%%%%%%%%%%%%%%%%%%%%%%%%%%%%%%%%

\begin{frame}
\frametitle{Announcements}

\begin{itemize}

\item 

\end{itemize}

\end{frame}

%%%%%%%%%%%%%%%%%%%%%%%%%%%%%%%%%%%%

\section{Main Ideas}

%%%%%%%%%%%%%%%%%%%%%%%%%%%%%%%%%%%%

\subsection{\bonferroni}
\label{mi1}

%%%%%%%%%%%%%%%%%%%%%%%%%%%%%%

\begin{frame}
  \frametitle{Last time}

  \vfill

  \url{http://imgs.xkcd.com/comics/significant.png}

  \vfill

\end{frame}

%%%%%%%%%%%%%%%%%%%%%%%%%%%%%%

\begin{frame}
  \frametitle{Last time}

  \textbf{Suppose} $\alpha = 0.05$.

  \disc{What is the probability of correctly failing to reject
    \[
    H_0: \mu_{\textmd{purple}} = \mu_{\textmd{placebo}} \; ?
    \]
    }

    \pause

    \hfill \\

    \clicker{If all the tests are independent and if no color of Jelly bean has any link to acne, what is the
      probability of making at least one type I error in the 20 trials?}
      \begin{enumerate}[(a)]
      \item 5\%
      \item 36\%
      \item \solnMultOn{2}{3}{64\%}
      \item 95\%
      \end{enumerate}

\end{frame}

%%%%%%%%%%%%%%%%%%%%%%%%%%%%%%

\begin{frame}
  \frametitle{Last Time}

Is some jelly bean linked to acne?  Use the $F$-test.

\begin{itemize}
\item $H_0$: the placebo and all of the jelly beans have the same group means.

\item $H_A$: at least one of the group's has a mean that is different from the others.
\end{itemize}

\clicker{For the $F$-test, what is the probability of incorrectly rejecting the null?}
\begin{enumerate}[(a)]
\item \solnMult{5\%}
\item 36\%
\item 64\%
\item 95\%
\end{enumerate}

\end{frame}

%%%%%%%%%%%%%%%%%%%%%%%%%%%%%%

\begin{frame}
  \frametitle{The Bonferroni correction}

  \vfill 

  \disc{How do we figure out \emph{which} jelly bean(s) is(are) linked to acne?}

  \begin{center}
  The Bonferroni correction.
  \end{center}

  \vfill

\end{frame}

%%%%%%%%%%%%%%%%%%%%%%%%%%%%%%

\begin{frame}
  \frametitle{1. \bonferroni}

\vfill

Bonferroni correction: 
\begin{itemize}
\item Target type I error rate: $\alpha$.

\item Number of null/alt hypotheses to be tested using the same data set: $K$

\item If you set the significance level for each test to be
\[
\alpha^* = \alpha / K,
\]
then the probability of making one or more type I errors is $ \leq \alpha$.

\end{itemize}

\vfill

\end{frame}

%%%%%%%%%%%%%%%%%%%%%%%%%%%%%%

\begin{frame}
  \frametitle{1. \bonferroni}

\vspace{-0.5cm}
{\small
\begin{align*}
01.\ &  \; \; H_0: \; \mu_{\text{placebo}} = \mu_{\text{purple}} \\
     &  \; \; H_A: \; \mu_{\text{placebo}} \neq \mu_{\text{purple}}  \\
02.\ &  \; \; H_0: \; \mu_{\text{placebo}} = \mu_{\text{brown}} \\
     &  \; \; H_A: \; \mu_{\text{placebo}} \neq \mu_{\text{brown}}  \\
& \ldots \\
% 19.\ &  \; \; H_0: \; \mu_{\text{placebo}} = \mu_{\text{peach}} \\
%      &  \; \; H_A: \; \mu_{\text{placebo}} \neq \mu_{\text{peach}}  \\
20.\ &  \; \; H_0: \; \mu_{\text{placebo}} = \mu_{\text{orange}} \\
     &  \; \; H_A: \; \mu_{\text{placebo}} \neq \mu_{\text{orange}} 
\end{align*}
}

\clicker{What significance level shoud we use to test all of these hypotheses simultaneously for a type I error rate of 5\%?}
\begin{center}
(a) 0.0024 \; (b) \solnMult{0.0025} (c) 0.0026 \; (d) 0.05
\end{center}

\end{frame}

%%%%%%%%%%%%%%%%%%%%%%%%%%%%%%

\begin{frame}
  \frametitle{1. \bonferroni}

\vfill

\app{4.5 ANOVA - Pt 2}{See the course webpage for details.}

\vfill

\end{frame}


%%%%%%%%%%%%%%%%%%%%%%%%%%%%%%%%%%%%

\subsection{\mainideaA}
\label{mi1}

%%%%%%%%%%%%%%%%%%%%%%%%%%%%%%%%%%%%

\begin{frame}
  \frametitle{2. \mainideaA}

  Cartoon for incorporating cholcolate.

  Jelly beans are associated with Acne.
  OK.
  I hear chocholate makes it works.

\end{frame}

%%%%%%%%%%%%%%%%%%%%%%%%%%%%%%%%%%%%

\begin{frame}
  \frametitle{2. \mainideaA}

  Treatments:

  \begin{center}
  \begin{tabular}{l l}
    purple \& choc.\ & purple \& no choc. \\
    brown \& choc.\ & brown \& no choc. \\
    \ldots & \ldots \\
    orange \& chol.\ & orange \& no choc.
    \end{tabular}
    \end{center}

    \clicker{We want to test if eating chocolate and a certain color jelly bean has a different impact than just eating that color jelly bean.  Which type of null hypothesis should we use?}
    \begin{enumerate}[(a)]
      \item $H_0$: $\mu_{\textmd{color \& choc.}} = \mu_{\textmd{color \& no choc.}}$ for every color.
      \item $H_0$: $\mu_{\textmd{color A \& choc.}} = \mu_{\textmd{color B \& choc.}}$ for every pair of colors.
      \end{enumerate}

\end{frame}

%%%%%%%%%%%%%%%%%%%%%%%%%%%%%%%%%%%%

\begin{frame}
  \frametitle{2. \mainideaA}

Picture of within group variation for different scenarios.

\end{frame}

%%%%%%%%%%%%%%%%%%%%%%%%%%%%%%%%%%%%

\begin{frame}
  \frametitle{2. \mainideaA}

  $F$-test in terms of difference of within group variation.

\end{frame}

%%%%%%%%%%%%%%%%%%%%%%%%%%%%%%%%%%%%

\begin{frame}
  \frametitle{2. \mainideaA}

  App Ex.

\end{frame}

%%%%%%%%%%%%%%%%%%%%%%%%%%%%%%%%%%

\section{Summary}

%%%%%%%%%%%%%%%%%%%%%%%%%%%%%%%%%%%%

\begin{frame}
\frametitle{Summary of main ideas}

\vfill

\begin{enumerate}

\item \nameref{mi1}

\end{enumerate}

\vfill

\end{frame}

%%%%%%%%%%%%%%%%%%%%%%%%%%%%%%%%%%%

\end{document}