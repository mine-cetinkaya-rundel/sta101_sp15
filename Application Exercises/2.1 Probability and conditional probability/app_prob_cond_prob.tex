\documentclass[11pt]{article}
\input{../app_style.tex}

%%%%%%%%%%%%%%%%
% Timing
%%%%%%%%%%%%%%%%

% 12-15 minutes

%%%%%%%%%%%%%%%%
% Sakai link for course
%%%%%%%%%%%%%%%%

% UPDATE FOR OWN COURSE
% LINK TO ASSIGNMENTS TOOL IN SAKAI

\newcommand{\Sakai}[1]
{\href{https://sakai.duke.edu/portal/site/ba0d1c18-ba55-473f-9d70-b6a1f9559bbe/page/9870858b-a1a9-481e-8497-8a6ffe9e5be2}{Sakai}}

%%%%%%%%%%%
% App Ex number    %
%%%%%%%%%%%

% DON'T FORGET TO UPDATE

\newcommand{\appno}[1]
{2.1}

%%%%%%%%%%%%%%
% Turn on/off solutions       %
%%%%%%%%%%%%%%

% Off
\newcommand{\soln}[1]{
\vskip5pt
}

%% On
%\newcommand{\soln}[1]{
%\textit{\textcolor{custom_darkGray}{#1}}
%}

%%%%%%%%%%%%%%%%
% Document
%%%%%%%%%%%%%%%%

\begin{document}
\fontspec[Ligatures=TeX]{Helvetica Neue Light}

Dr. \c{C}etinkaya-Rundel \hfill Data Analysis and Statistical Inference \\

\ttl{Application exercise \appno{}: \\
Probability and conditional probability}

\inst{Submit your responses on \Sakai{}, under the appropriate assignment. Only one submission per team is required. One team will be randomly selected and their responses will be discussed.}

%%%%%%%%%%%%%%%%%%%%%%%%%%%%%%%%%%%%

The following table shows the distribution of class year and whether or not students voted in the last presidential election for 176 Sta 101 students. \\

\begin{center}
\begin{tabular}{rrrr|r}
  \hline
 & no, eligible but didn't & no, not eligible & yes & total \\ 
  \hline
first-year & 3 & 38 & 3 & 44 \\ 
  sophomore & 10 & 40 & 14 & 64 \\ 
  junior & 7 & 6 & 41 & 54 \\ 
  senior & 4 & 1 & 9 & 14 \\ 
  \hline
  total & 24 & 85 & 67 & 176 \\ 
   \hline
\end{tabular}
\end{center}

Answer the following questions based on these data. Make sure to show all your work. \\

\begin{enumerate}

\item  What is the probability that a randomly chosen student has voted in the last presidential election?

\item What is the probability that a randomly chosen student is a junior \underline{and} has voted in the last presidential election?

\item What is the probability that a randomly chosen student is a junior \underline{given that} s/he has voted in the last presidential election?

\item Categorize the three probabilities you calculated above as marginal, conditional, or joint.

\item What is the probability that a randomly chosen student is a junior \underline{or} has voted in the last presidential election?

\item What percent of students are junior \underline{or} have voted in the last presidential election?

\item What is the probability that a randomly chosen student is a first year \underline{given that} s/he has voted in the last presidential election? What about sophomore, and senior?

\item Do these data suggest an association between class year and whether or not students have voted in the last presidential election? Explain your reasoning in one or two sentences.

\end{enumerate}

%%%%%%%%%%%%%%%%%%%%%%%%%%%%%%%%%%%%

\end{document}