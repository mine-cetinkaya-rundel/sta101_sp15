\documentclass[11pt]{article}
\usepackage[top=1.5cm,bottom=1.5cm,left=1.5cm,right= 1.5cm]{geometry}
%\geometry{landscape}                % Activate for for rotated page geometry
\usepackage[parfill]{parskip}    % Activate to begin paragraphs with an empty line rather than an indent
\usepackage{graphicx}
\usepackage{amssymb}
\usepackage{epstopdf}
\usepackage{setspace}            
\usepackage{amsmath}            
\usepackage{multirow}    
\usepackage{changepage}
\usepackage{lscape}
\usepackage{ulem}
\usepackage{multicol}
\usepackage{dashrule}
\usepackage[usenames,dvipsnames]{color}       
\usepackage{enumerate}
\newcommand{\urlwofont}[1]{\urlstyle{same}\url{#1}}
\newcommand{\degree}{\ensuremath{^\circ}}

\DeclareGraphicsRule{.tif}{png}{.png}{`convert #1 `dirname #1`/`basename #1 .tif`.png}

\newenvironment{choices}{
\begin{enumerate}[(a)]
}{\end{enumerate}}

\pagestyle{empty}

%\newcommand{\soln}[1]{\textcolor{MidnightBlue}{\textit{#1}}}	% delete #1 to get rid of solutions for handouts
\newcommand{\soln}[1]{ \vspace{2.7cm} }

\newcommand{\solnMult}[1]{\textbf{\textcolor{MidnightBlue}{\textit{#1}}}}	% uncomment for solutions
%\newcommand{\solnMult}[1]{ #1 }	% uncomment for handouts

%\newcommand{\pts}[1]{ \textbf{{\footnotesize \textcolor{black}{(#1)}}} }	% uncomment for handouts
\newcommand{\pts}[1]{ \textbf{{\footnotesize \textcolor{red}{(#1)}}} }	% uncomment for handouts

\newcommand{\note}[1]{ \textbf{\textcolor{red}{[#1]}} }	% uncomment for handouts

\definecolor{oiG}{rgb}{.298,.447,.114}
\definecolor{oiB}{rgb}{.337,.608,.741}

\usepackage[colorlinks=false,pdfborder={0 0 0},urlcolor= oiG,colorlinks=true,linkcolor= oiG, citecolor= oiG,backref=true]{hyperref}

%\usepackage{draftwatermark}
%\SetWatermarkScale{4}

\usepackage{titlesec}
\titleformat{\section}
{\color{oiB}\normalfont\Large\bfseries}
{\color{oiB}\thesection}{1em}{}
\titleformat{\subsection}
{\color{oiB}\normalfont}
{\color{oiB}\thesubsection}{1em}{}

\newcommand{\ttl}[1]{ \textsc{{\LARGE \textbf{{\color{oiB} #1} } }}}

\newcommand{\tl}[1]{ \textsc{{\large \textbf{{\color{oiB} #1} } }}}

\begin{document}

Dr. \c{C}etinkaya-Rundel \hfill Data Analysis and Statistical Inference \\

\ttl{Application exercise: 5.3 \\
Chi-square tests, Pt. 2}

\section*{Football and politics in NC}

Public Policy Polling surveyed 780 likely voters from October 16th to 18th, 2014. The full report on the poll results can be found at \url{http://www.publicpolicypolling.com/pdf/2014/PPP_Release_NC_10201118.pdf}. We are going to focus on responses to the following questions:

\begin{itemize}
\item Which is your favorite college football team in North Carolina?
\item In the last presidential election, did you vote for Barack Obama or Mitt Romney?
\end{itemize}

%
Below is the distribution of the responses are as follows:
\begin{center}
\begin{tabular}{r | l | l | l | l}
				& Barack Obama	& Mitt Romney	& Other / Don't remember	& Total \\
\hline
Duke			& 53				& 69			& 7					& 129 \\
UNC				& 95				& 76			& 8					& 179 \\
NC State			& 35				& 54			& 8					& 97 \\
Other / no favorite	& 168			& 183		& 24					& 375 \\
\hline
Total				& 351			& 382		& 47					& 780 \\
\end{tabular}
\end{center}

$\:$ \\
We want to evaluate whether college football allegiance of NC residents is associated with how they voted in the 2012 presidential election.
$\:$ \\
\begin{enumerate}

\item What type of test is most appropriate? Explain your reasoning.

\item What are the hypotheses?

\item Calculate the expected counts for cells with low observed counts (Other / Don't remember and Duke, UNC, or NC State).

\item Assuming that all other expected counts are high enough, check if the conditions for inference are met.

\item Calculate the contribution of the Duke and Other / Don't remember cell to the test statistic.

\item The test statistic is given as 9.4063. Calculate the p-value.

\item What is the conclusion of the hypothesis test?

\end{enumerate}

This analysis completed in R can be found at \url{https://stat.duke.edu/courses/Fall14/sta101.001/app/football_politics.html}.

%

\end{document}