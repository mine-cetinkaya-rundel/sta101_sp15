\documentclass[11pt]{article}
\usepackage[top=1.5cm,bottom=1.5cm,left=1.5cm,right= 1.5cm]{geometry}
%\geometry{landscape}                % Activate for for rotated page geometry
\usepackage[parfill]{parskip}    % Activate to begin paragraphs with an empty line rather than an indent
\usepackage{graphicx}
\usepackage{amssymb}
\usepackage{epstopdf}
\usepackage{setspace}            
\usepackage{amsmath}            
\usepackage{multirow}    
\usepackage{changepage}
\usepackage{lscape}
\usepackage{ulem}
\usepackage{multicol}
\usepackage{dashrule}
\usepackage[usenames,dvipsnames]{color}       
\usepackage{enumerate}
\newcommand{\urlwofont}[1]{\urlstyle{same}\url{#1}}
\newcommand{\degree}{\ensuremath{^\circ}}

\DeclareGraphicsRule{.tif}{png}{.png}{`convert #1 `dirname #1`/`basename #1 .tif`.png}

\newenvironment{choices}{
\begin{enumerate}[(a)]
}{\end{enumerate}}

\pagestyle{empty}

%\newcommand{\soln}[1]{\textcolor{MidnightBlue}{\textit{#1}}}	% delete #1 to get rid of solutions for handouts
\newcommand{\soln}[1]{ \vspace{2.7cm} }

\newcommand{\solnMult}[1]{\textbf{\textcolor{MidnightBlue}{\textit{#1}}}}	% uncomment for solutions
%\newcommand{\solnMult}[1]{ #1 }	% uncomment for handouts

%\newcommand{\pts}[1]{ \textbf{{\footnotesize \textcolor{black}{(#1)}}} }	% uncomment for handouts
\newcommand{\pts}[1]{ \textbf{{\footnotesize \textcolor{red}{(#1)}}} }	% uncomment for handouts

\newcommand{\note}[1]{ \textbf{\textcolor{red}{[#1]}} }	% uncomment for handouts

\definecolor{oiG}{rgb}{.298,.447,.114}
\definecolor{oiB}{rgb}{.337,.608,.741}

\usepackage[colorlinks=false,pdfborder={0 0 0},urlcolor= oiG,colorlinks=true,linkcolor= oiG, citecolor= oiG,backref=true]{hyperref}

%\usepackage{draftwatermark}
%\SetWatermarkScale{4}

\usepackage{titlesec}
\titleformat{\section}
{\color{oiB}\normalfont\Large\bfseries}
{\color{oiB}\thesection}{1em}{}
\titleformat{\subsection}
{\color{oiB}\normalfont}
{\color{oiB}\thesubsection}{1em}{}

\newcommand{\ttl}[1]{ \textsc{{\LARGE \textbf{{\color{oiB} #1} } }}}

\newcommand{\tl}[1]{ \textsc{{\large \textbf{{\color{oiB} #1} } }}}

\begin{document}

Dr. \c{C}etinkaya-Rundel \hfill Data Analysis and Statistical Inference \\

\ttl{Application exercise: 5.1 \\
Inference for comparing two proportions}

\section*{Raise teacher pay vs. cut taxes}

A 2014 Public Policy Polling survey asked NC residents whether they would prefer that the General Assembly raises teacher pay or cuts taxes. Overall it was found that 54\% of respondents preferred raising teacher pay, 36\% prefer cutting taxes, and 10\% were not sure. 

Some of these interviews were conducted over the phone, and others (for those without a landline) were conducted on the internet. Ideally we would like to see no significant difference based on whether the interviews were conducted over the phone or on the internet.

Findings of the report can be found at \url{http://www.publicpolicypolling.com/pdf/2014/PPP_Release_NC_514.pdf}.

\begin{enumerate}

\item How many NC residents were polled as part of this survey?

\item How many of them were polled via phone vs. on the internet?

\item State the distribution of responses to the question on raising teacher pay vs. cutting taxes by mode of delivery of the survey. \textit{Hint:} All you need to do is to find the corresponding table on the report.

\item Conduct a hypothesis test evaluating whether opinion on raising teacher pay varies by mode of delivery (phone / internet). \textit{Hint:} Make sure to state the hypotheses, check conditions, calculate the appropriate test statistic, and the p-value.

\item If you find a significant difference between these proportions, look through the remainder of the findings from the report to try to explain what might be causing this difference?

\end{enumerate}

%

\end{document}